%-*- coding:UTF-8 -*-
% dna-align.tex
\documentclass[hyperref,UTF8]{ctexart}
\usepackage{geometry}
\usepackage{enumerate}
\usepackage{amsmath}
\usepackage{amssymb}
\usepackage{dsfont}
\usepackage{amsthm}
\usepackage{listings} %插入代码
\usepackage{xcolor} %代码高亮
\usepackage{blkarray}
\usepackage{diagbox}
\usepackage{tabularx}
\usepackage{graphicx}
\usepackage{caption}
\usepackage{subcaption}
\usepackage{float}
\usepackage{color}
\usepackage{multirow}
\usepackage[all,pdf]{xy}
\usepackage{verbatim}   %comment
\usepackage{cases}
\usepackage{clrscode3e}	% need clrscode3e package which is not included in CTex.
\usepackage{hyperref}

\geometry{screen}
\hypersetup{
    colorlinks=true,
    bookmarks=true,
    bookmarksopen=false,
    pdfpagemode=FullScreen,
    pdfstartview=fit,
    pdftitle={DNA-Align},
    pdfauthor={Trasier}
}

% THEOREM Environments --------------------------------------------------------
\newtheorem{thm}{Theorem}[subsection]
\newtheorem{cor}[thm]{Corollary}
\newtheorem{lem}[thm]{Lemma}
\newtheorem{prop}[thm]{Proposition}
\newtheorem{prob}[thm]{Problem}
\newtheorem{mthm}[thm]{Main Theorem}
\theoremstyle{definition}
\newtheorem{defn}[thm]{Definition}
\theoremstyle{remark}
\newtheorem{rem}[thm]{Remark}
\numberwithin{equation}{subsection}
% MATH ------------------------------------------------------------------------
\DeclareMathOperator{\RE}{Re}
\DeclareMathOperator{\IM}{Im}
\DeclareMathOperator{\ess}{ess}
\newcommand{\eps}{\varepsilon}
%\newcommand{\To}{\longrightarrow}  conflict with \package{clrscode3e}p
\newcommand{\h}{\mathcal{H}}
\newcommand{\s}{\mathcal{S}}
\newcommand{\A}{\mathcal{A}}
\newcommand{\J}{\mathcal{J}}
\newcommand{\M}{\mathcal{M}}
\newcommand{\W}{\mathcal{W}}
\newcommand{\X}{\mathcal{X}}
\newcommand{\BOP}{\mathbf{B}}
\newcommand{\BH}{\mathbf{B}(\mathcal{H})}
\newcommand{\KH}{\mathcal{K}(\mathcal{H})}
\newcommand{\Real}{\mathbb{R}}
\newcommand{\Complex}{\mathbb{C}}
\newcommand{\Field}{\mathbb{F}}
\newcommand{\RPlus}{\Real^{+}}
\newcommand{\Polar}{\mathcal{P}_{\s}}
\newcommand{\Poly}{\mathcal{P}(E)}
\newcommand{\EssD}{\mathcal{D}}
\newcommand{\Lom}{\mathcal{L}}
\newcommand{\States}{\mathcal{T}}
\newcommand{\abs}[1]{\left\vert#1\right\vert}
\newcommand{\set}[1]{\left\{#1\right\}}
\newcommand{\seq}[1]{\left<#1\right>}
\newcommand{\norm}[1]{\left\Vert#1\right\Vert}
\newcommand{\essnorm}[1]{\norm{#1}_{\ess}}


% Some setup
\pagestyle{plain}
\geometry{a4paper, top=2cm, bottom=2cm, left=2cm, right=2cm}
\CTEXsetup[format={\raggedright\bfseries\Large}]{section}
\lstset{numbers=left, %设置行号位置
        numberstyle=\small, %设置行号大小
        keywordstyle=\color{blue}, %设置关键字颜色
        commentstyle=\color{purple}, %设置注释颜色
        %frame=single, %设置边框格式
        escapeinside=``, %逃逸字符(1左面的键),用于显示中文
        breaklines, %自动折行
        extendedchars=false, %解决代码跨页时,章节标题,页眉等汉字不显示的问题
        %xleftmargin=2em,xrightmargin=2em, aboveskip=1em, %设置边距
        tabsize=4, %设置tab空格数
        showspaces=false %不显示空格
       }

% About math
\newcommand{\rmnum}[1]{\romannumeral #1}
\newcommand{\Emph}{\textbf}
\newcolumntype{Y}{>{\centering\arraybackslash}X}
\newcommand{\resetcounter}{\setcounter{equation}{0}}
\newcommand{\equsuf}[1][x]{\equiv_{\textit{Suff(#1)}}}	
\newcommand{\Suff}{\textit{Suff}}
\newcommand{\len}[1][x]{\textit{length}_{#1}}

% section deeep to 3 1.1.1
\setcounter{secnumdepth}{3}

\begin{document}

\title{\Huge DNA-Align}
\vspace{2cm}
\author{\Large Trasier}
\date{\today}
\maketitle

\section{Problem Statement}
\label{sec:problem_statement}
	
	There are two major challenges within the alignment problem that we hope you will address in this competition:
	\begin{enumerate}
	
		\item \Emph{A Read May Not Perfectly Match the Reference.}
		A fraction of the reads will have no exact substring match in the reference due to genetic
		variation and sequencing errors.
		Although it is possible to obtain the exact alignment of two strings via the
        \href{https://en.wikipedia.org/wiki/Smith-Waterman_algorithm}{Smith-Waterman} or
        \href{https://en.wikipedia.org/wiki/Needleman-Wunsch_algorithm}{Needleman-Wunsch}
		algorithms, this procedure runs in time proportional to the product of the strings' lengths;
		In theory, this could be done for every read, but would be computationally infeasible.
		
		\Emph{There is no known efficient solution to inexact substring matching.}
		
		\item \Emph{Matching May be Ambiguous.}
		Portitions of the genome are highly-repretitive.
		In extreme cases, thousands of characters may be repeated thousands of times.
		It is impossible for 150 character reads to unambiguously map to these regions,
		but is is very important to know that the reported match may not be accurate.
		
		\Emph{The aligned must accurately measure the ambiguity of each mapping and report via a mapping confidence score.}
		
	\end{enumerate}

	
\section{Data Description}
	
	For the competition, submissions will be provided with same reference genome and new set of simulated reads.
	Each submission will undergo up to threee competition tests with increasing difficulty,
	with access to more difficult tests contingent on adequate performance on easier tests.
	The tests differ in complexity by the number of reads to be aligned and
	the number of chromatids the reads are to be aligned to.
	
	\begin{enumerate}
	
		\item \Emph{Small test}
		
		Chromosome 20($6.4\times10^7$ characters), up to $10^4$ read pairs
		
		\item \Emph{Medium test}
		
		Chromosomes $1, 11, 20$($4.5\times10^8$), up to $10^6$ read pairs.
		
		\item \Emph{Large test}
		
		Chromosomes $1, 2, \cdots, 24$($3.2\times10^9$), up to $10^7$ read pairs.
		
	\end{enumerate}
	
\section{Functions}
	
	The example and provisional test case will contain two tests with different difficuly(small, medium)
	while the system test case will contain three tests(small, medium large).
	The DNASequencing class won't be reinstantiated between the small and medium tests.
	For the large system test, solutions will be run stand-alone outside of the normal submission
	system on VMs with similar characteristics to the test processors.
	
	This set of functions will be called at most 2 times from smaller test to bigger ones.
	For large system tests, it will only be called once.
	Functions call order for a test within a testcase as follows:
	\begin{enumerate}
	
		\item --- Preprocessing time limit starts counting here
		
		\item int \proc{initTest}(int testDifficulty)
		
		testDifficulty can be 0: small, 1: medium, 2: large. Return can be any valid integer.
		
		\item int \proc{passReferenceGenome}(int chromatidSequenceId, string[] chromatidSequence)
		
		will be called 1, 3 or 24 times according to the test difficuly. Return can be any valid integer.
		
		\item int \proc{preProcessing}()
		
		Return can be any valid integer.
		
		\item --- Preprocessing time limit ends here
			  --- Time Cutoff starts counting here (the time used for scoring)
			
		\item string[] \proc{getAlignment}(int N, double normA, double normS, string[] readName, string[] readSequence)
		
		will be called once per test case.
		
		N is the number of reads in the test.
		Both array will contain exactly N elements as well as the retrun array.
		Elements at (2*i) and (2*i+1) are representing a read pair.
		
		Format of the return string:
		
		ReadName,\ ChromatidSequenceId,\ Start position(1-based),\ End position(1-based), strand(reference [+] or reverse complement [-]), mapping confidence score	\\
		"sim1/1,1,10000,10149,+,0.89"
		
		\item --- Time Cutoff ends here
			
	\end{enumerate}

\section{Notes}	
\label{sec:notes}

	\begin{itemize}
	
		\item Memory limit is 16GB
		
		\item Test processors are AWS m3.2xlarge instances
		
		\item There is no explicit code size limit. The implicit source code size limit is around 1MB
		(it is not advisable to submit codes of size close to that or large). The compilation time limit is 60 seconds.
		
		\item You will be allowed to make example submissions each \Emph{30 minutes} and provisional submissions \Emph{each 3 hours}.
		
	\end{itemize}
	
	
\section{Observation}
\label{sec:observation}
	
	\begin{enumerate}
		
		\item \Emph{Ground truth}中的read对是否一个为[+],另一个为[-]
		
		结果是成对出现的

		\item \Emph{Ground truth}中的$end\_position - start\_position$的长度是否一定为150
		
		不一定,结果在$[110, 348]$之间

		\item \Emph{Ground truth}中的read对的间距情况
		
		结果在$[171, 542]$之间
	
		\item $[-]$是否表示原串的$\Emph{Reverse Complement}$
		
		确实是
				
		\item 染色体的中间是否含有未定义字符
		
		\Emph{smallCase}不含有

		\item \Emph{Ground truth}中read的得分是多少,是否可以找到阈值
		
		显然存在\kw{insert}和\kw{delete},导致计分有问题
		
		\item 统计子串频度
		
		暂时没统计
	
	\end{enumerate}
	
\section{Thought}
\label{sec:thought}
		
	\begin{enumerate}
	
		\item 如何表示染色体
		
		\begin{enumerate}[(1)]
		
			\item 完整的字符串
			
			\item 使用子串重排序列
			
		\end{enumerate}
		
		\item 如何剪枝
		
		\begin{enumerate}[(1)]
		
			\item 根据染色体的$AGCT$数量与当前基因片的$AGCT$数量的\Emph{余弦相似度}
			
			\item 根据子序列的$AGCT$数量与当前基因片的$AGCT$数量的\Emph{欧式距离}
			
			\item 根据子串的分布于当前基因片的子串的分布
		
		\end{enumerate}
		
		\item 如何计算\Emph{Confidence}
		
		\item 如何利用$|truth_st-st| \ge 300$
		
		\item 如何提高空间利用率
	
	\end{enumerate}
	
\end{document}

