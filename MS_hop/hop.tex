%-*- coding:UTF-8 -*-
% MS-bop2016.tex
\documentclass[hyperref,UTF8]{ctexart}
\usepackage{geometry}
\usepackage{enumerate}
\usepackage{amsmath}
\usepackage{amssymb}
\usepackage{dsfont}
\usepackage{amsthm}
\usepackage{listings} %插入代码
\usepackage{xcolor} %代码高亮
\usepackage{blkarray}
\usepackage{diagbox}
\usepackage{tabularx}
\usepackage{graphicx}
\usepackage{caption}
\usepackage{subcaption}
\usepackage{float}
\usepackage{color}
\usepackage{multirow}
\usepackage[all,pdf]{xy}
\usepackage{verbatim}   %comment
\usepackage{cases}
\usepackage{clrscode3e}	% need clrscode3e package which is not included in CTex.
\usepackage{hyperref}

\geometry{screen}
\hypersetup{
    colorlinks=true,
    bookmarks=true,
    bookmarksopen=false,
    %pdfpagemode=FullScreen,
    pdfstartview=fit,
    pdftitle={DNA-Align},
    pdfauthor={Trasier}
}

% THEOREM Environments --------------------------------------------------------
\newtheorem{thm}{Theorem}[subsection]
\newtheorem{cor}[thm]{Corollary}
\newtheorem{lem}[thm]{Lemma}
\newtheorem{prop}[thm]{Proposition}
\newtheorem{prob}[thm]{Problem}
\newtheorem{mthm}[thm]{Main Theorem}
\theoremstyle{definition}
\newtheorem{defn}[thm]{Definition}
\theoremstyle{remark}
\newtheorem{rem}[thm]{Remark}
\numberwithin{equation}{subsection}
% MATH ------------------------------------------------------------------------
\DeclareMathOperator{\RE}{Re}
\DeclareMathOperator{\IM}{Im}
\DeclareMathOperator{\ess}{ess}
\newcommand{\eps}{\varepsilon}
%\newcommand{\To}{\longrightarrow}  conflict with \package{clrscode3e}p
\newcommand{\h}{\mathcal{H}}
\newcommand{\s}{\mathcal{S}}
\newcommand{\A}{\mathcal{A}}
\newcommand{\J}{\mathcal{J}}
\newcommand{\M}{\mathcal{M}}
\newcommand{\W}{\mathcal{W}}
\newcommand{\X}{\mathcal{X}}
\newcommand{\BOP}{\mathbf{B}}
\newcommand{\BH}{\mathbf{B}(\mathcal{H})}
\newcommand{\KH}{\mathcal{K}(\mathcal{H})}
\newcommand{\Real}{\mathbb{R}}
\newcommand{\Complex}{\mathbb{C}}
\newcommand{\Field}{\mathbb{F}}
\newcommand{\RPlus}{\Real^{+}}
\newcommand{\Polar}{\mathcal{P}_{\s}}
\newcommand{\Poly}{\mathcal{P}(E)}
\newcommand{\EssD}{\mathcal{D}}
\newcommand{\Lom}{\mathcal{L}}
\newcommand{\States}{\mathcal{T}}
\newcommand{\abs}[1]{\left\vert#1\right\vert}
\newcommand{\set}[1]{\left\{#1\right\}}
\newcommand{\seq}[1]{\left<#1\right>}
\newcommand{\norm}[1]{\left\Vert#1\right\Vert}
\newcommand{\essnorm}[1]{\norm{#1}_{\ess}}


% Some setup
\pagestyle{plain}
\geometry{a4paper, top=2cm, bottom=2cm, left=2cm, right=2cm}
\CTEXsetup[format={\raggedright\bfseries\Large}]{section}
\lstset{numbers=left, %设置行号位置
        numberstyle=\small, %设置行号大小
        keywordstyle=\color{blue}, %设置关键字颜色
        commentstyle=\color{purple}, %设置注释颜色
        %frame=single, %设置边框格式
        escapeinside=``, %逃逸字符(1左面的键),用于显示中文
        breaklines, %自动折行
        extendedchars=false, %解决代码跨页时,章节标题,页眉等汉字不显示的问题
        %xleftmargin=2em,xrightmargin=2em, aboveskip=1em, %设置边距
        tabsize=4, %设置tab空格数
        showspaces=false %不显示空格
       }

% About math
\newcommand{\rmnum}[1]{\romannumeral #1}
\newcommand{\Emph}{\textbf}
\newcolumntype{Y}{>{\centering\arraybackslash}X}
\newcommand{\resetcounter}{\setcounter{equation}{0}}
\newcommand{\equsuf}[1][x]{\equiv_{\textit{Suff(#1)}}}	
\newcommand{\Suff}{\textit{Suff}}
\newcommand{\len}[1][x]{\textit{length}_{#1}}
\newcommand{\Evaluate}{\Emph{Evaluate}}
\newcommand{\Id}{\Emph{Id}}
\newcommand{\RId}{\Emph{RId}}
\newcommand{\FFId}{\Emph{F.FId}}
\newcommand{\CCId}{\Emph{C.CId}}
\newcommand{\JJId}{\Emph{J.JId}}
\newcommand{\AAAuId}{\Emph{AA.AuId}}
\newcommand{\AAAfId}{\Emph{AA.AfId}}


% section deeep to 3 1.1.1
\setcounter{secnumdepth}{3}

\begin{document}

\title{\Huge MS-bop2016}
\vspace{2cm}
\author{\Large Trasier}
\date{\today}
\maketitle

\section{Problem Statement}
\label{sec:problem_statement}
	
	Microsoft Academic Graph(MAG) is a large heterogeneous graph containing entities such as
	authors, papers, journals, conferences and relations between them.
	The Entity attributes are defined
	\href{https://www.microsoft.com/cognitive-services/en-us/academic-knowledge-api/documentation/entityattributes}{here}.
	Participants are supposed to provide a \Emph{REST} service endpoint that can find all the
	\Emph{1-hop}, \Emph{2-hop}, and \Emph{3-hop} graph paths connecting a given pair of entity identifiers in MAG.
	The given pair of entity identifiers could be
	\begin{itemize}
		\item $[\kw{Id},\quad \kw{Id}]$
		\item $[\kw{Id},\quad \kw{AA.AuId}]$
		\item $[\kw{AA.AuId},\quad \kw{Id}]$
		\item $[\kw{AA.AuId},\quad \kw{AA.AuId}]$
	\end{itemize}
	Each node of a graph should be one of the following identifiers:
	\begin{itemize}
		\item \kw{Id}
		\item \kw{F.FId}
		\item \kw{J.Jid}
		\item \kw{C.CId}
		\item \kw{AA.AuId}
		\item \kw{AA.AfId}
	\end{itemize}
	Possible edges (a pair of adjacent nodes) of a path are:
	\begin{enumerate}[(1)]
		
		\item $Id_1 \rightarrow Id_2$ if and only if the \kw{RId} of the paper
		with $Id_1$ is equal to $Id_2$.
		
		\item $Id_1 \rightarrow F.FId_2$ if and only if the \kw{F.FId} of the paper
		with $Id_1$ is equal to $F.FId_2$.
		
		\item $F.FId_1 \rightarrow Id_2$ if and only if the \kw{F.FId} of the paper
		with $Id_2$ is equal to $F.FId_1$.
		
		\item $Id_1 \rightarrow C.CId_2$ if and only if the \kw{C.CId} of the paper
		with $Id_1$ is equal to $C.CId_2$.
		
		\item $C.CId_1 \rightarrow Id_2$ if and only if the \kw{C.CId} of the paper
		with $Id_2$ is equal to $C.CId_1$.
		
		\item $Id_1 \rightarrow J.JId_2$ if and only if the \kw{J.JId} of the paper
		with $Id_1$ is equal to $J.JId_2$.
		
		\item $J.JId_1 \rightarrow Id_2$ if and only if the \kw{J.JId} of the paper
		with $Id_2$ is equal to $J.JId_1$.
		
		\item $AA.AuId_1 \rightarrow AA.AFId_2$ if and only if the \kw{AA.AFId} of the author
		with $AA.AuId_1$ is equal to $AA.AFId_2$.
		
		\item $AA.AFId_1 \rightarrow AA.AuId_1$ if and only if the \kw{AA.AFId} of the author
		with $AA.AuId_2$ is equal to $AA.AFId_1$.
		
		\item $AA.AuId_1 \rightarrow Id_2$ if and only if the \kw{Id} of the author
		with $AA.AuId_1$ is equal to $Id_2$.
		
		\item $Id_1 \rightarrow AA.AuId_2$ if and only if the \kw{AA.AuId} of the paper
		with $Id_1$ is equal to $AA.AuId_2$.
		
		
	\end{enumerate}
	For each test case, the REST service endpoint will receive a
	\Emph{JSON} arrat via \Emph{HTTP} with a pair of entity identifiers,
	where the identifiers are 64-bit integers, e.g. [123, 456].
	The service endpoint needs to respond with a \Emph{JSON} array within \Emph{300 seconds}.
	The response \Emph{JSON} array consists of a list of graph paths in to form of
	$[path_1, path_2, \cdots, path_n]$, where each path is an array of identifiers.
	For exmaple, if your program finds one 1-hop paths, two 2-hop paths, and one 3-hop paths,
	the results may look like this:
	$[[123, 456], [123, 2, 456], [123, 3, 456], [123, 4, 5, 6]]$.
	For a path such as $[123, 4, 5, 6]$, the integers are the identifiers of the entities on the path.
	After receiving the response, the evaluator will wait for a random period of time
	before sending the next requests.
	
\section{Evaluation}
\label{sec:evaluation}

	The \Emph{REST} service must be developed to a \Emph{Standard\_A3} virtual machine for the final test.
	There are no constraints on the programing 	language you can use.
	
	The test cases are not  available before the final evaluation.
	When the evaluation starts, the evaluator system sends test cases
	to the \Emph{REST} endpoint of each team individually.
	Each team will receive \Emph{10} test case ($Q_1$ to $Q_{10}$).
	The response time for test case \kw{$Q_i$} is recorded as \kw{$T_i$}$(1 \le i \le 10)$.
	The final score is calculated using:
	\[
		Score = \frac{10}{99} \sum_{i=1}^{10} \Big( 100^{\frac{\kw{$M_i$}}{\max(\kw{$K_i$}, \kw{$N_i$})}} \Big) \big( 1 - \frac{T_i}{300} \big)
	\]
	where \kw{$N_i$} is the size of the solution (the total number of correct paths) for \kw{$Q_i$},
	\kw{$K_i$} is the total number of paths returned by the \Emph{REST} service,
	\kw{$M_i$} is the number of distinct correct paths returned by the \Emph{REST} service.
	
\section{Background}
\label{sec:Background}

\subsection{MAG}

	\begin{figure}[H]
	\centering
	\includegraphics[scale=0.6]{mag.png}
    \caption{MAG的构建关系图}
	\end{figure}
	
\section{Thought}	
\label{sec:thought}

\subsection{All possible paths}
\label{subsec:all_possible_paths}

	根据题意Query共有4种类型,path有10种类型。hop最多为3跳,
	因此,我们可以预处理对于每种查询的所有可能路径种类,包含1跳、2跳、3跳。
	程序见\Emph{preprocess.cpp}。
	
\subsubsection{From \Emph{Id} to \Emph{Id}}
\label{subsub:Id2Id}	

	\begin{enumerate}[(1)]
		
		\item 1-hop
		
		\begin{align}
		\resetcounter
			\Id &\rightarrow \Id	\label{equ:id-id}
		\end{align}
		需要\Evaluate 一次,条件是\Id ,属性需要包含\RId.
		
		\item 2-hop
		
		\begin{align}
			\Id &\rightarrow \Id \rightarrow \Id	\label{equ:id-id-id}	\\
			\Id &\rightarrow \FFId \rightarrow \Id	\label{equ:id-fid-id}	\\
			\Id &\rightarrow \CCId \rightarrow \Id	\label{equ:id-cid-id}	\\
			\Id &\rightarrow \JJId \rightarrow \Id	\label{equ:id-jid-id}	\\
			\Id &\rightarrow \AAAuId \rightarrow \Id \label{equ:id-auid-id}
		\end{align}
		
		其中,$\Id \rightarrow \Id \rightarrow \Id$实际含义可以理解为论文A引用了论文B,论文B引用了论文C。
		这里可以初步确定论文B的发表时间和领域。
		其余path仅需要\Evaluate两次,条件分别是\Id,属性需要包含\FFId,\CCId,\JJId,\AAAuId。
		
		\item 3-hop
		
		\begin{align}
			\Id &\rightarrow \Id 	\rightarrow 	\Id	\rightarrow \Id		\label{equ:id-id-id-id}		\\
			\Id &\rightarrow \FFId 	\rightarrow 	\Id	\rightarrow \Id		\label{equ:id-fid-id-id}	\\
			\Id &\rightarrow \CCId 	\rightarrow 	\Id	\rightarrow \Id		\label{equ:id-cid-id-id}	\\
			\Id &\rightarrow \JJId 	\rightarrow 	\Id	\rightarrow \Id		\label{equ:id-jid-id-id}	\\
			\Id &\rightarrow \AAAuId \rightarrow 	\Id	\rightarrow \Id		\label{equ:id-auid-id-id}	\\
			\Id &\rightarrow \Id 	\rightarrow	\FFId \rightarrow \Id		\label{equ:id-id-fid-id}	\\
			\Id &\rightarrow \Id 	\rightarrow	\CCId \rightarrow \Id		\label{equ:id-id-cid-id}	\\
			\Id &\rightarrow \Id 	\rightarrow	\JJId \rightarrow \Id		\label{equ:id-id-jid-id}	\\
			\Id &\rightarrow \Id 	\rightarrow	\AAAuId \rightarrow \Id		\label{equ:id-id-auid-id}
		\end{align}
		
		其中,$\Id \rightarrow \Id \rightarrow \Id \rightarrow \Id$应该是最难并且数据量最大的,
		这里可以使用论文D的引用量简单判定是否存在解。
		接下来的4种情况其实是在2-hop的基础上进行延拓,一定存在与论文相同领域、会议、期刊的论文C
		引用了论文D的情况,这是显然的。同样这里可以使用时间进行剪枝。
		最后的4种情况相对简单,对论文D进行一次\Evaluate,属性需要包属性\FFId,\CCId,\JJId,\AAAuId。(2-hop已经进行)
		对论文A的参考文献分别\Evaluate一次,属性需要包含\FFId,\CCId,\JJId,\AAAuId。
		
	\end{enumerate}
	
	综上所述,我们需要对论文\id{stId}进行一次\Evaluate,属性需要包含\FFId,\CCId,\JJId,\AAAuId,\RId,\Emph{Y}。
	需要对论文\id{edId}进行一次\Evaluate,属性需要包含\FFId,\CCId,\JJId,\AAAuId,\Emph{Y}。
	
\subsubsection{From \Emph{Id} to \Emph{AA.AuId}}
\label{subsub:Id2AAAuId}

	\begin{enumerate}[(1)]
		
		\item 1-hop
		
		\begin{align}
			\Id &\rightarrow \AAAuId	\label{equ:id-auid}
		\end{align}
		需要\Evaluate一次,条件是\Id,属性需要包含\AAAuId.
		
		\item 2-hop
		
		\begin{align}
			\Id &\rightarrow \Id \rightarrow \AAAuId	\label{equ:id-id-auid}
		\end{align}
		首先\Evaluate一次,条件是\Id,属性需要包含\RId,
		再对每个\RId进行遍历\Evaluate,条件是包含\AAAuId.
		(亦可对\AAAuId进行\Evaluate,属性需要包含\Id)
		
		\item 3-hop
		
		\begin{align}
			\Id &\rightarrow \Id \rightarrow \Id \rightarrow \AAAuId		\label{equ:id-id-id-auid}	\\
			\Id &\rightarrow \FFId \rightarrow \Id \rightarrow \AAAuId		\label{equ:id-fid-id-auid}	\\
			\Id &\rightarrow \CCId \rightarrow \Id \rightarrow \AAAuId		\label{equ:id-cid-id-auid}	\\
			\Id &\rightarrow \JJId \rightarrow \Id \rightarrow \AAAuId		\label{equ:id-jid-id-auid}	\\
			\Id &\rightarrow \AAAuId \rightarrow \Id \rightarrow \AAAuId	\label{equ:id-auid-id-auid}	\\
			\Id &\rightarrow \AAAuId \rightarrow \AAAfId \rightarrow \AAAuId \label{equ:id-auid-afid-auid}
		\end{align}
		显然论文级联引用比较麻烦,可以采用~\ref{subsub:Id2Id}的策略,
		这样需要预先\Evaluate,条件是\AAAuId,属性需要包含\Id。
		接下来的4个相对简单,预先对论文\id{Id}根据\Id进行\Evaluate,属性需要包含\FFId,\CCId,\JJId,\AAAuId。
		可以对\AAAuId进行\Evaluate,属性需要包含\Id,
		然后对每个\Id进行\Evaluate,属性需要包含\FFId,\CCId,\JJId,\AAAuId。
		然后进行比较即可。
		最后一种情况同样需要对论文\id{Id}进行预处理,同时需要\id{AA.AuId}的\AAAfId属性。然后再做第三次查询。
		
	\end{enumerate}
	
	综上所述,我们需要对论文\id{stId}进行一次\Evaluate,属性需要包含\FFId,\CCId,\JJId,\AAAuId,\AAAfId。
	需要对\id{AA.AuId}进行一次\Evaluate,属性需要包含\Id, \FFId,\CCId,\JJId,\AAAuId,\AAAfId。
	
\subsubsection{From \Emph{AA.AuId} to \Emph{Id}}
\label{subsub:AAAuId2Id}

	\begin{enumerate}[(1)]
		
		\item 1-hop
		
		\begin{align}
			\AAAuId &\rightarrow \Id	\label{equ:auid-id}
		\end{align}
		需要\Evaluate一次,条件是\Id,属性需要包含\AAAuId.
		
		\item 2-hop
		
		\begin{align}
			\AAAuId &\rightarrow \Id \rightarrow \Id	\label{equ:auid-id-id}
		\end{align}
		首先\Evaluate一次,条件是\AAAuId,属性需要包含\RId,查询数据是\Id.
		
		\item 3-hop
		
		\begin{align}
			\AAAuId &\rightarrow \Id \rightarrow \Id \rightarrow \Id		\label{equ:auid-id-id-id}	\\
			\AAAuId &\rightarrow \Id \rightarrow \FFId \rightarrow \Id		\label{equ:auid-id-fid-id}	\\
			\AAAuId &\rightarrow \Id \rightarrow \CCId \rightarrow \Id		\label{equ:auid-id-cid-id}	\\
			\AAAuId &\rightarrow \Id \rightarrow \JJId \rightarrow \Id		\label{equ:auid-id-jid-id}	\\
			\AAAuId &\rightarrow \Id \rightarrow \AAAuId \rightarrow \Id	\label{equ:auid-id-auid-id}	\\
			\AAAuId &\rightarrow \AAAfId \rightarrow \AAAuId \rightarrow \Id \label{equ:auid-afid-auid-id}
		\end{align}
		预先执行\Evaluate,条件是\Id,查询数据是\FFId,\CCId,\JJId,\AAAuId,\AAAfId。
		并执行\Evaluate,条件是\AAAuId,查询数据是\Id,\RId。
		然后组合属性\FFId,\CCId,\JJId,\AAAuId,并且满足\AAAuId属性为\id{stId}执行\Evaluate,
		查询数据为\Id,\FFId,\CCId,\JJId,\AAAuId。
		同理,组合户型\AAAfId,并且满足\AAAuId属性为\id{stId}执行\Evaluate,
		查询数据为\Id,\AAAfId。
		针对论文级联引用,对每组\RId执行一次\Evaluate,组合属性为\RId满足\id{edId},
		查询数据为\Id。
		
	\end{enumerate}
	
	综上所述,我们需要对论文\id{stId}进行一次\Evaluate,属性需要包含\FFId,\CCId,\JJId,
    \AAAuId,\AAAfId。
	需要对\id{AA.AuId}进行一次\Evaluate,属性需要包含\Id, \FFId,\CCId,\JJId,\AAAuId,\AAAfId。
	
\subsubsection{From \Emph{AA.AuId} to \Emph{AA.AuId}}

	\begin{enumerate}[(1)]
	
		\item 1-hop
		
		$\nexists.$
		
		\item 2-hop
		
		\begin{align}
			\AAAuId &\rightarrow \AAAfId \rightarrow \AAAuId	\label{equ:auid-afid-auid}	\\
			\AAAuId &\rightarrow \Id	\rightarrow \AAAuId		\label{equ:auid-id-auid}
		\end{align}
		
		组合查找就好了。
		
		\item 3-hop
		
		\begin{align}
			\AAAuId &\rightarrow \Id \rightarrow \Id \rightarrow \AAAuId	\label{equ:auid-id-id-auid}
		\end{align}
		
		分别预处理两个作者的paper,然后遍历作者\id{stId}的论文,分别查找\RId,判断是否存在可行路径。
		
	\end{enumerate}
	
\subsection{refined}
\label{subsec:refined}

\subsubsection{From \Emph{Id} To \Emph{Id}}

	\begin{enumerate}[(1)]
	
		\item \Evaluate条件是$\Emph{Id}=stId$,属性为\FFId,\CCId,\JJId,\AAAuId,\RId。
		
		\item \Evaluate条件是$\Emph{Id}=edId$,属性为\FFId,\CCId,\JJId,\AAAuId。
		
		判定$edId$是否在$stId$的\RId中,处理路径~\ref{equ:id-id}。
		
		由$stId$与$edId$两者主属性的交集,处理路径~\ref{equ:id-fid-id}、~\ref{equ:id-cid-id}、~\ref{equ:id-jid-id}、~\ref{equ:id-auid-id}。
		
		\item \Evaluate条件是$stId$的主属性并且$\Emph{RId}=edId$,属性为\Id, \FFId,\CCId,\JJId,\AAAuId。
		
		遍历得到的每个entry,求每个主属性的交集,处理路径~\ref{equ:id-fid-id-id}、~\ref{equ:id-cid-id-id}、~\ref{equ:id-jid-id-id}、~\ref{equ:id-auid-id-id}。
		
		\item \Evaluate条件是$\Emph{Id} \in \attrib{stId}{RId}$,属性为\FFId,\CCId,\JJId,\AAAuId,\RId。
		
		遍历每个$\attrib{st}{RId}$,获得它的\RId集合以及主属性集合,其中
		主属性集合与$edId$的主属性集合的交集处理了路径~\ref{equ:id-id-fid-id}、~\ref{equ:id-id-cid-id}、~\ref{equ:id-id-jid-id}、~\ref{equ:id-id-auid-id}。
		
		\item \Evaluate条件是$\Emph{Id} \in \attribb{stId}{RId}{RId}$并且$\Emph{RId}=edId$,属性为\Id。
		
		预先生成$\attribb{stId}{RId}{RId}$的倒排字典,即主键为三层\Emph{Id},值为二层\Emph{Id}的集合。
        由这个倒排表的键值为$edId$对应的值,可以处理路径~\ref{equ:id-id-id}。
		与刚得到的每个entry的\Id集合求交集,并通过倒排字典可以处理路径~\ref{equ:id-id-id-id}。
		
	\end{enumerate}

\subsubsection{From \Emph{Id} To \Emph{AA.AuId}}

	\begin{enumerate}[(1)]
	
		\item \Evaluate条件是$\Emph{Id}=stId$,属性为\FFId,\CCId,\JJId,\AAAuId,\RId。
	
		\item \Evaluate条件是$\Emph{AuId}=edId$,属性为\Id,\FFId,\CCId,\JJId,\AAAuId,\AAAfId。
		
		判定$\attrib{stId}{AuId}$的集合中是否包含$edId$,可以处理路径~\ref{equ:id-auid}。
		
		由$\attrib{stId}{RId}$与$\attrib{edId}{Id}$的交集,可以处理路径~\ref{equ:id-id-auid}。
		
		生成$stId$的主属性集合,遍历$edId$的每个entry,生成主属性集合,两者做交集。
		可以处理路径~\ref{equ:id-fid-id-auid},~\ref{equ:id-cid-id-auid},~\ref{equ:id-jid-id-auid},~\ref{equ:id-auid-id-auid}。
		
		预处理$\attrib{edId}{AfId}$中属于$edId$这个作者的\Emph{AfId}集合。
		
		\item \Evaluate条件是$\Emph{AuId} \in \attrib{stId}{AuId}$并且$\Emph{AfId} \in \attrib{edId}{AfId}$。
		
		生成刚得到数据的\AAAfId的倒排字典,即主键为\Emph{AfId},值为\Emph{AuId}的集合。
		遍历$\attrib{edId}{AfId}$的每个\Emph{AfId},字典中对应值的集合与$\attrib{stId}{AuId}$集合的交集,
		可以处理路径~\ref{equ:id-auid-afid-auid}。
		
		\item \Evaluate条件是$\Emph{Id} \in \attrib{stId}{RId}$并且$\Emph{RId} \in \attrib{edId}{Id}$,属性为\Id, \RId。
		
		或者\Evaluate条件是$\Emph{Id} \in \attrib{stId}{RId}$,属性为\Id, \RId。
		对于新得到的每个entry,生成$\Emph{RId}$集合,并与$\attrib{edId}{Id}$求交集,
		可以处理路径~\ref{equ:id-id-id-auid}。
	
	\end{enumerate}

\subsubsection{From \Emph{AA.AuId} To \Emph{Id}}

	\begin{enumerate}[(1)]
	
		\item \Evaluate条件是$\Emph{AuId}=stId$,属性为\Id,\FFId,\CCId,\JJId,\AAAuId,\AAAfId,\RId。
	
		\item \Evaluate条件是$\Emph{AuId}=edId$,属性为\Id,\FFId,\CCId,\JJId,\AAAuId。
		
		由$\attrib{stId}{Id}$中是否包含$edId$,处理路径~\ref{equ:auid-id}。
		
		由$\attrib{stId}{RId}$的集合中是否包含$edId$,处理路径~\ref{equ:auid-id-id}。
		
		遍历$\attrib{stId}{Id}$中每个\Emph{Id}对应的主属性集合,与$edId$的主属性集合做交集,
		可以处理路径~\ref{equ:auid-id-fid-id},~\ref{equ:auid-id-cid-id},~\ref{equ:auid-id-jid-id},~\ref{equ:auid-id-auid-id}。
		
		预处理$\attrib{stId}{AfId}$中属于$stId$这个作者的\Emph{AfId}集合。
		预处理$\attrib{edId}{AuId}$中的\Emph{AuId}集合。
		
		\item \Evaluate条件是$\Emph{AuId} \in \attrib{edId}{AuId}$并且$\Emph{AfId} \in \attrib{stId}{AfId}$,属性为\AAAuId, \AAAfId。
		
		生成刚得到数据的\AAAfId的倒排字典,即主键为\Emph{AfId},值为\Emph{AuId}的集合。
		遍历$\attrib{stId}{AfId}$的每个\Emph{AfId},字典中对应值的集合与$\attrib{edId}{AuId}$集合的交集,
		可以处理路径~\ref{equ:auid-afid-auid-id}。
		
		预先生成$\attrib{stId}{RId}$的倒排字典,即主键为$\Emph{RId}$,值为$\Emph{Id}$。
		
		\item \Evaluate条件是$\Emph{Id} \in \attrib{stId}{RId}$并且$\Emph{RId}=edId$,属性为\Id。
		
		遍历刚得到数据的每个entry,并通过倒排字典中对应$\Emph{Id}$的集合,可以处理路径~\ref{equ:auid-id-id-id}。
	
	\end{enumerate}

\subsubsection{From \Emph{AA.AuId} To \Emph{AA.AuId}}

	\begin{enumerate}[(1)]
		
		\item \Evaluate条件是$\Emph{AuId}=stId$,属性为\AAAuId, \AAAfId, \Id, \RId。
		
		\item \Evaluate条件是$\Emph{AuId}=edId$,属性为\AAAuId, \AAAfId, \Id。
		
		由$\attrib{stId}{Id}$与$\attrib{edId}{Id}$的交集可以处理路径~\ref{equ:auid-id-auid}。
		
		在$stId$的所有entry中查找$\Emph{AuId}=stId$的$\Emph{AfId}$,采用同样的方式处理$edId$。
		两者的交集可以处理路径~\ref{equ:auid-afid-auid}。
		
		预先生成$\attrib{stId}{RId}$的倒排字典,即主键为两层\Emph{Id},值为一层\Emph{Id}的集合。
		并与$edId$的\Id集合求并集,并通过倒排字典可以处理路径~\ref{equ:auid-id-id-auid}。
		
	\end{enumerate}
	
\end{document}

