%-*- coding:UTF-8 -*-
% 算法导论-第20章 van Emde Boas树.tex
\documentclass[UTF8]{ctexart}
\usepackage{geometry}
\usepackage{enumerate}
\usepackage{amsmath}
\usepackage{listings} %插入代码
\usepackage{xcolor} %代码高亮
\usepackage{diagbox}
\usepackage{tabularx}
\usepackage{graphicx}
\usepackage{caption}
\usepackage{subcaption}
\usepackage{float}

% Some setup
\pagestyle{plain}
\geometry{a4paper, top=2cm, bottom=2cm, left=2cm, right=2cm}
\CTEXsetup[format={\raggedright\bfseries\Large}]{section}
\lstset{numbers=left, %设置行号位置
        numberstyle=\small, %设置行号大小
        keywordstyle=\color{blue}, %设置关键字颜色
        commentstyle=\color{purple}, %设置注释颜色
        %frame=single, %设置边框格式
        escapeinside=``, %逃逸字符(1左面的键),用于显示中文
        breaklines, %自动折行
        extendedchars=false, %解决代码跨页时,章节标题,页眉等汉字不显示的问题
        %xleftmargin=2em,xrightmargin=2em, aboveskip=1em, %设置边距
        tabsize=4, %设置tab空格数
        showspaces=false %不显示空格
       }

% About math
\newcommand{\rmnum}[1]{\romannumeral #1}

\begin{document}

\title{\Huge算法导论习题\\}
\vspace{2cm}
\author{\Large Trasier}
\date{}
\maketitle

%\newpage
\section*{第20章\quad van\ Emde\ Boas树}
\begin{enumerate}
    \item 20.1-1 \\
    解:\\
	将0-1标记修改为0-n标记即可以支持重复数字。
	
	\item 20.1-2 \\
	解:\\
	引入一个指针数组或整形数组对应于关键字,指向卫星数据的地址。
	
	\item 20.1-3 \\
	解:\\
	注意题目问的是BST中如何查询,而这是x其实并不是BST的一个结点。
	基本思路是二分。(注意返回INT\_MAX表示不存在)
	\begin{lstlisting}[language=C++]
	int Successor(int x) {
		int ret = INT_MAX;
		Node *p = rt;
		
		while (p) {
			if (x < p->val) {
				ret = p->val;
				p = rt->left;
			} else {
				p = rt->right;
			}
		}
		
		return ret;
	}
	\end{lstlisting}
	
	\item 20.1-4 \\
	解:\\
	这里注意$u=2^{2k}, degree = u^{1/k} = 2^2 = 4$.
	这样一棵树的高度为k,时间复杂度为O(k)。
	
	\item 20.2-1 \\
	解:\\
	见源代码。
	
	\item 20.2-2 \\
	解:\\
	时间复杂度为$O(n)$。
	
	\item 20.2-3 \\
	解:\\
	时间复杂度为$\theta (lglgu)$。
	
	\item 20.2-6 \\
	解:\\
	见Build。
	
	\item 20.2-7 \\
	解:\\
	summary是对当前结点指向的cluster的一个概述。
	若summary中的min$-$cluster为空则表示当前结点的下属cluster内没有任何元素,即为空。
	
	\item 20.2-8 \\
	解:\\
	以Minimum为例。
	\begin{align*}
		T(u) &= 2T(u^{3/4}) + O(1)	\\
		T(2^m) &= 2T(2 ^ {m \ast 3/4}) + O(1)  \\
        S(m) &= 2T(m \ast 3/4) + O(1) \\
        S(m) &= O(m^{2.4}) \quad m = lgu  \\
        T(u) &= O({(lgu)}^{2.4})
	\end{align*}
	
\end{enumerate}


\end{document}

