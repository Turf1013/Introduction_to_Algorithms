%-*- coding:UTF-8 -*-
% TerrainCrossing.tex
\documentclass[hyperref,UTF8]{ctexart}
\usepackage{geometry}
\usepackage{enumerate}
\usepackage{amsmath}
\usepackage{amssymb}
\usepackage{dsfont}
\usepackage{amsthm}
\usepackage{listings} %插入代码
\usepackage{xcolor} %代码高亮
\usepackage{blkarray}
\usepackage{diagbox}
\usepackage{tabularx}
\usepackage{graphicx}
\usepackage{caption}
\usepackage{subcaption}
\usepackage{float}
\usepackage{color}
\usepackage{multirow}
\usepackage[all,pdf]{xy}
\usepackage{verbatim}   %comment
\usepackage{cases}
\usepackage{clrscode3e}	% need clrscode3e package which is not included in CTex.
\usepackage{hyperref}

\geometry{screen}
\hypersetup{
    colorlinks=true,
    bookmarks=true,
    bookmarksopen=false,
    % pdfpagemode=FullScreen,
    pdfstartview=fit,
    pdftitle={TerrainCrossing},
    pdfauthor={Trasier}
}

% THEOREM Environments --------------------------------------------------------
\newtheorem{thm}{Theorem}[subsection]
\newtheorem{cor}[thm]{Corollary}
\newtheorem{lem}[thm]{Lemma}
\newtheorem{prop}[thm]{Proposition}
\newtheorem{prob}[thm]{Problem}
\newtheorem{mthm}[thm]{Main Theorem}
\theoremstyle{definition}
\newtheorem{defn}[thm]{Definition}
\theoremstyle{remark}
\newtheorem{rem}[thm]{Remark}
\numberwithin{equation}{subsection}
% MATH ------------------------------------------------------------------------
\DeclareMathOperator{\RE}{Re}
\DeclareMathOperator{\IM}{Im}
\DeclareMathOperator{\ess}{ess}
\newcommand{\eps}{\varepsilon}
%\newcommand{\To}{\longrightarrow}  conflict with \package{clrscode3e}p
\newcommand{\h}{\mathcal{H}}
\newcommand{\s}{\mathcal{S}}
\newcommand{\A}{\mathcal{A}}
\newcommand{\J}{\mathcal{J}}
\newcommand{\M}{\mathcal{M}}
\newcommand{\W}{\mathcal{W}}
\newcommand{\X}{\mathcal{X}}
\newcommand{\BOP}{\mathbf{B}}
\newcommand{\BH}{\mathbf{B}(\mathcal{H})}
\newcommand{\KH}{\mathcal{K}(\mathcal{H})}
\newcommand{\Real}{\mathbb{R}}
\newcommand{\Complex}{\mathbb{C}}
\newcommand{\Field}{\mathbb{F}}
\newcommand{\RPlus}{\Real^{+}}
\newcommand{\Polar}{\mathcal{P}_{\s}}
\newcommand{\Poly}{\mathcal{P}(E)}
\newcommand{\EssD}{\mathcal{D}}
\newcommand{\Lom}{\mathcal{L}}
\newcommand{\States}{\mathcal{T}}
\newcommand{\abs}[1]{\left\vert#1\right\vert}
\newcommand{\set}[1]{\left\{#1\right\}}
\newcommand{\seq}[1]{\left<#1\right>}
\newcommand{\norm}[1]{\left\Vert#1\right\Vert}
\newcommand{\essnorm}[1]{\norm{#1}_{\ess}}


% Some setup
\pagestyle{plain}
\geometry{a4paper, top=2cm, bottom=2cm, left=2cm, right=2cm}
\CTEXsetup[format={\raggedright\bfseries\Large}]{section}
\lstset{numbers=left, %设置行号位置
        numberstyle=\small, %设置行号大小
        keywordstyle=\color{blue}, %设置关键字颜色
        commentstyle=\color{purple}, %设置注释颜色
        %frame=single, %设置边框格式
        escapeinside=``, %逃逸字符(1左面的键),用于显示中文
        breaklines, %自动折行
        extendedchars=false, %解决代码跨页时,章节标题,页眉等汉字不显示的问题
        %xleftmargin=2em,xrightmargin=2em, aboveskip=1em, %设置边距
        tabsize=4, %设置tab空格数
        showspaces=false %不显示空格
       }

% About math
\newcommand{\rmnum}[1]{\romannumeral #1}
\newcommand{\Emph}{\textbf}
\newcolumntype{Y}{>{\centering\arraybackslash}X}
\newcommand{\resetcounter}{\setcounter{equation}{0}}
\newcommand{\equsuf}[1][x]{\equiv_{\textit{Suff(#1)}}}	
\newcommand{\Suff}{\textit{Suff}}
\newcommand{\len}[1][x]{\textit{length}_{#1}}

% section deeep to 3 1.1.1
\setcounter{secnumdepth}{3}

\begin{document}

\title{\Huge TerrainCrossing}
\vspace{2cm}
\author{\Large Trasier}
\date{\today}
\maketitle

\section{Problem Statement}
\label{sec:prob_statement}

	You are given a square map of $S \times S$ cells. Each cell is a square area of certain terrain type, with side of length 1.
	Terrain types are denoted with digits from 0 to 9, which describe the cost of passing through this terrain, per uni:
	0 is the easiest to cross, and 9 is the hardest. The cost of passing through the square is calculated by multiplying the
	\Emph{Euclidean} distance travelled with the terrain type. When you cross a border between two terrain types, you incur additional
	cost of (ddifference of types)\^2.
	
	
	There are \id{N} items located on the map, and \id{N} target locations to which these items have to be delivered.
	Items are identical, so each item can be delivered to any location, but each location must have exactly one item delivered to it.
	You can carry at most \id{capacity} items at once. You automatically pick up an item if you stop winth in $10^{-3}$ from it and
	still have capacity to carry it, and you automatically drop off an item at a target location if you stop within $10^{-3}$ from it while
	carrying at least one item and no item has been delivered to this location yet.
	
	
	You task is to enter the map at any place along its border, pick up all items and deliver them to target locations, and exit the map
	at any place along its border.

\section{Implementation}	
\label{sec:Implementation}

	You code must implement one method \proc{getPath}(vector$\langle$string$\rangle$ \Emph{terrain}, vector$\langle$double$\rangle$ \Emph{locations}, int \Emph{capacity}).
	\begin{itemize}
	
		\item \id{terrain} gives the map of terrain types in the area. $terrain[i][j]$ describes the type of terrain in the square with
		coordinates $[j, j+1] \times [i, i+1]$. Character $0 \ldots 9$ 表明类型 $0 \ldots 9$.
		
		\item \id{locations} gives the list of items and target locations for them. For \id{N} items and \id{N} target locations,
		\id{locations} will contain $4 \ast N$ elements. First $2 \ast N$ elements will describe positions of items: \id{i}th item is located
		at coordinates ($locations[2*i], locations[2*i+1]$). Next $2 \ast N$ elements will describe target locations: \id{j}th target location
		has coordinates ($locations[2*N+2*j], locations[2*N+2*j+1]$).
		
		\item \id{capacity} gives the maximum number of items you can carry at once.
		
	\end{itemize}
	
	The return from this method will describe a path you want to take. The path is sequence of points within the map connected by segments;
	\id{i}-th point of the path has coordinates ($return[2*i], return[2*i+1]$). The path must statisfy the following conditions:
	\begin{itemize}
		
		\item The path must have between $2$ and $4 \ast S^2 \ast N$ points, inclusive.
		
		\item Each point of the path must be within the map, i.e. bot coordinates must be between $0$ and $S$.
		
		\item The first and the last points of the path must be within $10^{-3}$ from the outer border of the map.
		
		\item All points of the path must be at least $10^{-3}$ away from internal borders between the cells of the map (even if cells on both
		sides of the border are of the same terrain type).
		
		\item Consecutive points of the path must be at least $10^{-3}$ away from each other. (\Emph{Euclidean} distance).
		
		\item Each segment of the path can cross at most one boundary between cells of the map, i.e. the Manhattan distance between cells to
		which consecutive points of the path belong can be at most 1.
		
		\item After the path is walked, all items must be picked up and all target locations must have an item delivered to them.
		
	\end{itemize}
	
\section{Scoring}
\label{sec:Scoring}

	For each test case we will calculate your raw and normalized scores. If you solution returned invalid path (some points outside
	the map or too close to the borders, not all items picked up and delivered etc.), raw score will be 0. Otherwise, raw score will be
	the total cost of the path, calculated as sum of costs of passing through terrain done on each segment of the path (for each part
	of the path which passes through terrain of the same type its cost is the length of the part, multipiled by cost of this terrain type)
	and additional cost of changing terrain types. The normalized score for each test is $1,000,000.0 \ast BEST / YOUR$, where \Emph{BEST}
	is the lowest raw score currently obtained on this test case(considering only the last submission from each competitor). Finally,
	your total score is equal to the arithmetic average of normalized scores on all test cases.
	
\section{Notes}	

	\begin{enumerate}[(1)]
	
		\item The time limit is \Emph{10} seconds per test case (this includes only the time spent in your code). The memory limit is \Emph{1024} megabytes.
		
		\item There is no explicit code size limit. The implicit source code size is around \Emph{1MB}. Once your code is compiled,
		the binary size should not exceed \Emph{1MB}.
		
		\item The compilation time limit is \Emph{30} seconds.
		
		\item There are \Emph{10} example test cases and \Emph{100} full submission (provisional) test cases.
		
	\end{enumerate}
	
	
\section{Constraints}	
	
	\begin{enumerate}[(1)]
	
		\item The size of the map \id{S} will be between \Emph{10} and \Emph{50}, inclusive.
		
		\item The number of terrain types \id{T} will be between \Emph{2} and \Emph{10}, inclusive.
		
		\item The number of items \id{N} will be between \Emph{5} and $S \ast S / 10$, inclusive.
		
		\item The carrying \id{capacity} will be between \Emph{1} and \Emph{10}, inclusive.
		
	\end{enumerate}
	
\end{document}
