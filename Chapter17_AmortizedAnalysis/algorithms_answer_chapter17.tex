%-*- coding:UTF-8 -*-
% 算法导论-第17章 摊还分析.tex
\documentclass[UTF8]{ctexart}
\usepackage{geometry}
\usepackage{enumerate}
\usepackage{amsmath}
\usepackage{listings} %插入代码
\usepackage{xcolor} %代码高亮
\usepackage{diagbox}
\usepackage{tabularx}
\usepackage{graphicx}
\usepackage{caption}
\usepackage{subcaption}
\usepackage{float}

% Some setup
\pagestyle{plain}
\geometry{a4paper, top=2cm, bottom=2cm, left=2cm, right=2cm}
\CTEXsetup[format={\raggedright\bfseries\Large}]{section}
\lstset{numbers=left, %设置行号位置
        numberstyle=\small, %设置行号大小
        keywordstyle=\color{blue}, %设置关键字颜色
        commentstyle=\color{purple}, %设置注释颜色
        %frame=single, %设置边框格式
        escapeinside=``, %逃逸字符(1左面的键),用于显示中文
        breaklines, %自动折行
        extendedchars=false, %解决代码跨页时,章节标题,页眉等汉字不显示的问题
        %xleftmargin=2em,xrightmargin=2em, aboveskip=1em, %设置边距
        tabsize=4, %设置tab空格数
        showspaces=false %不显示空格
       }

% About math
\newcommand{\rmnum}[1]{\romannumeral #1}

\begin{document}

\title{\Huge算法导论习题\\}
\vspace{2cm}
\author{\Large曾宇祥\\SY1406122}
\date{}
\maketitle

%\newpage
\section*{第17章\quad摊还分析}
\begin{enumerate}
    \item 17.1-1 \\
    解:\\
    是,道理同MULTIPOP。
	
	\item 17.1-2 \\
	证明:\\
	\[
		\sum_{i=0}^{k-1}\left \lfloor \frac{n}{k^i} \right \rfloor <
		n\sum_{i=0}^{\infty} \frac{1}{2^i} = kn
	\]
	故,$T(n) = \theta{(kn)}$.
	
	\item 17.1-3 \\
	解:
	\[
		cost(n) = \left\{
		\begin{aligned}
		2^k	&\quad if n\ = 2^k \\
		1	&\quad if n\ = 2^k+m, 0<m<2^k-1
		\end{aligned}
        \right.
	\]	\\
	这里假设$n = 2^k + m$, 故
	\begin{align*}
		T(n)&= \sum_{i=1}^{n} cost(i)	\\
			&= 2^0 + 2^1 + 1 + 2^2 + 1 + 1 + \cdots + cost(n)	\\
			&= n - (k+1) + \sum_{i=0}^{k} 2^i	\\
			&= n - k - 1 + 2^{k+1} -1	\\
			&< n + 2*2^k	\\
			&< n + 2n	\\
			&= 3n	\\
			&= O(n)
	\end{align*}
	
	\item 17.2-1 \\
	证明:\\
	假设ith代表操作序列中的第i个操作,则摊还代价不妨设为
	\[
		\begin{aligned}
		c_{push} &= 2 + delta.	\\
		c_{pop} &= 0 + delta.	\\
		delta &= \left\{
		\begin{aligned}
			0 &\quad if\ n\mod k\neq0	\\
			k &\quad if\ n\mod k=0
		\end{aligned}
		\right.
		\end{aligned}
	\]
	则均摊代价恒为实际代价的上界,故均摊代价如下计算:
	\begin{align*}
		cost(n) &< 2n + \lfloor n/k \rfloor *k	\\
			&< 2n + n	\\
			&= O(n)
	\end{align*}
	则,实际代价$T(n) < cost(n) = O(n)$.
	
	\item 17.2-2 \\
	证明:\\
	假设i表示操作的第i个操作,则摊还代价不妨设为
	\[
		c_i = \left\{
		\begin{aligned}
			i	&\quad if \ i = 2^k	\\
			1	&\quad if \	i = 2^k+m, 0<m<2^k-1
		\end{aligned}
		\right.
	\]
	则易知均摊代价恒为实际代价的上界,故均摊代价如下计算(这里$n = 2^k+m$):
	\begin{align*}
		cost(n) &= 2^0 + 2^1 + 1 + 2^2 + 1 + 1 + \cdots + c_n
				&= n - k - 2 + 2^{k+1}
				&< n + 2n
				&< 3n
				&= O(n)
	\end{align*}
	则,实际代价$T(n) < cost(n) = O(n)$.
	
	\item 17.2-3 \\
	证明:\\
	\begin{lstlisting}[language=]
	INCREMENT(A):
		i = 0
		while i<A.length and A[i]==1:
			A[i] = 0
			i += 1
		if i < A.length:
			A[i] = 1
			if i > A.max:
				A.max = i
		else:
			A.max = -1
			
	RESET(A):
		for i=0 to A.max
			A[i] = 0
		A.max = -1
	\end{lstlisting}
	摊还代价不妨假设为
	\begin{align*}
		c_{increment} &= 4	\\
		c_{reset} &= 1
	\end{align*}
	reset的1个代价用来设置A.max, 而increment的4个代价分别用来作为置1、信用代价、
	可能更新A.max以及RESET清零的代价。故摊还代价为实际代价的上界。摊还代价计算如下:
	\begin{align*}
		cost(n) &\le 4n	\\
				&= O(n)
	\end{align*}
	则,实际代价$T(n) < cost(n) = O(n)$.
	
	\item 17.3-1	\\
	证明:\\
	不妨令$\Phi'(D_i) = \Phi(D_i) - \Phi(D_0)$. \\
	则$\Phi'$的摊还代价为
	\[
		\hat{c'} = c + \Phi'(D_i) - \Phi'(D_{i-1})
				 = c + (\Phi(D_i) - \Phi(D_0)) - (\Phi(D_{i-1}) - \Phi(D_0))
				 = c + \Phi(D_i) - \Phi(D_{i-1})
	\]	\\
	而$\Phi$的摊还代价为$\hat{c} = c + \Phi(D_i) - \Phi(D_{i-1})$.	\\
	故, $\hat{c'} = \hat{c}$. 因此,结论得证。
	
	\item 17.3-2	\\
	证明:\\
	由题意可知实际代价
	\[
		c_i = \left\{
		\begin{aligned}
			i	&\quad if \ i = 2^k	\\
			1	&\quad if \	i = 2^k+m, 0<m<2^k-1
		\end{aligned}
		\right.
	\]
	可令
	\[
		\Phi(D_i) = \sum_{i=\lfloor\log{i}\rfloor+1}^{\infty} 2^i,\quad \Phi(0) = \sum_{i=0}^{\infty} 2^i
	\]
	, 易知$\Phi(D_i) >= \Phi(D_0)$则. \\
	当$i=2^k, k>1$时,
	\begin{align*}
		\hat{c_i}	&= c_i + \Phi(D_i) - \Phi(D_{i-1})							\\
					&= 2^k + \sum_{i=k+1}^{\infty}2^i - \sum_{i=k}^{\infty}2^i				\\
					&= 2^k + (\sum_{i=k+1}^{\infty}2^i - \sum_{i=k+1}^{\infty}2^i - 2^k)	\\
					&= 2^k - 2^k												\\
					&= 0
	\end{align*}
	当$i=2^k+m, k>1, m>1$时,
	\begin{align*}
		\hat{c_i}	&= c_i + \Phi(D_i) - \Phi(D_{i-1})	\\
					&= 1 + \sum_{i=k+1}^{\infty}2^i - \sum_{i=k+1}^{\infty}2^i	\\
					&= 1 + 0	\\
					&= 1
	\end{align*}
	从而,可知每个操作的摊还代价都是O(1),则n个操作的摊还代价为O(n)。
	
	\item 17.3-3	\\
	证明:\\
	
	\item 17.3-4 \\
	解:\\
	不妨设PUSH的次数为$n_{push}$, 因此总的代价为$c = s_0 + n_{push} - s_n$.
	
	\item 17.3-5 \\
	解:\\
	由17.3-1证明可知$\Phi(D_0) = b$, 令$\Phi'(D_i) = \Phi(D_i) - \Phi(D_0)$. 故$\Phi'(D_i) - \Phi'(D_{i-1}) = 1 - t_i$, 则\\
	\begin{align*}
		\hat{c_i}	&= c_i + \Phi(D_i) - \Phi(D_{i-1})	\\
					&= c_i + (\Phi(D_i) - \Phi(D_0)) - (\Phi(D_{i-1}) - \Phi(D_0))	\\
					&= c_i + \Phi'(D_i) - \Phi'(D_{i-1})	\\
					&= t_i + 1 + 1 - t_i	\\
					&= 2
	\end{align*}
	又因为$n = \omega(b)$, 故
	\begin{align*}
		\sum_{i=1}^n c_i	&= \sum_{i=1}^n 2 - b_n + b_0	\\
							&= 2n - b_n + b_0	\\
							&< 2n + b	\\
							&= O(n)
	\end{align*}
	
	\item 17.3-6 \\
	证明:\\
	程序伪代码如下:
	\begin{lstlisting}[language=Python]
		Enqueue(x):
			s0.push(x)
		Dequeue():
			if !s1.empty():
				x = s1.top()
				s1.pop()
				return x
			else:
				while True:
					x = s0.top()
					s0.pop()
					if s0.empty()
						return x
					s1.push(x)	
	\end{lstlisting}
	不妨令势函数$\Phi(D_i) = (t1==0)*t0$,
	其中$t_i$表示栈i中当前元素个数。
	易知$\Phi(D_0) = 0$,而$\Phi(D_i) \ge 0 = \Phi(D_0)$恒成立。故\\
	对于Enqueue操作
	\begin{align*}
		\hat{c_i}	&= c + \Phi(D_i) - \Phi(D_{i-1})	\\
					&= 1 +	\left\{
								\begin{aligned}
									t0_{i-1}+1 - t0_{i-1}	&\quad if \ t1_{i-1}=0	\\
									0 - 0	&\quad if \ t1_{i-1}>0
								\end{aligned}
							\right.	\\
					&< 1 + 1	\\
					&= O(1)
	\end{align*}
	对于Dequeue操作
	\begin{align*}
		\hat{c_i}	&= c + \Phi(D_i) - \Phi(D_{i-1})	\\
					&=	\left\{
							\begin{aligned}
								1 + t0_{i-1} - t0_{i-1}	&\quad if \ t1_{i-1}>0	\\
								t0_{i-1} + 0 - t1_{i-1} &\quad if \ t1_i=0
							\end{aligned}
						\right.	\\
					&
	\end{align*}
	
\end{enumerate}


\end{document}

