\section{Problem Definition}
\label{sec:definition}
In this section, we formally define the \textit{Latency-oriented Task Completion} (LTC) problem.
We first present the LTC problem in the offline scenario and prove it is NP-hard.
Then we define the LTC problem in the online scenario, which is practical in many spatial crowdsourcing applications.

\subsection{Basics}
This subsection introduces important concepts which will be used throughout this work.

\begin{definition}[Micro Task]
\label{def:task}
A micro task (``task'' for short) is denoted by $t = <\boldsymbol{l}_t, \epsilon>$, where $\boldsymbol{l}_t$ is the location of $t$ with category $c_t$ and $\epsilon$ is a constant indicating the maximum tolerable error rate of task $t$.
\end{definition}

We assume a micro task requires a binary answer from workers, which is common in check-in based spatial crowdsourcing platforms such as Facebook Editor\footnote{https://www.facebook.com/editor}.
Without of loss generality, we denote $+1$ as a ``YES'' and $-1$ as a ``NO''.
\rev{In this work, we assume a constant tolerable error rate for all tasks for simplicity because this error rate is usually specified by the platform based on business logics and experiences.}

\begin{definition}[Crowd Worker]
\label{def:worker}
A crowd worker (``worker'' for short) $w = <o_w, \boldsymbol{l}_w, p_w, K>$, is the $o_w$-th person who checks in with location $\boldsymbol{l_w}$ on the platform.
$p_w$ is the historical accuracy of the worker and $K$ is a constant indicating the maximum number of tasks the worker can perform.
\end{definition}

\rev{In this work, we assume the same capacity for each worker.
This assumption is reasonable because the capacity of workers is usually estimated by the platform based on large amounts of surveys and human-computer interaction studies.}

\begin{definition}[Predicted Accuracy]
\label{def:acc}
The predicted accuracy that a worker $w$ performs task $t$ is measured by an accuracy function $Acc(w, t) \in [0,1]$, which takes the worker's historical accuracy and the locations of the worker and the task as input.
\end{definition}

In this work, we assume an accuracy function as follows.
\begin{equation}
\label{equ:predictAcc}
Acc(w, t) = \frac{p_w}{1 + e^{-(d_{max} - \|\boldsymbol{l}_w, \boldsymbol{l}_t\|)}}
\end{equation}
where $d_{max}$ is the largest Euclidean distance that workers are able to perform the tasks with high accuracy.
Other accuracy functions can also apply in our problem.

\begin{definition}[Task Completion]
\label{def:tc}
Given a task $t$ and a set of workers $W_{t}$ assigned to $t$, the platform determines the result of $t$ by taking a weighted majority voting among the workers.
That is, $\ell_t = sign(\sum_{w \in W_{t}}{weight_{w,t} \ell_{w,t}}), \ell_{w,t} \in \{+1, -1\}$.
We call a task which reaches the error rate $\epsilon$ as completed.
\end{definition}

Note that according to the Hoeffding's inequality~\cite{Ho2013}, when $weight_{w,t} = 2Acc(w,t) - 1$ and $\sum_{w \in W_t}(2Acc(w,t) - 1)^2 \ge 2\ln(1/\epsilon)$, the probability of an error in task $t$ is less than $\epsilon$.
We define $\delta = 2\ln(1/\epsilon)$ for brevity.
Based on the above definition of $\delta$, each task should be performed at least $\delta$ times to obtain the tolerable error rate.
Note that when $Acc(w,t) = 0.5$, $(2Acc(w,t) - 1)^2$ becomes zero.
To make sure that all tasks can reach the tolerable error rate, we assume that the value of $(2Acc(w,t) - 1)^2$ falls into $[\alpha, 1]$ and $\alpha > 0$ is a small number (\eg the precision of $\epsilon$).
For ease of presentation, we denote $Acc^{*}(w,t) = \min\{\alpha,\ (2Acc(w,t) - 1)^2\}$.

\begin{definition}[Task Latency]
\label{def:taskLatency}
The latency of a task is assessed by the completion time of a task.
Specifically, the latency of a task $L_t$ is denoted as the index of the last person on the platform who performs $t$.
Note that $L_t = \max_{w \in W_t} o_w$, if $\sum_{w \in W_t} Acc^{*}(w,t) \ge \delta$.
\end{definition}

\subsection{LTC in Offline Scenario}
Based on the basic concepts above, we define the LTC problem in the offline scenario as follows.
\begin{definition}[Offline LTC]
Assume a set of tasks $T$ and a tolerable error rate $\epsilon$.
Each task $t$ has a location $\boldsymbol{l}_t$.
Also assume a set of workers $W$, where each worker $w$ is associated with an index $o_w$, a location $\boldsymbol{l}_w$, a historical accuracy $p_t$ and capacity $K$.
Furthermore assume an accuracy function $Acc(w, t)$ which predicts the accuracy of $w$ performing $t$ with precision $\alpha > 0$.
The latency-orientated task completion (LTC) problem in the \textit{offline} scenario is to find an arrangement $M$ among tasks and workers to minimize the maximum latency of all tasks $MinMax(M) = \max_{t \in T}{\max_{w \in W_t} o_w}$ that meets the following constraints.
\begin{itemize}
  \item
  Invariable constraint: once a task $t$ is assigned to a worker $w$, the arrangement $(t, w)$ cannot be changed.
  \item
  Capacity Constraint: $\sum_{t \in T} \mathbbm{1}(w \in W_t) \le K, \forall w \in W$.
  \item
  Error Rate Constraint: $\sum_{w \in W_t} Acc^{*}(w,t) \ge \delta, \forall t \in T$ with $\delta = 2\ln(1/\epsilon), Acc^{*}(w,t) = \min\{\alpha,\ (2Acc(w,t) - 1)^2\}$.
\end{itemize}
\end{definition}

\begin{example}
Back to Example~\ref{exa:introExa}.
Since each time only two tasks can be assigned to a worker, $K = 2$.
To obtain a minimal accuracy of 83\%, we need $\delta = 2\ln(1.0/(1.0-0.83)) \approx 3.544$.
The predicted accuracies in \tabref{table:predictAcc} are the values for $Acc(w_i,t_j), \forall w_i \in W, t_j \in T $.
\end{example}

\begin{theorem}
The offline LTC problem is NP-hard.
\end{theorem}
\begin{proof}
We prove the NP-hardness of the offline LTC problem leveraging the 3-partition problem.
The 3-partition problem~\cite{ThreePar} is a well-known NP-complete problem.
We first reduce the 3-partition problem to the decision version of the offline LTC problem.
An instance of the 3-partition problem is as follows.
Given a list of $3m$ positive integers $X = \{x_1, x_2, \cdots, x_{3m}\}$ with $\sum_{i=1}^{3m} x_i = mB$ and each $x_i$ satisfying $\frac{B}{4} < x_i < \frac{B}{2}$, the 3-partition problem aims to decide whether it is possible to partition $X$ into $m$ triples such that each one sums to exactly $B$.
According to an instance of 3-partition problem, an instance of the offline LTC problem can be constructed as follows:
\begin{enumerate}[(1)]
  \item The $m$ triples of the 3-partition problem correspond to the first $m$ tasks with threshold $\epsilon = e^{-\frac{1}{2}}, \delta = 1$.
  \item The $3m$ elements of the list $X$ correspond to $3m$ workers.
  \item For a worker $w_i$, the function $Acc^{*}(,)$ has the same value as $Acc^{*}(w_i, t) = \frac{x_i}{B} \in (\frac{1}{4}, \frac{1}{2}), \forall t \in T$, but a worker can only preform $K = 1$ task.
\end{enumerate}
Given an instance of the above problem, we want to decide whether there is a feasible arrangement $M$ such that $MinMax(M) = 3m$.

Next, we prove that an instance of the 3-partition problem is YES if and only if an instance of the offline LTC problem is YES.
Since the value of $Acc^{*}(w, t)$ falls into the range $(\frac{1}{4}, \frac{1}{2})$, each task has to be performed by at least $\lceil \frac{\delta}{\max_{w,t} Acc^{*}(w, t)} \rceil > \lceil 1/{\frac{1}{2}} \rceil = 3$ workers.
However, there are only $3m$ workers but $m$ tasks.
So each task must be performed by exactly 3 workers and each worker must perform exactly one task ($K = 1$).
Otherwise at least one task cannot meet the error rate constraint.
Thereby, if there is a feasible arrangement for the offline LTC problem, we will get $m$ triples.
Since a worker $w$ performs all the tasks with the same value $Acc^*(w,) = \frac{x_i}{B}$ and only one task is assigned to this worker, each element $\frac{x_i}{B}, x_i \in X$ appears only once in the arrangement $M$.
Therefore, the $m$ triples are the partition of $X' = \{\frac{x_i}{B} | x_i \in X\}$ because they are disjoint and they can cover $X'$.
Then we need to prove that every $Acc^{*}$ sum of these triples with their workers is exactly $\delta = 1$.
Since the arrangement is feasible, each of the $m$ sums is no less than $\delta = 1$ and the sum of all these values is $\sum_{x_i \in X} {\frac{x_i}{B}} = \frac{sum_{x_i \in X} x_i}{B} = m$.
Thus, each of these sum is exactly $1$.
That is, the sum of these three $x$ of the function $Acc^{*}$ is exactly $B$.
Hence we obtain a feasible instance of the 3-partition problem via the offline LTC problem.

From the justification above, the decision version of the offline LTC problem is NP-complete and the offline LTC problem is NP-hard.
\end{proof}

\subsection{LTC in Online Scenario}

We finally define the LTC problem in the online scenario.

\begin{definition}[Online LTC]
Assume a set of tasks $T$, and a tolerable error rate $\epsilon$, where each task $t$ is at location $\boldsymbol{l}_t$.
Also assume a set of workers $W$, where each worker $w$ is associated with an index $o_w$, a location $\boldsymbol{l}_w$, a historical accuracy $p_t$ and a constant capacity $K$.
Further assume an accuracy function $Acc(w, t)$ which predicts the accuracy of $w$ performing $t$ with precision $\alpha > 0$.
The \textit{online} LTC problem is to find an arrangement $M$ among tasks and workers to minimize the maximum latency of all tasks
$MinMax(M) = \max_{t \in T}{\max_{w \in W_t} o_w}$ such that
\begin{itemize}
  \item The assignment for a new worker $w$ must be decided before the next worker appears and cannot be revoked.
  \item The three constraints of offline LTC are satisfied.
\end{itemize}
\end{definition}

Compared with the offline LTC problem, the major difference is that in the online LTC problem the selection of the $K$ tasks for the new worker should be determined immediately and cannot be revoked.
This setting is more practical in many real-world spatial crowdsourcing applications such as Facebook.
For example, when a user checks-in on Facebook, Facebook will immediately send the user some questions about the checked-in places.
However, the online LTC problem is also more challenging.
As next, we first design algorithms for the offline LTC problem and then explore solutions to the online LTC problem.
Note that the algorithms need to output both the arrangements among tasks and workers, as well as the maximum latency of all tasks.
The maximum latency of all tasks can be measured by the last index of the worker who performs the task before the set of tasks are completed.
Hence we will use the last (maximum) index of the worker in the task arrangement as the output of the algorithms in the rest of this paper.


\begin{table}[t]
	\centering
	\caption{Summary of symbol notations}
	\label{table:notations}
	\begin{tabular}{|c|c|}
		\hline
		Notation & Description \\
		\hline
		$T, W$ & a set of tasks and workers \\
		\hline
		$l_t, l_w$ & the location of task and worker \\
		\hline
		$K$ & the capacity of worker \\
		\hline
		$\epsilon,\delta$ & the tolerate error rate of task, the value of $2\ln(1/\epsilon)$ \\
		\hline
		$p_w$ & the historical accuracy of worker \\
		\hline
		$Acc(w,t)$ & the predicted accuracy for a worker performing a task\\ 
		\hline
		$Acc^*(w,t)$ & the value of $(2Acc(w,t) -1)^2$ \\
		\hline
	\end{tabular}
\end{table}

Table~\ref{table:notations} lists the notations used throughout the paper.