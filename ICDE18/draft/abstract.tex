\begin{abstract}
with the tremendous development of mobile Internet
and sharing economy, Location Based Service(LBS) has been utiltized by
more and more spatial crowdsourcing(SC) platforms.
Specifically, as soon as the users login the platform with their check-in places,
platform may send them some simple tasks just around their check-in places.
By this way, these SC platforms can collect or update some important information about these locations.
Quality and Latency are usually some of the most important considerations of such platforms.
Many existing studies focus on how to control the latency by dynamically adjusting the cost of task and task selection.
However, since people always volunteer to peform these tasks on such platforms, these solutions are not applicable.
On the other hand, even though there are many studies focus on quality control.
But since these platforms usually expect to achieve the good quality as fast as possible, which indicates organize the users very thoughtfully,
none of existing workers focus on these scenario.
In this paper, to solve the problems between quality and latency, 
we first formally define this problem particularly in SC platform, 
called the Latency-oriented Information Acquisition(LIA) problem.
Unfortunately the problem is proven as NP-hard, we first devise an MinimumCostFlow-based algorithm to solve the problem in offline scenario with approximation ratio guarantees.
Since these services are common in offline scenario, we also design two greedy algorithms with competitive ratio guarantees to solve the problem in online scenario.
Finally, we verify the effectiveness and efficiency of the proposed methods
through extensive experiments on real and synthetic datasets.
\end{abstract}