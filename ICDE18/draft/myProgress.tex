%-*- coding:UTF-8 -*-
% 公式推导的日常.tex
\documentclass[UTF8]{ctexart}
\usepackage{geometry}
\usepackage{enumerate}
\usepackage{amsmath}
\usepackage{amssymb}
\usepackage{dsfont}
\usepackage{amsthm}
\usepackage{listings} %插入代码
\usepackage{xcolor} %代码高亮
\usepackage{blkarray}
\usepackage{diagbox}
\usepackage{tabularx}
\usepackage{graphicx}
\usepackage{caption}
\usepackage{subcaption}
\usepackage{float}
\usepackage{color}
\usepackage{multirow}
\usepackage{color}
\usepackage[all,pdf]{xy}
\usepackage{verbatim}   %comment
\usepackage{cases}
%\usepackage{clrscode3e}
\usepackage[ruled, vlined, linesnumbered]{algorithm2e}
\usepackage{bbm}

% Some setup
\pagestyle{plain}
\geometry{a4paper, top=2cm, bottom=2cm, left=2cm, right=2cm}
\CTEXsetup[format={\raggedright\bfseries\Large}]{section}
\lstset{numbers=left, %设置行号位置
        numberstyle=\small, %设置行号大小
        keywordstyle=\color{blue}, %设置关键字颜色
        commentstyle=\color{purple}, %设置注释颜色
        %frame=single, %设置边框格式
        escapeinside=``, %逃逸字符(1左面的键),用于显示中文
        breaklines, %自动折行
        extendedchars=false, %解决代码跨页时,章节标题,页眉等汉字不显示的问题
        %xleftmargin=2em,xrightmargin=2em, aboveskip=1em, %设置边距
        tabsize=4, %设置tab空格数
        showspaces=false %不显示空格
       }

% About math
\newcommand{\rmnum}[1]{\romannumeral #1}
\newcommand{\Emph}{\textbf}
\newcolumntype{Y}{>{\centering\arraybackslash}X}
\newcommand{\resetcounter}{\setcounter{equation}{0}}
\newcommand{\equsuf}{\equiv_{\textit{Suff(x)}}}
\newcommand{\eqsuf}[1][x]{\equiv_{\textit{Suff(#1)}}}
\newcommand{\len}[1][x]{\textit{length}_{#1}}

\newtheorem{definition}{Definition} %[section]
\newtheorem{example}{Example} %[section]
\newtheorem{lemma}{Lemma} %[section]
\newtheorem{theorem}{Theorem} %[section]
\newtheorem{corollary}{Corollary} %[section]

\begin{document}

\title{\Huge 公式推导的日常}
\vspace{2cm}
\author{\Large Trasier}
\date{\today}
\maketitle

\section{推导}

\begin{align*}
    \text{Minimize } z &= \sum\limits_{i=1}^m y_i \\
                    y_i &= \left \{
                                \begin{aligned}
                                    1, &\text{ if used} \\
                                    0, &\text{ otherwise}
                                \end{aligned}
                            \right . \\
                    \sum\limits_{j=1}^n x_{ij} &= K \\
                    \sum\limits_{i=1}^m x_{ij}*u_j &\ge B
\end{align*}


\begin{align*}
    ratio &\le (\lceil(n/C)\rceil * \lceil(S/Umin)\rceil) / ((n/C) * \lceil(S/Umax)\rceil)  \\
          &\le (1+C/n) * (Umax/Umin + Umax/S) \\
          &\le (Umax/Umin + Umax/S)
\end{align*}

\newpage
\begin{enumerate}[(1)]
	
	\item preliminaries
	
	\begin{align*}
		S &\ge U_{max}					  \\
		n &\ge c						  \\
		q &= \left \{
				\begin{aligned}
					c,    &\text{ if } n\%c = 0 \\
					n\%c, &\text{ if } n\%c \neq 0
				\end{aligned}
			 \right . \\
		e &= \lceil \frac{S}{U_{min}} \rceil \\
		d &= \lceil \frac{S}{U_{max}} \rceil \\
		r &= \lceil \frac{n}{c}    \rceil \\
		\frac{e}{d} &= \left \{
						\begin{aligned}
							U_{max}, &\text{ if } U_{min} \ge 1 \\
							(1+\frac{U_{max}}{U_{min}}), &\text{ if } U_{min} > 0
						\end{aligned}
					\right .
	\end{align*}
	
	\newpage
	\item \Emph{Theorem}
	\begin{definition}[Specialize WSSC problem]
	假设初始状态存在$n$个task, 阈值为$\delta$, 一段时间后,每个task还分别需要$S = \{s_1, s_2, \cdots, s_n\}$的
	可信度才可以达到阈值$\delta$。不妨令$m = \sum\limits_{i=1}^{n} {\mathbbm{1} s_i>0}, S_{remain} = \sum_{i} {s_i}$
	从这时刻起,后续的全部工人对所有问题的可信度都为$r$,求至少还需要多少时间可以使得
	全部任务达到阈值?
	\end{definition}
	
	\begin{theorem}[WSSC* bound]
	\label{thm:optBound}
	The answer of WSSC problem is at least $\lceil \frac{S_{remain}}{K*r} \rceil$
	\end{theorem}
	
	\begin{proof}
	由于每个问题的可信度相同,则每个问题$t_i$至多还需要$\lceil \frac{s_i}{r} \rceil$个工人回答就可以达到阈值。
	不妨令$c_i = \lceil \frac{s_i}{r} \rceil$.
	则该问题可以看成一个$P|pmtn|C_{max}$的调度问题,根据McNaughton's Rule而该问题可以在多项式时间内求得最优解。
	因此,该问题可以看成有$m$个线程,每个线程还需要$c_i$的处理时间,共有$K$台处理器可以调度这些问题。如果调度,
	使得最大等待时间最小。
	因此有,$OPT_{pmtn} = \max \{ (1/K)\sum{c_i}, \max{c_i} \}$.
	所以上述定理的最优解为$OPT = OPT_{pmtn}$,进而有
	\begin{align*}
		OPT &= OPT_{pmtn} \\
			&= \max \{ (1/K)\sum{c_i},\ \max{c_i} \} \\
			&= \max \{ (1/K)\sum{\lceil \frac{s_i}{r} \rceil},\  \max{\lceil \frac{s_i}{r} \rceil} \}
	\end{align*}
	所以,可以得到$OPT$的下界$OPT_{lb}$
	\begin{align*}
		OPT_{lb} &\ge \max \{ (1/K)\sum{\frac{s_i}{r}},\ \max{\frac{s_i}{r}} \} \\
				 &\ge \max \{ \frac{S_{remain}}{rK},\ \max{\frac{s_i}{r}} \} \\
				 &\ge \max \{ \frac{S_{remain}}{rK},\ \frac{\max{s_i}}{r} \} \\
				 &\ge \max \{ \frac{S_{remain}}{rK},\ \frac{\text{avg }{s_i}}{r} \} \\
				 &\ge \max \{ \frac{S_{remain}}{rK},\ \frac{\frac{S_{remain}}{m}}{r} \} \\
				 &\ge \max \{ \frac{S_{remain}}{rK},\ \frac{S_{remain}}{rm} \} \\
				 &\ge \frac{S_{remain}}{r\min\{m,K\}} \\
	\end{align*}
	进一步,可以得到$OPT$的上界$OPT_{ub}$
	\begin{align*}
		OPT_{ub} &\le \max \{ (1/K)\sum{(\frac{s_i}{r}+1)},\ \max{\frac{s_i}{r}+1} \} \\
				 &\le \max \{ (1/K)\sum{\frac{s_i}{r}}+\frac{m}{K},\ \max{\frac{s_i}{r}}+1 \} \\
				 &\le \max \{ \frac{s_{remain}}{rK}+\frac{m}{K},\frac{\max{s_i}}{r}+1  \} \\
				 &\le \max \{ \frac{s_{remain}}{rK}+\frac{m}{K},\frac{\max{s_i}}{r}+\frac{K}{K}  \} \\
				 &\le \max \{ \frac{s_{remain}}{rK},\ \frac{\max{s_i}}{r} \} + \frac{\max\{m,K\}}{K} \\
				 &\le \max \{ \frac{s_{remain}}{rK},\ \frac{\delta}{r} \} + \frac{n}{K} \\
	\end{align*}
	\end{proof}
	
	\newpage
	\begin{theorem}[MCMF ratio]
	Assume $K \le \frac{n}{2}, U_{max} \le \frac{\delta}{2}$, the approximation of MCMF algorithm is $0.5U_{max} + O(1)$.
	\end{theorem}
	\begin{proof}
	令$t_0 = \lceil \frac{n}{K} \lfloor \frac{\delta}{U_{max}} \rfloor \rceil0$,
	$t_1 = \lceil \frac{n}{K} \lceil \frac{\delta}{U_{max}} \rceil \rceil$.
	因为每个$v$到汇点$ed$的容量都为$\lfloor \frac{\delta}{U_{max}} \rfloor$. 
	不妨假设执行最小花销最大流后得到的流量为$flow$, 花销为$cost$.
	而此时,$flow = n * \frac{\delta}{U_{max}}$.
	因此采用该网络流表示的安排,一定可以增加可信度$SANS_{increase} = |cost|$.
	由定理~\ref{thm:optBound}可知,$OPT \ge \frac{nS}{U_{max}K} \ge t_0$.
	不妨令此时最优解的可信度增加量$SOPT_{increase}$。
	根据最小花销最大流可知,$-SOPT_{increase} \ge cost \ge -SANS_{increase}$,从而有$SOPT_{increase} \le SANS_{increase}$.
	如果此时,$SOPT_{increase} \ge (1-\rho) n\delta$
	则有$SANS_{increase} \ge  (1-\rho) n\delta, SANS_{remain} \le  \rho n\delta$.
	再次利用定理~\ref{thm:optBound}可知
	\begin{align*}
		ANS &\le \max \{ \frac{\rho n\delta}{U_{min}K},\ \frac{\delta}{U_{min}} \} + \frac{n}{K} \\
			&\le \max \{ \frac{\rho n\delta}{K},\ \delta \} + \frac{n}{K} \\
		OPT &\ge \frac{n\delta}{U_{max}K} \\
		ratio &= \frac{ANS}{OPT} \\
			  &\le \frac{\max \{ \frac{\rho n\delta}{K},\ \delta \} + \frac{n}{K}}{\frac{n\delta}{U_{max}K}} \\
			  &\le \max \{ \rho U_{max},\ \frac{K}{n}U_{max} \} + \frac{U_{max}}{\delta} \\ 
			  &\le \max \{ \rho,\ \frac{K}{n} \} U_{max} + O(1)
 	\end{align*}
	若$\frac{K}{n} \le \rho$,则有此时$ratio \le \rho U_{max}$
	如果$SOPT_{increase} < (1-\rho) n\delta, SOPT_{remain} >  \rho n\delta$,
	当又过了$t_1 - t_0$个工人,则OPT最多增加可信度$\Delta$为
	$\frac{n}{K} * (\lceil \frac{\delta}{U_{max}} \rceil - \lfloor \frac{\delta}{U_{max}} \rfloor) \times KU_{max} \le nU_{max} (\lceil \frac{\delta}{U_{max}} \rceil - \lfloor \frac{\delta}{U_{max}} \rfloor)$
	易知,当$\delta\%U_{max} = 0$时,$\Delta = 0$,否则$\Delta = nU_{max}$,
	因此,$t_1$时刻,$SOPT_{remain}' > n\delta(\rho - \frac{U_{max}}{\delta})$.
	若$\frac{U_{max}}{\delta} \le \rho$,则此时$SOPT_{remain}' > 0$.
	从而由~\ref{thm:optBound}可知
	\begin{align*}
		ANS &\le \max \{ \frac{n\delta}{U_{min}K},\ \frac{\delta}{U_{min}} \} + \frac{n}{K} \\
			&\le \max \{ \frac{n\delta}{K},\ \delta \} + \frac{n}{K} \\
			&\le \frac{n(\delta+1)}{K} \\
		OPT &\ge t_1 + \frac{n\delta(\rho - \frac{U_{max}}{\delta})}{U_{max}K} \\
			&\ge \lceil \frac{n}{K} \lceil \frac{\delta}{U_{max}} \rceil \rceil + \frac{n}{K}\frac{\delta(\rho - \frac{U_{max}}{\delta})}{U_{max}} \\
			&\ge \frac{n}{K} (\frac{\delta}{U_{max}} + \frac{\delta(\rho - \frac{U_{max}}{\delta})}{U_{max}}) \\
 			&\ge \frac{n}{K} \frac{\delta}{U_{max}} (1 + \rho - \frac{U_{max}}{\delta})  \\
		ratio &= \frac{ANS}{OPT} \\
			  &\le \frac{\frac{n(\delta+1)}{K}}{\frac{n}{K} \frac{\delta}{U_{max}} (1 + \rho - \frac{U_{max}}{\delta})} \\
			  &\le \frac{1}{1 + \rho - \frac{U_{max}}{\delta}} U_{max} + O(1) \\
	\end{align*}
	不妨令$\frac{U_{max}}{\delta} \le 1-p, p \in (0,1)$
	从而$ratio = \max\{ \rho,\ \frac{1}{\rho + p} \} U_{max}$,
	不妨令$\rho = \frac{1}{\rho + p}$,从而有$\rho^2 + p\rho - 1 = 0$.
	解方程得$\rho = \frac{1}{2}(\sqrt{4+p^2}-p)$.
	从而有
	\begin{center}
	\begin{tabular}{|c|c|c|}
		\hline
			$\frac{U_{max}}{\delta}$的上界 &$\rho$ &$\rho$的小数点后三位 \\
		\hline
			$\frac{1}{2}$ &$\frac{\sqrt{17}-1}{2}$ &0.781 \\
		\hline
			$\frac{1}{3}$ &$\frac{\sqrt{10}-1}{3}$ &0.720 \\
		\hline
			$\frac{1}{4}$ &$\frac{\sqrt{73}-3}{8}$ &0.693 \\
		\hline
			$\frac{1}{10}$ &$\frac{\sqrt{481}-9}{20}$ &0.647 \\
		\hline
	\end{tabular}
	\end{center}
	很明显$\rho > \frac{\sqrt{5}-1}{2}$. 当且仅当$\delta\%U_{max} = 0$时,可以取到这个界。
	\begin{align*}
		ANS &\le \frac{n(\delta+1)}{K} \\
		OPT &\ge \frac{n}{K} \frac{\delta}{U_{max}} (1 + \rho)  \\
		ratio &= \frac{ANS}{OPT} \\
			  &\le \frac{\frac{n(\delta+1)}{K}}{\frac{n}{K} \frac{\delta}{U_{max}} (1 + \rho)} \\
			  &\le \frac{1}{1 + \rho} U_{max} + O(1) \\
	\end{align*}
	此时,令$\frac{1}{1 + \rho} = \rho$可解得$\rho = \frac{\sqrt{5}-1}{2} \approx 0.618$.
	则此时$ratio \le \frac{\sqrt{5}-1}{2} U_{max}$同时$\frac{K}{n} \le 0.618$.
	\end{proof}
	
	\newpage
	\item \Emph{Largest Reliability First}
	\begin{algorithm}[t]
		\SetKwInOut{Input}{input}\SetKwInOut{Output}{output}
		\Input{$T, W, R(.,.), K, \delta$}
		\Output{A feasible arrangement $M$}
		$Q \leftarrow \emptyset, n \leftarrow |T|, S[1 \ldots n] \leftarrow \{\delta,\delta,\cdots,\delta\}$\;
		\ForEach{new arrival worker $w$}{
			\For{$i=1$ \emph{\KwTo} $|T|$ and $n > 0$} {
				push $\{R(T[i], w), i\}$ into $Q$\;
				\If{$|Q| > K$}{
					extract the least reliability pair from $Q$\;
				}
			}
			\While{$Q \neq \emptyset$}{
				extract a reliability pair $\{r, i\}$ from $Q$\;
				$M \leftarrow M \cup (w, T[i])$\;
				\If{$S[i] \le r$} {
					$n \leftarrow n - 1, S[i] \leftarrow 0$\;
				}
				\Else{
					$S[i] \leftarrow S[i] - r$\;
				}
			}
		}
		\If{$n > 0$}{
			\KwRet{$\emptyset$}
		}
		\Else{
			\KwRet{$M$}
		}
	\caption{Largest Reliability First}
	\label{alg:LRF}
	\end{algorithm}
	
	\newpage
	\item \Emph{offline}
	\begin{align*}
		\text{dummy avg ratio} &\le \frac{\text{worst Greedy value}}{\text{best OPT value}} \\
						   &< \frac{e * (c+2c+\cdots+r*c)}{d * (c+2c+\cdots+(r-1)*c)} \\
						   &< \frac{e}{d} (1 + \frac{2}{r-1})
	\end{align*}
	由于$r = \lceil \frac{n}{c}    \rceil$ \\
	当$r = 1$时,贪心是最优解。				\\
	当$r > 4 \ge 5$时,$\text{dummy avg ratio} < \frac{e}{d} * 1.5$ \\
	当$r = 2$时,\\
	\begin{align*}
		\text{dummy avg ratio} &\le \frac{e}{d} \cdot \frac{c + (n-c)*2}{c + (n-c)*\frac{n}{c}} \\
								&\le \frac{e}{d} \cdot \frac{c + 2n - 2c}{c+\frac{n^2}{c} - n} \\
								&\le \frac{e}{d} \cdot \frac{2n-c}{\frac{n^2}{c}+c-n} \\
								&\le \frac{e}{d} \cdot \frac{1.5}{\frac{n}{c}+\frac{c}{n}-1} \\
								&\le \frac{e}{d} \cdot \frac{1.5}{(\sqrt{\frac{n}{c}} - \sqrt{\frac{c}{n}})^2+1} \\
								&<   \frac{e}{d} * 1.5
	\end{align*}
	当$r = 3$时,\\
	\begin{align*}
		\text{dummy avg ratio} &\le \frac{e}{d} \cdot \frac{c + 2c + (n-2c)*3}{c + 2c + (n-2c)*\frac{n}{c}} \\
								&\le \frac{e}{d} \cdot \frac{3n-3c}{\frac{n^2}{c}-2n+3c} \\
								&\le \frac{e}{d} \cdot \frac{2n}{\frac{n^2}{c}-2n+3c} \\
								&\le \frac{e}{d} \cdot \frac{2}{\frac{n}{c}+3\frac{c}{n}-2} \\
								&\le \frac{e}{d} \cdot \frac{2}{(\sqrt{\frac{n}{c}} - \sqrt{3c}{n})^2+2\sqrt{3}-2} \\
								&\le \frac{e}{d} \cdot \frac{1}{0.5 * (\sqrt{\frac{n}{c}} - \sqrt{3c}{n})^2 + \sqrt{3} - 1} \\
								&< \frac{e}{d} * 1.367
	\end{align*}
	当$r = 4$时,\\
	\begin{align*}
		\text{dummy avg ratio} &\le \frac{e}{d} \cdot \frac{c + 2c + 3c + (n-3c)*4}{c + 2c + 3c + (n-3c)*\frac{n}{c}} \\
								&\le \frac{e}{d} \cdot \frac{4n-6c}{\frac{n^2}{c}-3n+6c} \\
								&\le \frac{e}{d} \cdot \frac{2.5n}{\frac{n^2}{c}-3n+6c} \\
								&\le \frac{e}{d} \cdot \frac{2.5}{\frac{n}{c}+\frac{6c}{n}-3} \\
								%&\le \frac{e}{d} \cdot \frac{2.5}{(\sqrt{\frac{n}{c}} - \sqrt{6c}{n})^2+2\sqrt{6}-3} \\
								&\le \frac{e}{d} \cdot \frac{2.5}{(\sqrt{\frac{n}{c}} - \sqrt{\frac{6c}{n}})^2+2\sqrt{6}-3} \\
								&< \frac{e}{d} * 1.316
	\end{align*}
	所以,$\text{dummy avg ratio} < \frac{e}{d} * 1.5$.
	因为,我们这里采用的是贪心,那么前一半点的完成时间一定小于后于半点的完成时间。
	所以,至少有一半点的完成时间$t_{half} < \frac{e}{d}*OAVG$,否则$dummy avg ratio$一定不成立。
	那么,无论\Emph{offline}或者\Emph{online}贪心都可以重新证明。
	\begin{align*}
		\text{greedy ratio} &\le \frac{\text{worst Greedy value}}{\text{best OPT value}} \\
								&< \frac{t_{half} + \lceil e*\frac{\frac{n-1}{2}}{c} \rceil}{OPT} \\
								&< \frac{t_{half}}{OPT} + \frac{ \lceil e*\frac{(n-1)}{2*c} \rceil }{OPT} \\
								&< \frac{t_{half}}{OAVG} + \frac{e*\frac{n}{2*c}}{d*\frac{n}{c}} \\
								&< 0.5\rho*U_{max} + \frac{e}{d} * \frac{1}{2} \\
								&< 0.5\rho*U_{max} + 0.5*U_{max} \\
                                &< 0.5*(\rho+1)*U_{max}
 	\end{align*}
	因为$\rho<1.5$,所以$ratio < 0.5*(1.5+1)*U_{max} = 1.25U_{max}$
	
    \newpage
    \item \Emph{round robin}

    \begin{align*}
        \text{ratio} &\le \frac{\text{worst RR value}}{\text{best OPT value}} \\
					 &\le \frac{\lceil \frac{n*e}{c} \rceil}{\lceil \frac{n*d}{c} \rceil} \\
					 &\le \frac{e * \frac{n+c}{c}}{d * \frac{n}{c}} \\
					 &\le \frac{e}{d} * (1 + \frac{n}{c}) \\
					 &\le \frac{e}{d} * 2
    \end{align*}

    \newpage
    \item \Emph{Shortest Remaining Processing Time}

    \begin{align*}
        \text{ratio} &\le \frac{\text{worst SRPT value}}{\text{best OPT value}} \\
                     &\le \frac{\lceil \frac{n}{c} * e \rceil}{\lceil \frac{n*d}{c} \rceil} \\
					 &\le \frac{e * \frac{n+c}{c}}{d * \frac{n}{c}} \\
					 &\le \frac{e}{d} * 2
    \end{align*}
	$ratio_{RR} \le ratio_{SRPT}$
	
	\newpage
	\item \Emph{Bounded Ratio of online}

	当考虑$n=2*c, S=k*U_{max}$时,前(2*k-1)个人的可信度都为$U_{max}$。此时,至少一个问题达到了界。
	不失一般性让问题1达到界,下一个工人的可信度为$(U_{max}, U_{min})$。
	
	\begin{align*}
		S   &\ge U_{max} \rightarrow k \ge 1 \\
		ANS &=\frac{S}{U_{max}} + \frac{S-U_{max}}{U_{max}} + \frac{U_{max}}{U_{min}} \\
		OPT &= 2*\frac{S}{U_{max}} \\
		ratio  &\ge \frac{ANS}{OPT} \\
			   &\ge \frac{2k-1}{2k} + \frac{U_{max}}{U_{min}} \\
			   &\ge 1 - \frac{1}{2k} + \frac{1}{2k} \cdot \frac{U_{max}}{U_{min}} \\
			   &\ge \frac{1}{2} + \frac{1}{2k} \cdot \frac{U_{max}}{U_{min}} \\
			   &> 0.5U_{max}
	\end{align*}
	
	\newpage
    \item \Emph{MLF}
	
\end{enumerate}	

\newpage
\Emph{Find Minmax}
\begin{enumerate}[I]
	\item \Emph{offline method}
	
	\begin{enumerate}[(a)]
		\item 选择前$t_{0} = \lceil \frac{n * \lceil \frac{S}{U_{max}} \rceil}{c} \rceil$个工人回答问题,建立网络流,
		将工人看成点$U$,将问题看成点$V$,源点$s$到$U$的流量是$c$,$V$到汇点$t$的流量为$\lceil \frac{S}{U_{max}} \rceil$,
		$U$到$V$的流量为$1$、费用为$-Utility(u_i, v_i)$,解得这个网络流的流量为$flow_{0}$,费用为$cost_{0}$。
		显然有
		\begin{align*}
			flow_{0} &\le n * \lceil \frac{S}{U_{max}} \rceil \\
			|cost_{0}| &\ge SOPT_{subgraph}
		\end{align*}
		如果$|cost_{0}| \ge \frac{nS}{2}$,则有
		\begin{align*}
			|cost_{0}| &\ge \frac{nS}{2} \\
			ANS &\le t_0 + \lceil \frac{\frac{n}{2} * \lceil \frac{S}{U_{min}} \rceil}{c} \rceil \\
				&\le t_0 + \lceil \frac{nS}{2c} \rceil \\
			OPT &\ge t_0
		\end{align*}
		所以,此时有近似比
		\begin{align*}
			ratio &= \frac{ANS}{OPT} \\
				  &\le \frac{t_0 + \lceil \frac{n*S}{2c} \rceil}{t_0} \\
				  &\le 1 + \frac{\lceil \frac{n*S}{2c} \rceil}{\lceil \frac{n * \lceil \frac{S}{U_{max} \rceil} \rceil}{c}} \\
				  &\le 1 + \frac{\frac{n*S}{2c}}{\frac{n*S}{cU_{max}}} \\
				  &\le 1 + 0.5U_{max}
		\end{align*}
		如果$cost_{0} < \frac{nS}{2}$,则$SOPT_{subgraph} < \frac{nS}{2}$
		则选择$t_1 = t_{0} + \lceil \frac{\frac{n}{2} * \lceil \frac{S}{U_{max}} \rceil}{c} \rceil \ge 1.5t_{0}$建立网络流,
		解得这个网络流的流量为$flow_{1}$,费用为$cost_{1}$。
		\begin{align*}
			flow_{1} &\le n * \lceil \frac{S}{U_{max}} \rceil \\
			|cost_{1}| &\ge SOPT_{subgraph}
		\end{align*}
		如果$cost_{1} \ge \frac{nS}{2}$,则
		\begin{align*}
			ANS &\le 1.5t_{0} + \lceil \frac{\frac{n}{2} * \lceil \frac{S}{U_{min}} \rceil}{c} \rceil \\
				&< 1.5t_{0} + \frac{nS}{2c} \\
			OPT &\ge 1.5t_{0}
		\end{align*}
		所以,有近似比
		\begin{align*}
			ratio &= \frac{ANS}{OPT} \\
				  &< \frac{1.5t_{0} + \frac{nS}{2c}}{1.5t_{0}} \\
				  &< 1 + \frac{\frac{nS}{2c}}{{\frac{3}{2} * \lceil \frac{n * \lceil \frac{S}{U_{max}} \rceil}{c} \rceil}}  \\
				  &< 1 + \frac{\frac{nS}{2c}}{{\frac{3}{2} \frac{nS}{U_{max}}}} \\
				  &< 1 + \frac{U_{max}}{3}
		\end{align*}
		相反地,如果$cost_{1} < \frac{nS}{2}$,则
		\begin{align*}
			ANS &\le \lceil \frac{n * \lceil \frac{S}{U_{min}} \rceil }{c} \rceil \\
				&\le \frac{nS}{c} \\
			% OPT &\ge 1.5t_{0} + 0.5t_{0} \\
				% &\ge 2t_{0}
			OPT &\ge 1.5t_{0}
		\end{align*}
		从而,有近似比
		\begin{align*}
			ratio &= \frac{ANS}{OPT} \\
				  &< \frac{\frac{nS}{c}}{1.5t_{0}} \\
				  &< \frac{\frac{nS}{c}}{\frac{3}{2} * \lceil \frac{n * \lceil \frac{S}{U_{max}} \rceil}{c} \rceil} \\
				  &< \frac{\frac{nS}{c}}{\frac{3}{2} * \frac{nS}{cU_{max}}} \\
				  &< \frac{2}{3}U_{max}
		\end{align*}
		综上所述,近似比$ratio = O(\frac{2}{3} U_{max})$
		
	\end{enumerate}

    \newpage
    只做一次网络流即可($V$到汇点的容量为$\lfloor \frac{S}{U_{max}} \rfloor$),如果当前的$|cost| \ge \rho * nS$,则有
    \begin{align*}
        ANS &\le t_{0} + \lceil \frac{(1-\rho)nS}{c} \rceil \\
        OPT &\ge t_{0} \\
        ratio &= \frac{ANS}{OPT} \\
              &< \frac{t_{0} + \lceil \frac{(1-\rho)nS}{c} \rceil}{t_{0}} \\
              &< 1 + \frac{1}{t_{0}} + \frac{ \frac{(1-\rho)nS}{c} }{ \frac{nS}{cU_{max}} } \\
              &< 1.5 + (1-\rho)U_{max}
    \end{align*}	
    如果,当前的$|cost| < \rho * nS$,则至少还需要$(1-\rho)t_{0}$才能完成,则有
    \begin{align*}
        ANS &\le \lceil \frac{nS}{c} \rceil \\
        OPT &\ge \lceil t_{0} + (1-\rho)t_{0} \rceil \\
            &\ge \lceil (2-\rho)t_{0} \rceil \\
        ratio &= \frac{ANS}{OPT} \\
              &< \frac{1 + \frac{nS}{c}}{(2-\rho)t_{0}} \\
              &< 1 +  \frac{\frac{ns}{c}}{(2-\rho) \frac{nS}{cU_{max}}} \\
              &< 1 + \frac{U_{max}}{2-\rho}  
    \end{align*}
    显然,$ratio = \max \{1.5+(1-\rho)U_{max}, 1+\frac{U_{max}}{2-\rho} \}$。不妨令$(1-\rho)U_{max} = \frac{U_{max}}{{2-\rho}}$。
    可以解得$\rho = \frac{3-\sqrt5}{2}$。
    从而有$ratio = 1.5 + (1-\frac{3-\sqrt5}{2})U_{max} = 1.5 + \frac{\sqrt5-1}{2}U_{max} \approx 1.5 + 0.618 U_{max}$.

    \newpage
	\item \Emph{online Greedy}

	\begin{enumerate}[(a)]	
	
		\item \Emph{Heaviest Comes First}
		
		Find the heaviest utility and insert into the tasks.
		不妨令$t_0 = \lceil \frac{n * \lfloor \frac{S}{U_{max}} \rfloor}{c} \rceil$,所以到达$t_{0}$时刻,
		一定有$S_{remain} \ge 0$,其中$S_{remain}$当且仅当$S \mod U_{max} = 0$。
		不妨令此时,$\sum\limits_{i=1}^{n}{S-S_{remain}} = cost_{greedy}$。
		显然有$cost_{MCMF} = cost_{greedy}$,同理,如果$cost_{greedy} \ge \rho nS$。
		\footnote{这里说花销有些不严密,因为后面推导ANS的式子用$S_{remain}$的均值就不对了,但是可以改成$\rho n$个点达到了容量}。
		\begin{align*}
			ANS &\le t_{0} + \lceil \frac{ \sum\limits_{i=1}^n S_{remain} }{c} \rceil \\
			ANS &\le t_{0} + \lceil \frac{(1-\rho)nS}{c} \rceil \\
			OPT &\ge t_{0} \\
			ratio &= \frac{ANS}{OPT} \\
				  &< \frac{t_{0} + \lceil \frac{(1-\rho)nS}{c} \rceil}{t_{0}} \\
				  &< 1 + \frac{1}{t_{0}} + \frac{ \frac{(1-\rho)nS}{c} }{ \frac{nS}{cU_{max}} } \\
				  &< 1.5 + (1-\rho)U_{max}
		\end{align*}
		如果$cost_{greedy} < \rho nS$,则有
		\begin{align*}
			ANS &\le t_{0} \\
			OPT &\ge t_{0} + \lceil \frac{ \sum\limits_{i=1}^n \lceil \frac{S_{remain}}{U_{max}} \rceil }{c} \rceil \\
				&\ge t_{0} + \lceil \frac{ \sum\limits_{i=1}^n {S_{remain}} }{c*U_{max}} \rceil \\
				&\ge t_{0} + \lceil \frac{(1-\rho)nS } {c*U_{max}} \rceil \\
				&\ge t_{0} + (1-\rho)t_{0} \\
			ratio &= \frac{ANS}{OPT}			
		\end{align*}
		$ratio = 1.5 + 0.618 U_{max}$
		
		\item \Emph{Benefit Most comes First}
		Find the most benefit utiliy and choose the tasks. Here,
		\[
			benefit = \min \{ utility, S_{remain} \}
		\]
		Obviously, the true benefit from this algorithm in $t_0$ is slightly higher then the \Emph{offline} method.
		In fact, within $t_0$ the procedure is the same as \Emph{Heaviest Comes First}.
		所以,近似比为$1.5 + \frac{\sqrt5-1}{2} U_{max} = 1.5 + 0.618U_{max}$。
		
		\item \Emph{Weighted Comes First}
		Assign the tasks according to their weighted ratio.
		\[
			weighted ratio = \frac{\min \{ utility, S_{remain} \}}{S_{remain}}
		\]
		\begin{align*}
		\end{align*}
		
	\end{enumerate}
	
    
    \newpage
	\item \Emph{Baseline Online Algorithm}
	\begin{enumerate}[(a)]
		\item \Emph{Least Remaining Time}
		
		Which task has the least remaining time, then assign the utility to this task.
		\begin{align*}
			ANS &\le \lceil \frac{n * \lceil \frac{S}{U_{min}} \rceil}{c} \rceil \\
				&\le \lceil \frac{n*S}{c} \rceil \\
			OPT &\ge \lceil \frac{n*\lceil \frac{S}{U_{max}} \rceil}{c} \rceil \\
			\text{ratio} &= \frac{ANS}{OPT} \\
						 &< \frac{\lceil \frac{n*S}{c} \rceil}{ \lceil \frac{n*\lceil \frac{S}{U_{max}} \rceil}{c} \rceil } \\
						 &< \frac{1 + \frac{n*S}{c}}{\frac{n*S}{cU_{max}}} \\
						 &< \frac{c*U_{max}}{n*S} + U_{max} \\
						 &< \frac{1}{\frac{n}{c} * \frac{S}{U_{max}}} + U_{max} \\
						 &< 1 + U_{max} 
		\end{align*}
		
		举例:
		\begin{center}
		\begin{tabular}{|c|c|c|c|c|c|c|}
			\hline
			2 &1 &1 &$\cdots$ &1 &100 &$\cdots$ \\
			\hline
			1 &100 &100 &$\cdots$ &100 &1 &$\cdots$ \\	
			1 &100 &100 &$\cdots$ &100 &1 &$\cdots$ \\	
			\quad &\quad &\quad &$\cdots$ &\quad &\quad &\quad  \\
			1 &100 &100 &$\cdots$ &100 &1 &$\cdots$ \\	
			\hline
		\end{tabular}
		\end{center}
	
		$ANS = 99+100*98, OPT = 100, \frac{ANS}{OPT} = 98.99$
		
		\item \Emph{Round robin}
		Assign the utility to a subset of tasks in this interval, and change to another subset in the next interval.
		\begin{align*}
			ratio = U_{max}
		\end{align*}
		举例同下。
		
		\item \Emph{Largest Processing Time}
		
		Assign the tasks which have the largest processing time.
		\begin{equation*}
			ratio < 1 + U_{max}
		\end{equation*}
		
		举例:
		\begin{center}
		\begin{tabular}{|c|c|c|c|c|}
			\hline
			1   &100 &1   &1 &$\cdots$ \\
			\hline
			1   &1   &100 &1 &$\cdots$ \\
			\hline
			100 &1   &1   &1 &$\cdots$ \\
			\hline
		\end{tabular}
		\end{center}
		
		$ANS = 300, OPT = 3, \frac{ANS}{OPT} = 100$
		
		\newpage
		\item \Emph{Largest Utility First}
		
		Assign the tasks which have the largest utility at the present.
		因为$t_0 = \lceil \frac{n * \lceil \frac{S}{U_{max}} \rceil}{c} \rceil$,
		不妨假设在$\forall i \in [1,t_0]$,最优解的收益是$O_i$,LUF的收益是$G_i$,LUF的Utility总和是$S_i$。
		如果$\sum_{i} O_i \le \frac{nS}{2}$,则可以至少得到近似比为
		\begin{align*}
			ratio &= \frac{ANS}{OPT} \\
				  &< \frac{\lceil \frac{n*S}{c} \rceil}{1.5 * \lceil \frac{n * \lceil \frac{S}{U_{max}} \rceil}{c} \rceil}	\\
				  &< \frac{2}{3} U_{max}
		\end{align*}
		那么,我们可以假设$\sum_{i} O_i = \rho nS \ge \frac{nS}{2}$,从而易知$\sum_{i} S_i \ge \sum_{i} O_i \ge \rho nS$。
		因此,问题转化为求解$\sum_{i} G_i$和$\sum_{i} S_i$之间的关系,从而可以求得$ANS$的界。
		我们不妨先假定$\sum_{i} G_i = \delta nS$,先给此时$ANS$和$OPT$的表达式
		\begin{align*}
			ANS &\le t_0 + \lceil \frac{nS - \delta nS}{c} \rceil \\
				&\le t_0 + \lceil \frac{(1-\delta)nS}{c} \rceil \\
				&\le t_0 + 1 + \frac{(1-\delta)nS}{c} \\
			OPT &\ge t_0 + \lceil \frac{(1-\rho)nS}{cU_{max}} \rceil \\
				&\ge t_0 + \frac{(1-\rho)nS}{cU_{max}} \\
				&\ge (2-\rho) \frac{nS}{cU_{max}} \\
			ratio &= \frac{ANS}{OPT} \\
				  &< \frac{2 + \frac{nS}{cU_{max}} + \frac{(1-\delta)nS}{c}}{(2-\rho) \frac{nS}{cU_{max}}} \\
				  &< 1.5 + \frac{1-\delta}{2-\rho} U_{max}
		\end{align*}
		这里$\rho \in [\frac{1}{2}, 1], \delta \in [0,1]$。
		
		证明的关键是考虑最差情况系$\sum_{i} G_i$和$\sum_{i} S_i$的关系。
		因为,LUF每次取得的都是最大值,但是可能同时存在多个最大值,这也就导致了贪心的选择并不总是能够取到$O_i$。
		而没取$\lceil \frac{S}{U_{max}} \rceil$次时,至少可以掩盖掉$\lceil \frac{S}{U_{max}} \rceil$近似最大值,
		因此当$S = U_{max} + \Delta, \Delta>0$时,掩盖掉的近似最大值$U_{max} - \epsilon$最多。
		即可以掩盖掉$2$个,而$OPT$保证满$S$的项后续一定可以满$S$。
		因此,$\sum_{i} G_i \ge \frac{1}{3} \sum_{i} O_i = \frac{\rho nS}{3}$,从而$\delta = \frac{\rho}{3}$。
		从而有
		\begin{align*}
			ratio &= 1.5 + \frac{1-\frac{\rho}{3}}{2-\rho} U_{max} \\
				  &< \frac{1}{3}(1 + \frac{1}{2-\rho}) U_{max} + O(1) \\
				  &< \frac{2}{3} U_{max} + O(1) \\
				  &< 0.667 U_{max} + O(1) 
		\end{align*}
		
		\newpage
		\item \Emph{Largest Gain First}
		
		$gain = \min \{ u, S_{remain} \}$
		Assign the tasks which has the largest gain.
		同理$\sum_{i} S_i \ge \frac{1}{2} \sum_{i} O_i = \frac{\rho}{2} nS$。
		\begin{align*}
			ratio &= \frac{1-\frac{\rho}{2}}{2-\rho} U_{max} + O(1) \\
				  &< 0.5 U_{max} + O(1)
 		\end{align*}
		
		\item \Emph{Largest Gain2 First}
		
		$ANS \ge \max \{ \frac{1}{c} \sum_{i}{S_i}, \max_{i} S_i \}$。
		所以,分两种情况决策:
		\begin{enumerate}[(1)]
			\item $\frac{1}{c} \sum_{i}{S_i} \ge \max_{i} S_i$ 
			
			选择$gain$最大的前$c$个,相同点看$S_i$选择。
			
			\item $\frac{1}{c} \sum_{i}{S_i} < \max_{i} S_i$
			
			选择$S_i$最大的前$c$个,相同点看$gain$选择。
			
		\end{enumerate}
		
		
	\end{enumerate}
	
\end{enumerate}



\end{document}
