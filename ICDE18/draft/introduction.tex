\section{Introduction}
Crowdsourcing is a new computing paradigm where humans are actively enrolled to participate in the procedure of computing, 
especially for tasks that are intrinsically easier for humans than for computers \cite{TongCS17}.
Increasingly, many crowdsourcing platforms are established for people to release their own tasks, e.g. Amazon Mechanical Turk(AMT)\footnote{https://www.mturk.com/mturk/} and CrowdFlower\footnote{https://www.crowdflower.com/}.
For either monetary or reputation, more and more people are attracted to join the platform and accomplish these tasks.
The people who release the tasks are usually called requesters while those who perform the tasks are usually called crowd worker(worker for short)s.
In recent years along with the tremendous development of mobile Internet and sharing economy, 
a new type of crowdsourcing platform named spatial crowdsourcing(SC) platform has risen rapdily.
Different from traditional website crowdsourcing platform, SC platforms focuses on real-time tasks.
e.g. the real-time traffic updates(Waze) and the real-time taxi-calling service(Uber).

Since task assignment(a.k.a task allocation) has always been one of the most important issues in such SC platforms,
many existing studies focus on how to assign the real-time tasks to right workers for different purposes, 
such as maximizing the overall utility, minimizing the total cost, etc.
This typical problem is usually modeled as a static/online maximum weighted/unweighted bipartite graph matching problem.
Workers are usually regared as one side of the vertexes and then tasks are viewed as the other side.
As long as a task can be assigned to a worker unter some constraints, there will be an edge between their representative vertexes.
Then they assume that the worker should accept the tasks and the task should be perfectly finshed by the worker.

However, these existing studies take the assumption that a worker is enough for finishing the tasks or 
not all tasks need to be peformed. Some of them may assume that there always be a group of workers to perform the tasks.
All of these are infeasible in the applications of information acquisition on SC platform.
Imagine the following scenario. 
An SC platform plans to confirm the location of a parking lot of the Worker's Gymnasium, which is shown in Figure~\ref{fig:gym}.
The parking lot may be one of the six places, which is presented by colorful marks.
In a few hours, many users from this platform may check in with their locations around the gym, e.g. at an animal hospital, railway station, etc.
As you see, their locations are really closed to some of these areas, so they perhaps just pass by the candidate locations.
Since they probably know the answers to these tasks, the platform can ask them for help confirming whether there is a parking lot nearby via the mobiles.
In this way, their answers can be collected in less than a few seconds.
Based on the answers of enough users, the true location of this parking lot can be inferred with high accuracy.
What is more important, the platform always desires to know the answers as quickly as possible.

For these latency-oriented applications on spatial crowdsourcing platform, on one hand, tasks are usually prepared 
and expected to reach an requirement of error rate beforehand.
On the other hand, workers appear dynamically with his check-in locations and the allocation for him should be decided immediately.
This procedure doesnot end until all the tasks have reached the requirement of error rate.
Since there will always be workers check in with any locations, all the tasks would reach the requirement in the end.
particularly, we usually want to keep the high quality while using less latency.
Then it is worth conducting the trade-off between quality and latency at the application of information acquisition in SC platform.
To further illustrate this motivation, we go through a toy example as follows.

\begin{table*}[t]
	\centering
	\caption{each work's predicted accuracy for each tasks}
	\label{table:predictAcc}
	\begin{tabular}{|c|c|c|c|c|c|c|c|c|c|c|c|c|c|c|c|c|}
		\hline
		\quad &$w_1$ &$w_2$ &$w_3$ &$w_4$ &$w_5$ &$w_6$ &$w_7$ &$w_8$ &$w_9$ &$w_{10}$ &$w_{11}$ &$w_{12}$ &$w_{13}$ &$w_{14}$ &$w_{15}$ &$w_{16}$ \\
		\hline
		$t_1$ &0.971 &0.968 &0.974 &0.974 &0.984 &0.935 &0.957 &0.968 &0.976 &0.963 &0.979 &0.968 &0.944 &0.935 &0.957 &0.911 \\
		\hline
		$t_2$ &0.974 &0.971 &0.971 &0.971 &0.987 &0.955 &0.932 &0.974 &0.971 &0.952 &0.984 &0.971 &0.926 &0.952 &0.938 &0.914 \\
		\hline
		$t_3$ &0.971 &0.966 &0.971 &0.968 &0.932 &0.989 &0.957 &0.914 &0.926 &0.920 &0.926 &0.908 &0.941 &0.866 &0.929 &0.908  \\
		\hline
		$t_4$ &0.957 &0.968 &0.966 &0.971 &0.926 &0.987 &0.926 &0.923 &0.927 &0.896 &0.905 &0.923 &0.935 &0.873 &0.923 &0.938 \\
		\hline
		$t_5$ &0.938 &0.935 &0.923 &0.905 &0.971 &0.966 &0.963 &0.968 &0.923 &0.981 &0.976 &0.971 &0.926 &0.923 &0.932 &0.899 \\
		\hline
		$t_6$ &0.966 &0.926 &0.896 &0.923 &0.976 &0.963 &0.968 &0.960 &0.893 &0.977 &0.984 &0.964 &0.899 &0.905 &0.908 &0.930   \\
		\hline
	\end{tabular}
\end{table*}

\begin{figure}[t]
	\center
	\includegraphics[width=0.3\textwidth]{figure/example1.png}
	\caption{Locations of tasks and workers}
	\vspace{-3ex}
	\label{fig:gym}
\end{figure}

\begin{example} 
\label{exa:introExa}
Suppose we have six tasks $t_1$ - $t_6$, sixteen workers $w_1$ - $w_{16}$.
The workers appear on the platform orderly with their check-in locations.
Based on their current location, the locations of these tasks and their historic accuracy,
we can predict the accuracy for each worker to perform every task is shown in Table~\ref{table:predictAcc}.
As soon as the workers show up on the platform, we can choose no more than 2 tasks and ask for their help.
The answers for each task can be inferred based on their answers and 
we expect the accuracy of all these tasks can be no less than 83\%.
The biggest problem is 
how to choose tasks for each workers wise so that less latency is needed while
the accuracy of each task can still reach 83\%.
In other words, what is the minimum number of workers needed for tasks to fulfill the requirement of accuracy
since workers are chronological.
\end{example}

As discussed above, we propose a latency-oriented informatin acquisition problem in spatial crowdsourcing.
As the example above indicates, how to choose tasks for workers for the trade-off between quality and latency affects the performance of the algorithms.
A few existing studies focus on quality control.
CrowdDQS\cite{CrowdDQS} and QASCA\cite{QASCA} propose the methods to choose tasks for workers for high quality.
Even though quality is also important in our situation, latency is another significant consideration when assigning to workers
and their workers donot take this into account.
Some studies focus on latency control.
Finish them\cite{gao2014finish} and tDP\cite{verroios2015tdp} reduce the latency under the constraint of budgets.
Since the latency in our study is only related to quality, their methods cannot be transferred to our situation.
CLAMShell\cite{haas2015clamshell} propose several methods on low-latency data labeling.
However, their work is based on crowdsourcing platform where workers can continually perform tasks while
the workers on SC platform usually cannot do that.
So none of these methods are practical in our situation.
So we summarize our contributions of this paper as follows.

\begin{itemize}
\item We formulate the Latency-oriented Information Acquisition problem in online scenario, 
called the OnlineLIA problem, which aims to control the global latency while performing tasks with good quality.

\item We prove that LIA is NP-hard and design a MinimumCostFlow-Based algorithm to solve LIA problem in offline scenario with approximation ratio $\frac{0.625}{\alpha}$,
where $\alpha$ is the precision of the tolerate error rate of the task. We also design two online algorithms, Largest Acc First and Average And Maximum, with provable competitive ratio.

\item We study the effectiveness and efficiency of the proposed algorithms extensively on both synthetic and real datasets. 
\end{itemize}

The rest of this paper is organized as follows. 
In Section~\ref{sec:definition}, we formally formulate the LIA problem and OnlineLIA problem.
Section~\ref{sec:offline} presents an offline approximation algorithm and
Section~\ref{sec:online} presents two online algorithms with competitive ratio analysis.
Extensive experiments on both synthetic and real datasets are presented in Section~\ref{sec:experiment}. 
We review previous works in Section~\ref{sec:related}. 
We finally conclude this paper in Section~\ref{sec:conclusion}.
