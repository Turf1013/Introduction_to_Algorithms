\section{Offline Algorithm}
\label{sec:offline}
Since the offline LTC problem is NP-hard, we propose an approximate algorithm based on minimum cost flow.
Compared with the optimal latency, our algorithm has a approximation ratio of $\frac{0.625}{\alpha}$.

\fakeparagraph{Basic Idea}
Our algorithm utilizes the lower and upper bounds of the maximum latency.
%The basic idea of this algorithm is to make an arrangement closed to the optimal solution with a certain number of workers.
%Therefore, if the global optimal arrangement have large increasement of $Acc^*$ during this period,
%our method will also have large increasement though it may be less than the optimum.
%However, if the optimal arrangement can only increase a little amount,
%which suggests it may finish all the tasks more slowly,
%the approximation between our method and optimal may become smaller accordingly.
%Considering all these two cases, the approximation ratio is bounded by $\frac{0.625}{\alpha}$,
%which will be proved later.
%Since the key point of this method is how to evaluate the maximum latency in both worst case and best case,
%it is better to introduce Theorem~\ref{thm:boundOfLTC} before explaining the method.

\begin{theorem}[Bounds of Maximum Latency of All Tasks]
\label{thm:boundOfLTC}
Assume $|T| \ge K$ and $\delta \ge 3$, the maximum latency of all tasks in the offline LTC problem has a lower bound of $\frac{|T|\delta}{K}$ and an upper bound of $\frac{|T|\delta}{K\alpha} + \frac{|T|}{K} + 1$.
\end{theorem}

\begin{proof}
According to the McNaughton's Rule~\cite{NaughtonRule}, it is easy to make an optimal arrangement when every worker has the same accuracy for each tasks, \ie $Acc^{*}(w, t) = r, \forall w \in W, t \in T$.
Under this assumption, the maximum latency of all tasks is $\max\{\lceil \frac{|T| \cdot \lceil \frac{\delta}{r} \rceil}{K} \rceil,\ \lceil \frac{\delta}{r} \rceil\}$ and the optimal arrangement can be found in polynomial time.
We can get the lower and upper bounds of the maximum latency of all tasks by replacing $r$ with $1$ and $\alpha$, respectively.
\begin{align*}
	bound_{lower} &\ge \max\{\lceil \frac{|T| \cdot \lceil \frac{\delta}{1} \rceil}{K} \rceil,\ \delta\} \\
					&\ge \max\{\lceil \frac{|T|\delta}{K} \rceil,\ \delta \} \\
					&\ge \frac{|T|\delta}{K}  \\
	bound_{upper} &\le \max\{\lceil \frac{|T| \cdot \lceil \frac{\delta}{\alpha} \rceil}{K} \rceil,\ \lceil \frac{\delta}{\alpha} \rceil\} \\
					&\le \max\{ \frac{|T|\delta}{K\alpha} + \frac{|T|}{K} + 1,\ \lceil \frac{\delta}{\alpha} \rceil\} \\
					&\le \frac{|T|\delta}{K\alpha} + \frac{|T|}{K} + 1
\end{align*}
\end{proof}

From Theorem~\ref{thm:boundOfLTC}, the lower bound of the maximum latency can be achieved only when each task is assigned to the worker with the highest $Acc^*$, and the minimal latency achievable is $\frac{|T|\delta}{K}$.
The idea of our algorithm is to iteratively select a batch of workers (\eg $\frac{|T|\delta}{K}$ workers), and find a temporary arrangement leveraging solutions to the Minimum Cost Flow (MCF) problem.
The iteration continues among the workers who still have capacity available and the tasks which have not reached the tolerable error rate.
The algorithm finds a feasible arrangement and outputs the maximum latency or equivalently, the maximum index of workers in the arrangement.

\fakeparagraph{Algorithm Details}
We present the detailed algorithm, MCF-LTC, for the offline LTC problem as follows.

Given an instance of the offline LTC problem with workers $W'$ in a batch, we first construct a flow network $G_F = (N_F, E_F)$, where $N_F = T \cup W' \cup \{st, ed\}$.
$st$ is a source node and $ed$ is a sink node.
For each pair $w \in W', t \in T$, there is an edge $e_F(w, t) \in E_F$ from $w$ to $t$ with $e_F(w, t).cost = -Acc^*(w, t)$ and $e_F(w, t).capacity = 1$.
For each $w \in W'$, there is an edge $e_F(st, w) \in E_F$ from $st$ to $w$ with $e_F(st, w).cost = 0$ and $e_F(st, w).capacity = K$.
For each $t \in T$, there is an edge $e_F(t, ed) \in E_F$ from $t$ to $ed$ with $e_F(t, ed).cost = 0$ and $e_F(t, ed).capacity = \lceil \delta-S[t] \rceil$.
Here $S[t]$ indicates the amount of $Acc^*$ that $t$ has already got.
Note that we formulate an MCF problem, and we can get a temporary arrangement $M'$ by solving the MCF problem.
Specifically, we apply the Successive Shortest Path Algorithm (SSPA) to calculate the minimum cost flow.
Similar algorithms can also apply, but SSPA is suitable for large-scale data and many-to-many matching with real-valued arc costs~\cite{yiu2008capacity}.

Note that the capacity of the edge $e_F(t, ed)$ is tightly bounded by $\lceil \delta-S[t] \rceil$.
Therefore, it is possible that some workers can still perform tasks (\ie fewer than $K$ tasks have been assigned to the worker using $M'$).
Hence in the next batches, we allocate tasks to workers who can still perform tasks.
In each batch, the most reliable tasks that the worker has never performed before are assigned to him/her.

\begin{algorithm}[t]
	\SetKwInOut{Input}{input}\SetKwInOut{Output}{output}
	\Input{$T, W, Acc^{*}(.,.), K, \epsilon$}
	\Output{Maximum Index of worker in a feasible arrangement $M$}
	$\delta \leftarrow 2\ln{(1.0/\epsilon)}, Q \leftarrow \emptyset, m \leftarrow \frac{|T|\lceil \delta \rceil}{K}$\;
	$S \leftarrow \{0,\ldots,0\}$ \tcc*{S stores accumulated value for each task}
	\For{$i \leftarrow 2$ \emph{\KwTo} $\lceil \frac{|W|}{m} \rceil$ and not all $T$ have reached $\delta$}{
		$W' \leftarrow $ next $\lfloor m \rfloor$($\lfloor 2m \rfloor$ if $i=2$) workers from $W$\;
		construct $G_F = (N_F, E_F)$ according to $(W', T, S)$\;
		MinCostFlow$(G_F)$ and construct $M'$ accordingly\;
		update the $S$ according to $M'$\;
		\ForEach{$w \in W'$}{
			\ForEach{$i \leftarrow 1$ \emph{\KwTo} $|T|$}{
				\If{$T[i]$ has not reached $\delta$ and $w$ does not perform $T[i]$ according to $M'$}{
					push $Acc^{*}(w, T[i])$ with index $i$ into $Q$\;
					maintain size of $Q$ under capacity of $w$\;
				}
			}
			\While{$Q \neq \emptyset$}{
				extract an element with index $i$ from $Q$\;
				$S[i] \leftarrow S[i] + Acc^{*}(w, T[i]), M' \leftarrow M' \cup (w, T[i])$\;
			}
		}
		$M \leftarrow M \cup M'$\;
	}
	\KwRet{Maximum Index of worker in $M$}
\caption{MCF-LTC}
\label{alg:MCF}
\end{algorithm}

Algorithm~\ref{alg:MCF} shows the procedure of our MCT-LTC algorithm.
In lines 1-2, we initialize the parameters.
Specifically, $\delta$ is initialized based on the Hoeffding's inequality~\cite{Ho2013}.
It represents the minimal accumulated $Acc^*$ of each task.
$Q$ is a heap to maintain the largest $Acc^*$ for each worker and is an empty at the beginning.
$m$ is set as the lower bound of the maximum latency in Theorem~\ref{thm:boundOfLTC}, which means $m$ workers are always enough to finish all the remaining tasks if the accuracy of each worker is 100\%.
We use $S$ to store the accumulated $Acc^*$ for each task.
In lines 3-7, we first construct a flow network $G_F$ and calculate the minimum cost flow on $G_F$ among the workers and tasks in the current batch and then obtain a temporary matching $M$.
In lines 8-17, we greedily assign more tasks to the workers who can still perform tasks and update $M'$ accordingly.
Specifically, for each worker, we maintain the most reliable tasks using a heap $Q$ in lines 9-12.
Then in lines 13-15, the accuracy of tasks increases accordingly and the number of remaining tasks $n$ is reduced when any task reaches the tolerable error rate.
Finally the assignment is updated in line 16 and the maximum index of workers in $M$ is output as the latency to compete all the tasks.

\begin{table*}[t]
	\centering
	\caption{Assignment for each worker using different methods.}
	\label{table:solutions}
	\begin{tabular}{|c|c|c|c|c|c|c|c|c|c|c|c|c|c|c|c|c|}
		\hline
		\quad &$w_1$ &$w_2$ &$w_3$ &$w_4$ &$w_5$ &$w_6$ &$w_7$ &$w_8$ &$w_9$ &$w_{10}$ &$w_{11}$ &$w_{12}$ &$w_{13}$ &$w_{14}$ &$w_{15}$ &$w_{16}$ \\
		\hline
		$RR$ &$t_1,t_2$ &$t_3,t_4$ &$t_5,t_6$ &$t_1,t_2$ &$t_3,t_4$ &$t_5,t_6$ &$t_1,t_2$ &$t_3,t_4$ &$t_5,t_6$ &$t_1,t_2$ &$t_3,t_4$ &$t_5,t_6$ &$t_1,t_2$ &$t_3,t_4$ &$t_5,t_6$ &$t_6$ \\
		%\hline
		%$OPT$ &$t_1,t_2$ &$t_1,t_2$ &$t_1,t_2$ &$t_1,t_2$ &$t_1,t_2$ &$t_1,t_2$ &$t_1,t_2$ &$t_1,t_2$ &$t_1,t_2$ &$t_1,t_2$ &$t_1,t_2$ &$t_1,t_2$ &$t_1,t_2$ &$t_1,t_2$ &$t_1,t_2$ &$t_1,t_2$ \\
		\hline
		$MCF$ &$t_1,t_3$ &$t_3,t_4$ &$t_3,t_4$ &$t_1,t_4$ &$t_1,t_2$ &$t_3,t_4$ &$t_5,t_6$ &$t_2,t_5$ &$t_1,t_2$ &$t_5,t_6$ &$t_2,t_6$ &$t_5,t_6$ &$$ &$$ &$$ &$$ \\
		\hline
		$LAF$ &$t_2,t_1$ &$t_2,t_1$ &$t_1,t_2$ &$t_1,t_2$ &$t_6,t_5$ &$t_3,t_4$ &$t_6,t_5$ &$t_5,t_6$ &$t_4,t_3$ &$t_5,t_6$ &$t_6,t_3$ &$t_4,t_3$ &$t_3,t_4$ &$t_4$ &$$ &$$ \\
		\hline
		$AAM$ &$t_2,t_1$ &$t_2,t_1$ &$t_1,t_2$ &$t_4,t_1$ &$t_6,t_5$ &$t_3,t_4$ &$t_6,t_5$ &$t_5,t_2$ &$t_4,t_3$ &$t_5,t_6$ &$t_3,t_4$ &$t_6,t_3$ &$t_3,t_4$ &$$ &$$ &$$ \\
		\hline
	\end{tabular}
\end{table*}

\begin{figure}[t]
	\center
	\includegraphics[width=0.3\textwidth]{figure/flowExample.png}
	\caption{Flow graph constructed}
	\vspace{-3ex}
	\label{fig:flowConstruct}
\end{figure}

\begin{example}
Back to our running example in Example~\ref{exa:introExa}.
\figref{fig:flowConstruct} shows the flow network $G_F$.
The capacity between $st$ and $w$ is $K = 2$ and the capacity between $t$ and $ed$ is $\lceil\delta\rceil=\lceil2\ln_{1/(1-0.83)}\rceil=4$.
Since the predicted accuracy for $w_1$ to perform $t_1$ is $0.971$  (see \tabref{table:predictAcc}), the cost between $w_1$ and $t_1$ is $-Acc^*(w_1, t_1) = -(2Acc(w_1,t_1)-1)^2 = -(2\cdot0.971-1)^2 \approx 0.887$.
The flow graph is constructed after all the edges between $w$ and $t$ are added in the similar way.
After running SSPA on this cost flow network, an arrangement $M'$ is formed using the edges between $w$ and $t$ with non-zero flow.
And we obtain the allocation for each worker (see the first row in \tabref{table:solutions}).
For example, $w_1$ should be assigned to $t_1$ and $t_3$.
Based on this arrangement, only 12 workers are needed and we achieve a maximum latency of 12.
\end{example}

\fakeparagraph{Approximation Ratio}
Next we study the approximation ratio (in terms of the latency to complete all tasks) of MCF-LTC.

\begin{lemma}
\label{lem:ratioNotFinish}
If the latency of the optimal arrangement is smaller than $|T|\delta$ using the arrangement from the first $\lfloor 2\frac{|T|\lceil \delta \rceil}{K} \rfloor$ workers,
then the approximation ratio is $\frac{1}{2\alpha}$.
\end{lemma}
\begin{proof}
Based on the above assumption, the optimal arrangement needs at least another worker to complete all the tasks.
Therefore, at least $\lfloor 2\frac{|T|\lceil \delta \rceil}{K} \rfloor + 1 \ge 2\frac{|T|\delta}{K}$ workers are needed.
According to Theorem~\ref{thm:boundOfLTC}, our algorithm may need $\frac{|T|\delta}{K\alpha} + \frac{|T|}{K} + 1$ workers in the worst case.
Note that the number of workers required is proportional to the maximum latency to complete all tasks.
Thus, the approximation ratio is
\begin{equation*}
	\frac{\frac{|T|\delta}{K\alpha} + \frac{|T|}{K} + 1}{2\frac{|T|\delta}{K}} = \frac{1}{2\alpha} + \frac{1}{2\delta}(1+\frac{K}{|T|})  = \frac{1}{2\alpha} + O(1)
\end{equation*}
\end{proof}

\begin{lemma}
\label{lem:functionOfMCF}
Let $\delta \ge 3$.
If each task has been performed by $\lceil \delta \rceil$ workers from the first $\lfloor 2\frac{|T|\lceil \delta \rceil}{K} \rfloor$ ones,
Algorithm~\ref{alg:MCF} will get at least $\frac{3}{4}$ of the $Acc^{*}$ increase of the optimal arrangement.
\end{lemma}
\begin{proof}
Since the capacity between $t$ and $ed$ is no more than $\lceil \delta \rceil$, the absolute value of the cost from SSPA is no less than that of the optimal arrangement.
However, this does not mean Algorithm~\ref{alg:MCF} will increase $Acc^{*}$ by the same amount as the optimal arrangement
This is because when $\delta = \lfloor \delta \rfloor + \alpha$, Algorithm~\ref{alg:MCF} might waste $1 - \alpha$ for each task in the worst case
(\eg $\lceil \delta \rceil$ workers with $Acc^{*}(,) = 1$ perform this task).
Therefore, the actual increase in $Acc^*$ of Algorithm~\ref{alg:MCF} is at least $\lfloor \delta \rfloor / \lceil \delta \rceil \ge \frac{3}{4}$ of the absolute value of the minimum cost.
Thus Algorithm~\ref{alg:MCF} will get at least $\frac{3}{4}$ of the $Acc^{*}$ increase of the optimal arrangement.
\end{proof}

\begin{lemma}
\label{lem:portionFinish}
If the latency of optimal arrangement is no more than $\lfloor 2\frac{|T|\lceil \delta \rceil}{K} \rfloor$, then the sum of $Acc^*$ in Algorithm~\ref{alg:MCF} will be at least $\frac{3}{8}|T|\delta$.
\end{lemma}
\begin{proof}
Since each task in the optimal arrangement satisfies the error rate constraints, then each task should have been performed by no fewer than $\lceil \delta \rceil$ workers.
Otherwise the accumulation of $Acc^*$ of at least one task is less than $\delta$.
Meanwhile, since there are $\lfloor 2\frac{|T|\lceil \delta \rceil}{K} \rfloor$ workers, all these workers can perform tasks by at most $2|T|\lceil \delta \rceil$ times in total.
Then the sum of the largest $\lceil \delta \rceil$ increase in $Acc^{*}$ of each task is at least $\frac{|T|\delta}{2}$.
According to Lemma~\ref{lem:functionOfMCF}, Algorithm~\ref{alg:MCF} would increase the accumulated $Acc^*$ to at least $\frac{3|T|\delta}{8}$ after the arrangement among the first $\lfloor 2\frac{|T|\lceil \delta \rceil}{K} \rfloor$ workers.
\end{proof}

\begin{lemma}
\label{lem:ratioFinish}
If the latency of the optimal arrangement is no less than $|T|\delta$ using the arrangement from the first $\lfloor 2\frac{|T|\lceil \delta \rceil }{K}\rfloor$ workers, the approximation ratio is $\frac{5}{8\alpha}$.
\end{lemma}
\begin{proof}
According to Lemma~\ref{lem:portionFinish}, the accumulated $Acc^*$ in Algorithm~\ref{alg:MCF} will reach at least $\frac{3|T|\delta}{8}$ after arrangement among the current workers.
Then according to Theorem~\ref{thm:boundOfLTC}, it needs at most another $\frac{\frac{5}{8}|T|\delta}{K\alpha} + \frac{|T|}{K} + 1$ workers.
Thus, the ratio is
\begin{equation*}
	\frac{\frac{\frac{5}{8}|T|\delta}{K\alpha} + \frac{|T|}{K} + 1}{\frac{|T|\delta}{K}} = \frac{5}{8\alpha} + \frac{1}{\delta}(1 + \frac{K}{|T|}) = \frac{5}{8\alpha} + O(1) \\
\end{equation*}
\end{proof}

\begin{theorem}
\label{thm:ratioOfMCF}
The approximation ratio of Algorithm~\ref{alg:MCF} (MCF-LTC) is $\frac{0.625}{\alpha}$.
\end{theorem}
\begin{proof}
According to Lemma~\ref{lem:ratioNotFinish} and Lemma~\ref{lem:ratioFinish}, the ratio of Algorithm~\ref{alg:MCF} is
$\max\{\frac{1}{2\alpha},\ \frac{5}{8\alpha}\} = \frac{0.625}{\alpha}$.
\end{proof}

\fakeparagraph{Complexity Analysis}
The first batch takes $O(K|T| \cdot (\frac{|T|^2\delta}{K} + (|T|+\frac{|T|\delta}{K})\log{(|T|+\frac{|T|\delta}{K})})) = O(\delta |T|^3)$ time and
the remaining batches take $O(\frac{|T|\delta}{K} \cdot |T|\log{K}) = O(\delta |T|^2\frac{\log{K}}{K})$ time.
The iteration takes $O(\frac{|W|K}{\delta|T|})$ batches at most.
Since $K$ is relatively small compared to $T$, the total time cost is $O(\frac{|W|K}{\delta|T|} \cdot (\delta |T|^3 + \delta |T|^2\frac{\log{K}}{K}) = K|W||T|^2)$.
