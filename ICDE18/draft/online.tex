\section{Online Algorithm}
\label{sec:online}
In this section, we design two algorithms to solve the online LTC problem with competitive ratio guarantees.
In the online scenario, the decision for each new worker should be made immediately on his/her arrival, and we adopt a greedy strategy for online decision making.
We first propose the Largest Acc First (LAF) algorithm, which greedily selects the tasks with the largest $Acc^*$ for each new worker.
To avoid local optimal, we further propose a hybrid greedy algorithm, Average And Max (AAM), which has a competitive ratio of $\frac{0.708}{\alpha}$.

\subsection{Largest Acc First (LAF) Algorithm}
Largest Acc First (LAF) is a greedy algorithm for the online LTC problem.
The idea is to select the tasks with the highest $Acc^*$ for every worker.
Every time a worker shows up, the tasks with the $K$ largest $Acc^{*}$ are assigned to the worker.
LAF stops when all the tasks have reached $\delta$.

\fakeparagraph{Algorithm Details}
When a new worker appears, the LAF algorithm enumerates all the remaining tasks and calculate the $Acc^{*}$ value if the worker is to perform that task.
We use a heap to maintain tasks with the $K$ largest $Acc^*$ for each worker.
Once all the tasks reach the tolerable error rate (\ie task completed), the algorithm stops.

Algorithm~\ref{alg:LAF} illustrates the LAF algorithm.
In lines 3-7, the algorithm goes through all the remaining tasks to find the $K$ tasks with the highest $Acc^{*}$ for the new worker $w$.
These $K$ pairs of $Acc^{*}$ and tasks are put into heap $Q$.
In lines 8-11, we update the accumulate values $S$ for each task and the arrangement $M$ according to the selected pairs in $Q$.
Each new worker is assigned at most $K$ tasks on his/her arrival.
Finally the algorithm outputs the maximum index of workers in the arrangement.

%Then we add a pair of the calculated value and task into the heap.
%If the size of the heap exceeds $K$, we extract the pair with the smallest value from the heap.
%By this way, we get $K$ pairs with larger values and further update the arrangements.
%As long as all the tasks satisfy the error rate constraint, the workers are no longer needed and the procedure stops.



\begin{algorithm}[t]
	\SetKwInOut{Input}{input}\SetKwInOut{Output}{output}
	\Input{$T, W, Acc^{*}(.,.), K, \epsilon$}
	\Output{Maximum Index of worker in a feasible arrangement $M$}
	$\delta \leftarrow 2\ln{(1.0/\epsilon)}, Q \leftarrow \emptyset$\;
	$S \leftarrow \{0,\ldots,0\}$ \tcc*{S stores accumulated value for each task}
	\ForEach{new arrival worker $w$ and not all $T$ have reached $\delta$}{
		\ForEach{$i \leftarrow 1$ \emph{\KwTo} $|T|$}{
			\If{$T[i]$ has not reached $\delta$}{
				push $Acc^{*}(w, T[i])$ with index $i$ into $Q$\;
				maintain size of $Q$ under capacity of $w$\;
			}
		}
		\While{$Q \neq \emptyset$}{
			extract an element with index $i$ from $Q$\;
			$S[i] \leftarrow S[i] + Acc^{*}(w, T[i]), M \leftarrow M \cup (w, T[i])$\;
		}
	}
	\KwRet{Maximum index of worker in $M$}
\caption{Largest Acc First (LAF)}
\label{alg:LAF}
\end{algorithm}

\begin{example}
Assume the same settings in Example~\ref{exa:introExa}.
When $w_1$ shows up, the LAF algorithm calculates $Acc^*(w_1,t) = \{0.887,0.889,0.887,0.835,0.767,0.869\}$.
Since $K=2$, only two tasks with the largest $Acc^{*}$ (\ie $t_2$ and $t_1$) will be kept in the heap.
Finally, $t_2$ and $t_1$ are assigned to $w_1$ and the algorithm continues to select two tasks that have not reached $\delta$ for the next worker, as shown in the third row of \tabref{table:solutions}.
By running LAF, 14 workers are needed, which is 2 more than the result in the offline scenario.
\end{example}

\fakeparagraph{Competitive Ratio}
Next we study the competitive ratio of the LAF algorithm.

\begin{theorem}
\label{thm:ubCR}
Without any assumption, the competitive ratio of any online algorithm to solve the online LTC problem is no less than $\frac{0.5}{\alpha}$.
\end{theorem}
\begin{proof}
Suppose $\delta = 1, |T| > 2, K = 1$.
Further assume the first worker can perform two tasks ($t_1$ and $t_2$) equally accurate with $Acc^{*}(,) = 1$.
Without loss of generosity, $t_1$ is assigned to this worker.
In the adversarial model, the next worker may be good at $t_1$ with $Acc^{*}(,) = 1$ but poor at $t_2$ with $\alpha$.
Obviously, the optimal arrangement only needs $2$ workers for the two tasks ($w_1$ performs $t_2$ and $w_2$ performs $t_1$).
However, no matter which task between $t_1$ and $t_2$ is assigned to $w_1$, the adversarial model will assign a worker who is poor at the remaining task.
Accordingly, the competitive ratio is no less than $\frac{1+\frac{\delta}{\alpha}}{2} = \frac{0.5}{\alpha} + O(1)$.
\end{proof}

Theorem~\ref{thm:ubCR} shows that it is difficult to obtain the optimal arrangement when there are multiple equal values of $Acc^{*}$.
Yet the theorem indicates that we can analyze the competitive ratio of LAF by calculating the proportion of $Acc^{*}$ values that could have been achieved, \ie the loss of $Acc^{*}$ values.

\begin{lemma}
\label{lem:lossOfLAF}
Assume $k = \lfloor \delta \rfloor$.
LAF will get no less than $\frac{k}{2k+1}$ of the total increase of $Acc^{*}$ in the optimal arrangement from the first $\lfloor \frac{|T|\lceil \delta \rceil}{K} \rfloor$ workers.
\end{lemma}
\begin{proof}
Denote the total $Acc^{*}$ increase by Algorithm~\ref{alg:LAF} as $\Delta_{LAF}$ and the total $Acc^{*}$ increase by the optimal arrangement as $\Delta_{OPT}$.
$X = \{x_1, x_2, \cdots, x_{|T|}\}$ and $Y = \{y_1, y_2, \cdots, y_{|T|}\}$, which represent the $Acc^{*}$ increase of every task by Algorithm~\ref{alg:LAF} and the optimal arrangement, respectively.
Suppose to the contrary, $\Delta_{LAF} < \frac{k}{2k+1} \Delta_{OPT}$.
It means $\frac{2k+1}{k} \sum_{i=1}^{|T|} {x_i} < \sum_{i=1}^{|T|} {y_i}$.
If $\lceil \delta \rceil = k$, then every time we pick $k$ largest values (\eg $Acc^{*}(,) = 1$), in the worst case we miss other $k$ largest values as in the proof of Theorem~\ref{thm:ubCR}.
If $\lceil \delta \rceil = k + 1$ (\eg $\delta = k + \alpha$), then every time we pick $k+1$ largest values, only the first $k$ values may be significant.
In this case we need any worker instead of the current highly qualified worker.
So in the worst case, we may miss other $k+1$ largest values after picking $k+1$ values (one of them is insignificant).
\TODO{What do you mean by significant and insignificant here? Large in value? Important?}
\ANSWER{It means large in value, more specifically, large in the value of $Acc^*$}
Accordingly, we may lose at most $\frac{k+1}{2k+1}$ of these large values.
Without loss of generality, assume Algorithm~\ref{alg:LAF} misses the tasks $\{t_1, t_2, \cdots, t_j\}$.
Then it is obvious that
\begin{align*}
	\sum_{i=j+1}^{|T|}{x_i} &\ge \sum_{i=j+1}^{|T|}{y_i} \\
	\sum_{i=1}^{j}{x_i} + \frac{k+1}{k} \sum_{i=j+1}^{|T|}{x_i} &\ge \sum_{i=1}^{j}{y_i}
\end{align*}
After adding these two inequations, we have
\[
	\sum_{i=1}^{j}{x_i} + \frac{2k+1}{k} \sum_{i=j+1}^{|T|}{x_i} \ge \sum_{i=1}^{|T|}{y_i}
\]
We also know that
\[
	\sum_{i=1}^{|T|} {y_i} > \frac{2k+1}{k} \sum_{i=1}^{|T|} {x_i}
\]
This contradicts with the fact that
\[
	\sum_{i=1}^{|T|}{x_i} + \frac{2k+1}{k} \sum_{i=j+1}^{|T|}{x_i} < \frac{2k+1}{k} \sum_{i=1}^{|T|} {x_i}
\]
Hence LAF gets no less than $\frac{k}{2k+1}$ of the total $Acc^{*}$ increase in the optimal arrangement from the first $\lfloor \frac{|T|\lceil \delta \rceil}{K} \rfloor$ workers.
\end{proof}

\begin{theorem}
\label{thm:ratioOfLAF}
Assume $\delta \ge 3$, the competitive ratio of Algorithm~\ref{alg:LAF} (LAF) is $\frac{0.730}{\alpha}$.
\end{theorem}

\begin{proof}
Similar to the analysis of Theorem~\ref{thm:ratioOfMCF}, we analyze the ratio from two sides $\Delta_{OPT} \ge \rho |T|\delta$ and $\Delta_{OPT} < \rho |T|\delta$ and infer the proper $\rho$ to calculate the competitive ratio.

Without loss of generality, we assume $\Delta_{OPT} \ge \rho |T|\delta$ from the first $\lfloor \frac{|T|\lceil \delta \rceil}{K} \rfloor$ workers.
According to Theorem~\ref{thm:boundOfLTC}, the optimal latency is more than $\frac{|T|\delta}{K}$.
Also, LAF should have more than $\frac{3}{7} \cdot \rho |T|\delta$ increase in $Acc^*$ based on Lemma~\ref{lem:lossOfLAF}($\lfloor \delta \rfloor \ge 3$).
Thus according to Theorem~\ref{thm:boundOfLTC}, LAF needs less than $\frac{|T|\delta(1 - \frac{3\rho}{7})}{K\alpha} + \frac{|T|}{K} + 1$ workers.
Therefore the competitive ratio is
\begin{align*}
ratio &\le \frac{\frac{|T|\delta(1 - \frac{3\rho}{7})}{K\alpha} + \frac{|T|}{K} + 1}{\frac{|T|\delta}{K}} \\
		&\le (1-\frac{3\rho}{7})\frac{1}{\alpha} + \frac{1}{\delta}(1 + \frac{K}{|T|}) \\
		&\le (1-\frac{3\rho}{7})\frac{1}{\alpha} + O(1)
\end{align*}
Next, according to Theorem~\ref{thm:boundOfLTC}, if $\Delta_{OPT} < \rho |T|\delta$, the optimal arrangement still needs at least $(1-\rho)\frac{|T|\delta}{K} + 1$ workers.
Thus the competitive ratio becomes
\begin{align*}
ratio &\le \frac{\frac{|T|\delta}{K\alpha} + \frac{|T|}{K} + 1}{\lfloor \frac{|T|\lceil \delta \rceil}{K} \rfloor + (1-\rho)\frac{|T|\delta}{K} + 1} \\
		&\le \frac{\frac{|T|\delta}{K\alpha} + \frac{|T|}{K} + 1}{(2-\rho)\frac{|T|\delta}{K}} \\
		&\le \frac{1}{2-\rho}\frac{1}{\alpha} + \frac{1}{(2-\rho)\delta}(1 + \frac{K}{|T|}) \\
		&\le \frac{1}{2-\rho}\frac{1}{\alpha} + O(1)
\end{align*}
Finally, the competitive ratio is $\max\{1-\frac{3\rho}{7},\ \frac{1}{2-\rho}\} \frac{1}{\alpha} + O(1)$.
When $\rho = \frac{13-\sqrt{85}}{6}, ratio \le \frac{0.730}{\alpha}$.
\end{proof}

\fakeparagraph{Complexity Analysis}
The loop of lines 4-7 takes $O(|T|\log{K})$ time and the loop of lines 8-10 takes $O(K\log{K})$ time.
Since the main loop of line 3 is at most $|W|$ times, the total time cost is $O(|W|(|T|\log{K} + K\log{K})) = O(|W||T|\log{K}) = O(|W||T|)$.
\TODO{check whether the line numbers are correct}
\ANSWER{The line numbers are correct}

\subsection{Average And Max (AAM) Algorithm}
The Average and Max (AAM) algorithm is a hybrid greedy algorithm inspired by the McNaughton's Rule~\cite{NaughtonRule}.
The maximum completion time of all different processes is not only determined by the average execution time but also by the longest execution time.
Similarly, the bottleneck to complete all tasks might be some difficult tasks, which need many workers to perform before the tasks can reach the tolerable error rate.
Hence to complete all tasks with minimal numbers of workers, it is important to \textit{(i)} detect and start performing difficult tasks early and \textit{(ii)} finish other tasks with appropriate workers.
AAM first uses the Largest Gain First (LGF) strategy and switches to the Largest Remaining First (LRF) strategy when the ``difficult'' tasks become the bottleneck.


\fakeparagraph{Algorithm Details}
The key step in AAM is to maintain \textit{(i)} the average number of workers needed to finish all tasks (``average'' for short) and \textit{(ii)} the maximum number of workers needed in any remaining task (``maximum'' for short).
AAM switches between the LGF and the LRF strategies based on these two values.
If the ``average'' is larger than the ``maximum'', it indicates that the total number of tasks is the bottleneck.
Then AAM adopts the LGF strategy. 
The LGF strategy chooses the most profitable tasks for the worker.
Here the most profitable task means the largest amount of increase in $Acc^{*}$ without exceeding $\delta$.
\TODO{Here need to explain the difference between LGF and LAF, and the advantages of LGF over LAF}
\ANSWER{LAF selects the higest values of $Acc^*$ for each worker(i.e. the accuracy of the worker is high) without consideration of the current status of the task.
E.g. when the task actually only needs a small value of $Acc^*$ to reach $\delta$, 
we may try to arrange some other tasks for this highly accurate worker.
Therefore, we try to use LGF to fix this drawback. 
For each task, at the begining(i.e the accumulated $Acc^*$ is far from $\delta$), LGF will try to select highly accurate worker to it.
when it becomes close to $\delta$, LGF will try to select a just fit worker to him.
}
If the ``average'' is no larger than the ``maximum'', it indicates that some tasks may need large amounts of workers, and they are the bottleneck to complete all tasks quickly.
In this case AAM adopts the LRF strategy.
The LRF strategy chooses tasks that need more increase in $Acc^{*}$ to reach $\delta$ so that they can be finished quickly.

Algorithm~\ref{alg:AAM} illustrates the procedure.
In lines 4-5, we calculate the average number $avg$ and maximum remaining number $maxRemain$.
In lines 6-12, AAM goes through all tasks to find the best $K$ selections for the new worker $w$ based on LGF or LRF.
\TODO{If line 9 is how the ``gain'' is calculated, you should mention it.}
\ANSWER{line 9 is how the gain is calculated. Gain means the minimum value between ``the increase of value if the task is performed by the worker(i.e., $Acc^*(w, T[i])$)'' and 
``how much left for this task to reach $\delta$(i.e., $\delta-S[i]$)''}
These $K$ pairs are maintained in heap $Q$.
In lines 13-15, the selected tasks are extracted from the heap and AAM updates the arrangement $M$ and other state parameters.
Finally, the algorithm outputs the maximum index of workers in the arrangement.


\begin{algorithm}[t]
	\SetKwInOut{Input}{input}\SetKwInOut{Output}{output}
	\Input{$T, W, Acc^{*}(.,.), K, \epsilon$}
	\Output{Maximum Index of worker in a feasible arrangement $M$}
	$\delta \leftarrow 2\ln{(1.0/\epsilon)}, Q \leftarrow \emptyset$\;
	$S \leftarrow \{0,\ldots,0\}$ \tcc*{S stores accumulated value for each task}
	\ForEach{new arrival worker $w$ and not all $T$ have reached $\delta$}{
		$avg \leftarrow \frac{\sum_{i = 1}^{|T|} {(\delta - S[i])}}{K}$\;
		$maxRemain \leftarrow \max_{i = 1}^{|T|} {(\delta - S[i])}$\;
		\For{$i \leftarrow 1$ \emph{\KwTo} $|T|$} {
			\If{$T[i]$ has not reached $\delta$}{		
				\If{$avg \ge maxRemain$}{
					push $\min\{Acc^{*}(w, T[i]), \delta-S[i]\}$ with index $i$ into $Q$\;
				}
				\Else{
					push $\delta-S[i]$ with an index $i$ into $Q$\;
				}
				maintain size of $Q$ under capacity of $w$\;
			}
		}
		\While{$Q \neq \emptyset$}{
			extract an element with index $i$ from $Q$\;
			$S[i] \leftarrow S[i] + Acc^{*}(w, T[i]), M \leftarrow M \cup (w, T[i])$\;
		}
	}
	\KwRet{Maximum index of worker in $M$}
\caption{Average And Max (AAM) Algorithm}
\label{alg:AAM}
\end{algorithm}

\begin{example}
Back to the settings in Example~\ref{exa:introExa}.
For the first three workers, the process is the same as in LAF (see the forth row of \tabref{table:solutions}).
Now $S[1] = (2\cdot0.971-1)^2+(2\cdot0.968-1)^2+(2\cdot0.974-1)^2 \approx 2.662$ and $S = \{2.662, 2.673, 0, 0, 0, 0\}$. 
Thus $avg = \frac{6\cdot\delta-2.662-2.673}{2} \approx 7.964$ and $maxRemain = \delta \approx 3.544$.
When the forth worker $w_4$ appears, AAM still applies the LGF strategy because $avg > maxRamain$.
% The gain of $t_1$ can be calculated as $\min\{(2\cdot0.974-1)^2, \delta-2.662\} \approx 0.882$.
The gain of $t_1$ can be calculated as $\min\{Acc^*(w_4, t_1), \delta-S[1]\}=\min\{(2\cdot0.974-1)^2, \delta-2.662\} \approx 0.882$.
\TODO{``Gain'' is used without definition. Is it line 9 in Algorithm~\ref{alg:AAM}?}
\ANSWER{Yes, it is calculate as line 9 in Algorithm~\ref{alg:AAM}.}
We can calculate the gains of $t_1$ to $t_6$ in a similar way, which are $\{0.882, 0.870, 0.876, 0.887, 0.656, 0.716\}$, respectively.
Based on the LGF strategy, $t_1$ and $t_4$ are assigned to $w_4$, and $t_5$ and $t_2$ are assigned to $w_8$.
Note that these assignments are different from those by adopting the LAF algorithm.
Now when $w_{11}$ comes, $S = \{3.561, 3.572, 1.682, 2.565, 3.546, 2.693\}$ and $avg \approx 1.846, maxRemain = \max_{i=1}^{6}\{3.544 - S[i]\} \approx 1.862$.
Because $avg < maxRemain$, AAM switches to the LRF strategy.
Hence $t_3$ and $t_4$ are assigned to $w_{11}$.
Finally, the output of AAM is 13, which needs one fewer worker than the LAF algorithm. 
If AAM does not change to the LRF strategy when $w_{11}$ comes, the result will still be 14 workers.
\end{example}
	
\fakeparagraph{Competitive Ratio}
Next, we study the competitive ratio of Algorithm~\ref{alg:AAM} (AAM).

\begin{lemma}	
\label{lem:lgfFirst}
Algorithm~\ref{alg:AAM} will always adopt the LGF strategy for the first $\frac{(|T|-K)\delta}{K}$ workers.
\end{lemma}
\begin{proof}
Suppose to the contrary, the strategy changes to the LRF for the $i$-th worker, where $i < \frac{(|T|-K)\delta}{K}$.
This indicates $avg < maxRemain$.
Since only $i - 1$ workers have come, the minimum average value can be
\begin{align*}
avg &\ge \frac{|T|\delta - (i-1) \cdot K \cdot 1}{K} \\
	&> \frac{|T|\delta + K - \frac{(|T|-K)\delta}{K} \cdot K}{K}  \\
	&> \frac{K(\delta+1)}{K} = \delta + 1
\end{align*}
However, $maxRemain$ must be no more than $\delta$, which contradicts with the fact that $avg$ should be less than $maxRemain$.
It follows that AAM applies the LGF strategy for the first $\frac{(|T|-K)\delta}{K}$ workers.
\end{proof}

\begin{lemma}	
\label{lem:lossOfAAM}
Assume $k = \lfloor \delta \rfloor$.
Then AAM will get no less than $\frac{1}{2}$ of the total $Acc^{*}$ increase in the optimal arrangement from the first $\lfloor \frac{|T|\lceil \delta \rceil}{K}\rfloor$ workers.
\end{lemma}
\begin{proof}
Denote the total $Acc^{*}$ increase using LGF as $\Delta_{LGF}$ and the total $Acc^{*}$ increase using AAM as $\Delta_{AAM}$.
$X = \{x_1, x_2, \cdots, x_{|T|}\}$ and $Y = \{y_1, y_2, \cdots, y_{|T|}\}$, which represent the $Acc^{*}$ increase of every task using AAM and the optimal arrangement, respectively.
Suppose to the contrary, $\Delta_{LGF} < \frac{1}{2} \Delta_{OPT}$.
It means $2\sum_{i=1}^{|T|} {x_i} < \sum_{i=1}^{|T|} {y_i}$.
Different from Lemma~\ref{lem:lossOfLAF}, each time the greedy solution \TODO{be specific, greedy refers to LGF or LRF or both?}\ANSWER{LGF} picks $k$ tasks to workers, while the optimal arrangement may pick the other $k$ tasks with the same gain.
Since these values indicates the real amount of increase, we would not choose the large value which turns out to be insignificant.
\TODO{What do ``these values'' refer to? Again, what does insignificant mean here?}
\ANSWER{``these values'' refers to the sum of gain from these $k$ tasks. Here ``insignificant'' means the gain is little.
I want to express that the value is large but only a small portion in it helps tasks to reach $\delta$}
Accordingly, we may lose at most $\frac{k}{2k}=\frac{1}{2}$ of these large values.
Without loss of generality, assume $\{t_1, t_2, \cdots, t_j\}$ are the tasks we miss.
Then we have
\begin{align*}
	\sum_{i=j+1}^{|T|}{x_i} &\ge \sum_{i=j+1}^{|T|}{y_i} \\
	\sum_{i=1}^{j}{x_i} + \sum_{i=j+1}^{|T|}{x_i} &\ge \sum_{i=1}^{j}{y_i}
\end{align*}
After adding these two inequations, we have
\[
	\sum_{i=1}^{j}{x_i} + 2\sum_{i=j+1}^{|T|}{x_i} \ge \sum_{i=1}^{|T|}{y_i}.
\]
Also
\[
	\sum_{i=1}^{|T|} {y_i} > 2\sum_{i=1}^{|T|} {x_i}.
\]
This contradicts with the fact that
\[
	\sum_{i=1}^{|T|}{x_i} + 2\sum_{i=j+1}^{|T|}{x_i} < 2\sum_{i=1}^{|T|} {x_i}
\]
If follows that $\Delta_{LGF} \ge \frac{1}{2} \Delta_{OPT}$
According to Lemma~\ref{lem:lgfFirst}, $\Delta_{LGF} = \Delta_{AAM}$ for the first $\lfloor \frac{|T|\lceil \delta \rceil}{K}\rfloor$ workers.
Thus $\Delta_{AAM} \ge \frac{\Delta_{OPT}}{2}$ for the first $\lfloor \frac{|T|\lceil \delta \rceil}{K}\rfloor$ workers.
\end{proof}
	
\begin{theorem}
Assume $\delta \ge 3$, the competitive ratio of Algorithm~\ref{alg:AAM} (AAM) is $\frac{0.708}{\alpha}$.
\end{theorem}
\begin{proof}
Similar to Theorem~\ref{thm:ratioOfLAF}, when the total $Acc^{*}$ increase of the optimal arrangement is more than $\rho n\delta$, AAM will need less than
$\frac{(1-\frac{\rho}{2})|T|\delta}{K\alpha} + \frac{|T|}{K} + 1$ workers according to Theorem~\ref{thm:boundOfLTC} and Lemma~\ref{lem:lossOfAAM}.
In this condition, the competitive ratio is $(1-\frac{\rho}{2})\frac{1}{\alpha} + O(1)$.
Otherwise the optimal arrangement would need at least $(1-\rho)\frac{|T|\delta}{K\alpha} + 1$ workers.
Thus, the competitive ratio is $\frac{1}{2-\rho} \frac{1}{\alpha} + O(1)$.
Finally, when $\rho = 2 - \sqrt{2}$, the competitive ratio is $(1 - \frac{2 - \sqrt{2}}{2}) \frac{1}{\alpha} \le \frac{0.708}{\alpha}$.
\end{proof}	
	
\fakeparagraph{Complexity Analysis}
The calculation of the average value and maximum value in lines 4-5 takes $O(|T|)$ time.
The loop of lines 6-12 takes $O(|T|\log{K})$ time and the loop of lines 13-15 takes $O(K\log{K})$ time.
Since the main loop of line 3 is at most $|W|$ times, the total time cost is $O(|W|(|T| + |T|\log{K} + K\log{K})) = O(|W||T|\log{K})) = O(|W||T|$.
\TODO{check line numbers}
\ANSWER{line numbers are correct}