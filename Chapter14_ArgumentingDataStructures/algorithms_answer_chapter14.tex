%-*- coding:UTF-8 -*-
% 算法导论-第14章 数据结构的扩张.tex
\documentclass[UTF8]{ctexart}
\usepackage{geometry}
\usepackage{enumerate}
\usepackage{amsmath}
\usepackage{listings} %插入代码
\usepackage{xcolor} %代码高亮
\usepackage{diagbox}
\usepackage{tabularx}
\usepackage{graphicx}
\usepackage{caption}
\usepackage{subcaption}
\usepackage{float}

% Some setup
\pagestyle{plain}
\geometry{a4paper, top=2cm, bottom=2cm, left=2cm, right=2cm}
\CTEXsetup[format={\raggedright\bfseries\Large}]{section}
\lstset{numbers=left, %设置行号位置
        numberstyle=\small, %设置行号大小
        keywordstyle=\color{blue}, %设置关键字颜色
        commentstyle=\color{purple}, %设置注释颜色
        %frame=single, %设置边框格式
        escapeinside=``, %逃逸字符(1左面的键),用于显示中文
        breaklines, %自动折行
        extendedchars=false, %解决代码跨页时,章节标题,页眉等汉字不显示的问题
        %xleftmargin=2em,xrightmargin=2em, aboveskip=1em, %设置边距
        tabsize=4, %设置tab空格数
        showspaces=false %不显示空格
       }

% About math
\newcommand{\rmnum}[1]{\romannumeral #1}

\begin{document}

\title{\Huge算法导论习题\\}
\vspace{2cm}
\author{\Large Trasier}
\date{}
\maketitle

%\newpage
\section*{第14章\quad数据结构的扩张}
\begin{enumerate}
    \item 14.1-5 \\
    解:\\
        先使用Search函数找到关键字为x的结点p,再从p.right的子树中找到第i个数。\\
		$T(n) = T_{search}(x) + T_{select}(i) = O(lgn) + O(lgn) = O(lgn)$.
		
	
	\item 14.1-6 \\
	解:\\
		注意在新的秩的定义下, $x.size = x.left.size + 1$. 因此,相关函数修改为
	\begin{lstlisting}[language=C++]
	void RBTree_Insert(Tree_t *t, Node_t *z) {
		......
		while (x != t->NIL) {
			y = x;
			if (z->key < x->key) {
				++x->size;
				x = x->left;
			} else {
				x = x->right;
			}
		}
		......
	}

	void Left_Rotate(Tree_t *t, Node_t *x) {
		......
		y->size += x->left->size;
	}

	void Right_Rotate(Tree_t *t, Node_t *y) {
		......
		y->size -= x->left->size;
	}

	void RBTree_Delete(Tree_t *t, Node_t *z) {
		......
		while (q != t->NIL) {
			if (q == q->p->left)
				--q->size;
			q = q->p;
		}
	}
	\end{lstlisting}
	
	\item 14.1-7 \\
	解:\\
		这道题英文参考答案说的很清楚。\\
		假定数组元素为A[1...n],
		Inv(A[j])(元素A[j]的逆序数)=$\#\{i|i<j, A[i]>A[j]\}$. \\
		因此,$j = rank(A[j]) + Inv(A[j])$.
		故$Inv(A[j]) = j - rank(A[j])$. 从而使用如下算法计算逆序对:
		\begin{enumerate}[(1)]
			\item 将元素A[1...n]插入到OS-Tree中;
			\item 对1...n的n个元素,使用Search与OS-Rank求得元素A[j]的rank值,从而求得$Inv[j] = j - rank_j$.
		\end{enumerate}
		时间复杂度$T(n) = n(T_{search} + T_{rank}) = O(nlgn)$.
	
	\item 14.2-1 \\
	解:\\
		其实就和线索树一样,增加4个指针并进行维护。
		
	\item 14.2-2 \\
	解:\\
		可以,算法导论参考答案说的很清楚了。
		
	\item 14.3-1 \\
	解:
	\begin{lstlisting}[language=C++]
	void Left_Rotate(Tree_t *t, Node_t *x) {
		Node_t *y = x->right;
		x->right = y->left;
		if (y->left != t->NIL)
			y->left->p = x;
		y->p = x->p;
		if (x->p == t->NIL) {
			// x is root
			t->root = y;	
		} else if (x == x->p->left) {
			x->p->left = y;
		} else {
			x->p->right = y;
		}
		y->left = x;
		x->p = y;
		// maintenance attr mmax
		x->mmax = max(
			max(x->left->mmax, x->right->mmax), x->intr.high
		);
		y->mmax = max(
			max(y->left->mmax, y->right->mmax), y->intr.high
		);
	}
	\end{lstlisting}
	
	\item 14.3-2 \\
	解:
	\begin{lstlisting}[language=c++]
	bool overlap(interval_t a, interval_t b) {
		return b.low<a.high && a.low<b.high;	// <= -> <
	}
	
	Node_t *RBTree_Search(Tree_t *t, interval_t intr) {
		Node_t *x = t->root;
		while (x!=t->NIL && !overlap(intr, x->intr)) {
			if (x->left!=t->NIL && x->left->mmax>intr.low) {	// >= -> >
				x = x->left;
			} else {
				x = x->right;
			}
		}
		return x;
	}
	\end{lstlisting}
	
	\item 14.3-3 \\
	解:
	\begin{lstlisting}[language=c++]
	Node_t *RBTree_Search_Min(Tree_t *t, Node_t *z, interval_t intr) {
		Node_t *x;	
		
		if (z == t->NIL)
			return t->NIL;
		
		if (z->left!=t->NIL && z->left->mmax>=intr.low) {
			x = RBTree_Search_Min(t, z->left, intr);
			if (x != t->NIL)
				return x;
		}
		
		if (overlap(intr, z->intr))
			return z;
		
		if (z->right != t->NIL)
			return RBTree_Search_Min(t, z->right, intr);
		
		return t->NIL;
	}
	\end{lstlisting}
	
	\item 14.3-4 \\
	解:
	\begin{lstlisting}[language=C++]
	void RBTree_Search_All(Tree_t *t, Node_t *z, interval_t intr) {
		Node_t *x;
		
		if (z == t->NIL)
			return ;
		
		if (z->left!=t->NIL && z->left->mmax>=intr.low) {
			RBTree_Search_All(t, z->left, intr);
		}
		
		if (overlap(intr, z->intr)) {
			printf("[%d, %d]\n", z->intr.low, z->intr.high);
		}
		
		if (z->right!=t->NIL && z->right->mmax>=intr.low) {
			RBTree_Search_All(t, z->right, intr);
		}
	}
	\end{lstlisting}
	
	\item 14.3-5 \\
	解:
	\begin{lstlisting}[language=C++]
	Node_t *RBTree_Seach_Exactly(Tree_t *t, interval_t intr) {
		Node_t *x = t->root;
		
		while (x != t->NIL) {
			if (overlap(intr, x->intr)) {
				if (intr == x->intr)
					break;
				if (x->left!=t->NIL && intr.low<x->key)
					x = x->left;
				else
					x = x->right;
			} else {
				if (x->left!=t->NIL && x->left->mmax>=intr.low)
					x = x->left;
				else
					x = x->right;
			}
		}
		return x;
	}
	\end{lstlisting}
	
\end{enumerate}


\end{document}

