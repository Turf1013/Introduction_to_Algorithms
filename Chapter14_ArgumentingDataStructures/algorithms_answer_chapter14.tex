%-*- coding:UTF-8 -*-
% 算法导论-第14章 数据结构的扩张.tex
\documentclass[UTF8]{ctexart}
\usepackage{geometry}
\usepackage{enumerate}
\usepackage{amsmath}
\usepackage{listings} %插入代码
\usepackage{xcolor} %代码高亮
\usepackage{diagbox}
\usepackage{tabularx}
\usepackage{graphicx}
\usepackage{caption}
\usepackage{subcaption}
\usepackage{float}

% Some setup
\pagestyle{plain}
\geometry{a4paper, top=2cm, bottom=2cm, left=2cm, right=2cm}
\CTEXsetup[format={\raggedright\bfseries\Large}]{section}
\lstset{numbers=left, %设置行号位置
        numberstyle=\small, %设置行号大小
        keywordstyle=\color{blue}, %设置关键字颜色
        commentstyle=\color{purple}, %设置注释颜色
        %frame=single, %设置边框格式
        escapeinside=``, %逃逸字符(1左面的键),用于显示中文
        breaklines, %自动折行
        extendedchars=false, %解决代码跨页时,章节标题,页眉等汉字不显示的问题
        %xleftmargin=2em,xrightmargin=2em, aboveskip=1em, %设置边距
        tabsize=4, %设置tab空格数
        showspaces=false %不显示空格
       }

% About math
\newcommand{\rmnum}[1]{\romannumeral #1}

\begin{document}

\title{\Huge算法导论习题\\}
\vspace{2cm}
\author{\Large曾宇祥\\SY1406122}
\date{}
\maketitle

%\newpage
\section*{第14章\quad数据结构的扩张}
\begin{enumerate}
    \item 14.1-5 \\
    解:\\
        先使用Search函数找到关键字为x的结点p,再从p.right的子树中找到第i个数。\\
		$T(n) = T_{search}(x) + T_{select}(i) = O(lgn) + O(lgn) = O(lgn)$.
		
	
	\item 14.1-6 \\
	解:\\
		注意在新的秩的定义下, $x.size = x.left.size + 1$. 因此,相关函数修改为\\
	\begin{lstlisting}[language=C++]
		void RBTree_Insert(Tree_t *t, Node_t *z) {
			......
			while (x != t->NIL) {
				y = x;
				if (z->key < x->key) {
					++x->size;
					x = x->left;
				} else {
					x = x->right;
				}
			}
			......
		}

		void Left_Rotate(Tree_t *t, Node_t *x) {
			......
			y->size += x->left->size;
		}

		void Right_Rotate(Tree_t *t, Node_t *y) {
			......
			y->size -= x->left->size;
		}

		void RBTree_Delete(Tree_t *t, Node_t *z) {
			......
			while (q != t->NIL) {
				if (q == q->p->left)
					--q->size;
				q = q->p;
			}
		}
	\end{lstlisting}
	
	\item 14.1-7 \\
	解:\\
		这道题英文参考答案说的很清楚。\\
		假定数组元素为A[1...n], 
		Inv(A[j])(元素A[j]的逆序数)=$\#\{i|i<j, A[i]>A[j]\}$. \\
		因此,$j = rank(A[j]) + Inv(A[j])$. 
		故$Inv(A[j]) = j - rank(A[j])$. 从而使用如下算法计算逆序对:
		\begin{enumerate}[(1)]
			\item 将元素A[1...n]插入到OS-Tree中;
			\item 对1...n的n个元素,使用Search与OS-Rank求得元素A[j]的rank值,从而求得$Inv[j] = j - rank_j$.
		\end{enumerate}
		时间复杂度$T(n) = n(T_{search} + T_{rank}) = O(nlgn)$.
		
	
\end{enumerate}


\end{document}

