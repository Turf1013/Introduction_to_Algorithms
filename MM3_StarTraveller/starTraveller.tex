%-*- coding:UTF-8 -*-
% starTraveller.tex
\documentclass[hyperref,UTF8]{ctexart}
\usepackage{geometry}
\usepackage{enumerate}
\usepackage{amsmath}
\usepackage{amssymb}
\usepackage{dsfont}
\usepackage{amsthm}
\usepackage{listings} %插入代码
\usepackage{xcolor} %代码高亮
\usepackage{blkarray}
\usepackage{diagbox}
\usepackage{tabularx}
\usepackage{graphicx}
\usepackage{caption}
\usepackage{subcaption}
\usepackage{float}
\usepackage{color}
\usepackage{multirow}
\usepackage[all,pdf]{xy}
\usepackage{verbatim}   %comment
\usepackage{cases}
\usepackage{clrscode3e}	% need clrscode3e package which is not included in CTex.
\usepackage{hyperref}

\geometry{screen}
\hypersetup{
    colorlinks=true,
    bookmarks=true,
    bookmarksopen=false,
    % pdfpagemode=FullScreen,
    pdfstartview=fit,
    pdftitle={DNA-Align},
    pdfauthor={Trasier}
}

% THEOREM Environments --------------------------------------------------------
\newtheorem{thm}{Theorem}[subsection]
\newtheorem{cor}[thm]{Corollary}
\newtheorem{lem}[thm]{Lemma}
\newtheorem{prop}[thm]{Proposition}
\newtheorem{prob}[thm]{Problem}
\newtheorem{mthm}[thm]{Main Theorem}
\theoremstyle{definition}
\newtheorem{defn}[thm]{Definition}
\theoremstyle{remark}
\newtheorem{rem}[thm]{Remark}
\numberwithin{equation}{subsection}
% MATH ------------------------------------------------------------------------
\DeclareMathOperator{\RE}{Re}
\DeclareMathOperator{\IM}{Im}
\DeclareMathOperator{\ess}{ess}
\newcommand{\eps}{\varepsilon}
%\newcommand{\To}{\longrightarrow}  conflict with \package{clrscode3e}p
\newcommand{\h}{\mathcal{H}}
\newcommand{\s}{\mathcal{S}}
\newcommand{\A}{\mathcal{A}}
\newcommand{\J}{\mathcal{J}}
\newcommand{\M}{\mathcal{M}}
\newcommand{\W}{\mathcal{W}}
\newcommand{\X}{\mathcal{X}}
\newcommand{\BOP}{\mathbf{B}}
\newcommand{\BH}{\mathbf{B}(\mathcal{H})}
\newcommand{\KH}{\mathcal{K}(\mathcal{H})}
\newcommand{\Real}{\mathbb{R}}
\newcommand{\Complex}{\mathbb{C}}
\newcommand{\Field}{\mathbb{F}}
\newcommand{\RPlus}{\Real^{+}}
\newcommand{\Polar}{\mathcal{P}_{\s}}
\newcommand{\Poly}{\mathcal{P}(E)}
\newcommand{\EssD}{\mathcal{D}}
\newcommand{\Lom}{\mathcal{L}}
\newcommand{\States}{\mathcal{T}}
\newcommand{\abs}[1]{\left\vert#1\right\vert}
\newcommand{\set}[1]{\left\{#1\right\}}
\newcommand{\seq}[1]{\left<#1\right>}
\newcommand{\norm}[1]{\left\Vert#1\right\Vert}
\newcommand{\essnorm}[1]{\norm{#1}_{\ess}}


% Some setup
\pagestyle{plain}
\geometry{a4paper, top=2cm, bottom=2cm, left=2cm, right=2cm}
\CTEXsetup[format={\raggedright\bfseries\Large}]{section}
\lstset{numbers=left, %设置行号位置
        numberstyle=\small, %设置行号大小
        keywordstyle=\color{blue}, %设置关键字颜色
        commentstyle=\color{purple}, %设置注释颜色
        %frame=single, %设置边框格式
        escapeinside=``, %逃逸字符(1左面的键),用于显示中文
        breaklines, %自动折行
        extendedchars=false, %解决代码跨页时,章节标题,页眉等汉字不显示的问题
        %xleftmargin=2em,xrightmargin=2em, aboveskip=1em, %设置边距
        tabsize=4, %设置tab空格数
        showspaces=false %不显示空格
       }

% About math
\newcommand{\rmnum}[1]{\romannumeral #1}
\newcommand{\Emph}{\textbf}
\newcolumntype{Y}{>{\centering\arraybackslash}X}
\newcommand{\resetcounter}{\setcounter{equation}{0}}
\newcommand{\equsuf}[1][x]{\equiv_{\textit{Suff(#1)}}}	
\newcommand{\Suff}{\textit{Suff}}
\newcommand{\len}[1][x]{\textit{length}_{#1}}

% section deeep to 3 1.1.1
\setcounter{secnumdepth}{3}

\begin{document}

\title{\Huge StarTraveller}
\vspace{2cm}
\author{\Large Trasier}
\date{\today}
\maketitle

\section{Problem Statement}
\label{sec:problem_statement}
	
	You are given \Emph{NStar} stars in 2D space.
	You have \Emph{NShip} space ships that can travel between stars.
	The amount of energy used by a space ship to travel from one star to another is calculated as the
	\Emph{Euclidean} distance between these stars.
	There are \Emph{NUfo} unidentified flying objects (UFO) moving around in space.
	UFO's reduce the energy required to travel between two stars.
	The energy consumed by your space ship is multiplied by 0.001 when your ship travels in the same direction at the same time as a UFO.
	For example, if you travel between star A(0,0) and B(10,0) it will cost you 10 energy.
	However, if one UFO is flying from A to B at the same time, it will cost you $10 \times 0.001=0.01$ energy.
	If two UFO's are flying from A to B at the same time, it will cost you $10 \times 0.001 \times 0.001=0.00001$ energy.
	Your task is to minimize the total energy used by your ships in order to visit every star at least once.

\section{Functions}
	
	Your code should implement the methods \proc{init}(vector<int> stars) and \proc{makeMoves}(vector<int> ufos, vector<int> ships).
	Your \proc{init} method will be called once and can return any integer.
	Your \proc{makeMoves} method will be called until all stars have been travelled to or when you reached a maximum the NStar*4 turns.
	Which means, the \Emph{Data Generator} will continue to generate data for calling \proc{makeMoves}
	until one of previous two condition fits.
	Your \proc{makeMoves} method should return a \Emph{vector<int>} containing your space ship moves for a single turn.
	\begin{itemize}
		\item \id{stars} gives you the location of each star.
		
		The 2D location of the ith star is given by $(\id{stars}[i*2], \id{stars}[i*2+1])$.
		The range of values will be in $[0, 1023]$.
		
		\item \id{ufos} gives you the location of each ufo and the star indices where it is travelling towards in the next two turns.
		
		The star index of where the ith ufo is located, is given by $\id{ufos}[i*3]$.
		In the next move, the ith ufo will travel to $\id{ufos}[i*3+1]$ and it will travel to $\id{ufos}[i*3+2]$ after that.
		
		\item \id{ships} gives you the location of each space ship.
		
		The star index of where the ith ship is located, is given by $\id{ships}[i]$.
		A space ship can only travel directly between stars, therefor it's location is described by the star index where it is located.
		
	\end{itemize}
	You must return the list of moves for each space ship.
	The ith element of your return should give the zero based star index of the star where space ship i should travel towards.

\section{Scoring}	
\label{sec:scoring}

	For each test case we will calculate your raw and normalized scores.
	If you were not able to produce a valid return value, then your raw score is -1 and the normalized score is 0.
	Otherwise, the raw score is equal to the sum of energy used over all space ship moves.
	The normalized score for each test is $1,000,000.0 * BEST / YOUR$,
	where BEST is the lowest raw score currently obtained on this test case (considering only the last submission from each competitor).
	Finally, your total score is equal to the arithmetic average of normalized scores on all test cases.
	You can see your raw scores on each example test case by making an example submit.
	You can also see total scores of all competitors on provisional test set in the match standings.
	No other information about scores is available during the match.

	
\section{Clarifications}
\label{sec:clarifications}
	
	\begin{itemize}
	
		\item Space ships start at reandomly selected stars, these stars are not marked as visited
		initially and need to travelled during a turn.
		
		\item A space ship may remain stationary at the same star. (The star will be marked as visited).
		
		\item You can visit the same star multiple times.
		
		\item A maximum of $\Emph{Nstar}*4$ turns are allowed, thereafter you will score zero for the test.
		
		\item A space ship can travel from it's current star to any other star.
		
		\item \Emph{NStar} will be in the range of $[100, 200]$ (Except for seed 1).
		
		\item \Emph{NShip} will be in the range of $[1, 10]$.
		
		\item \Emph{NUfo} will be in the range of $[0, \Emph{NStar}/100)$.
		
		\item The range of the integer star coordinates is $[0, 1023]$.
		
		\item Stars are generated around a random number of galaxy centers with a \Emph{Gaussian} distribution.
		See the visualizer source code for exact implementation.
	
	\end{itemize}
	
\section{Notes}	
\label{sec:notes}

	\begin{itemize}
	
		\item The time limit is \Emph{20 seconds} per test case (this includes only the time spent in you code.)
		
		\item The memory limit is \Emph{1024 MB}.
		
		\item The implicit source code size limit is around \Emph{1 MB}.
		Once you code is compiled, the binary size should not exceed \Emph{1 MB}.
		
		\item The compilation time limit is \Emph{30 seconds}.
		
		\item There are \Emph{10} example test cases and \Emph{100} full submission (provisional) test cases.

		\item The match is \Emph{rated}.
		
	\end{itemize}
	
\section{Thought}
\label{sec:thought}
	
\end{document}
