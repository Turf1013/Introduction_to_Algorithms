%-*- coding:UTF-8 -*-
% starTraveller.tex
\documentclass[hyperref,UTF8]{ctexart}
\usepackage{geometry}
\usepackage{enumerate}
\usepackage{amsmath}
\usepackage{amssymb}
\usepackage{dsfont}
\usepackage{amsthm}
\usepackage{listings} %插入代码
\usepackage{xcolor} %代码高亮
\usepackage{blkarray}
\usepackage{diagbox}
\usepackage{tabularx}
\usepackage{graphicx}
\usepackage{caption}
\usepackage{subcaption}
\usepackage{float}
\usepackage{color}
\usepackage{multirow}
\usepackage[all,pdf]{xy}
\usepackage{verbatim}   %comment
\usepackage{cases}
\usepackage{clrscode3e}	% need clrscode3e package which is not included in CTex.
\usepackage{hyperref}

\geometry{screen}
\hypersetup{
    colorlinks=true,
    bookmarks=true,
    bookmarksopen=false,
    % pdfpagemode=FullScreen,
    pdfstartview=fit,
    pdftitle={DNA-Align},
    pdfauthor={Trasier}
}

% THEOREM Environments --------------------------------------------------------
\newtheorem{thm}{Theorem}[subsection]
\newtheorem{cor}[thm]{Corollary}
\newtheorem{lem}[thm]{Lemma}
\newtheorem{prop}[thm]{Proposition}
\newtheorem{prob}[thm]{Problem}
\newtheorem{mthm}[thm]{Main Theorem}
\theoremstyle{definition}
\newtheorem{defn}[thm]{Definition}
\theoremstyle{remark}
\newtheorem{rem}[thm]{Remark}
\numberwithin{equation}{subsection}
% MATH ------------------------------------------------------------------------
\DeclareMathOperator{\RE}{Re}
\DeclareMathOperator{\IM}{Im}
\DeclareMathOperator{\ess}{ess}
\newcommand{\eps}{\varepsilon}
%\newcommand{\To}{\longrightarrow}  conflict with \package{clrscode3e}p
\newcommand{\h}{\mathcal{H}}
\newcommand{\s}{\mathcal{S}}
\newcommand{\A}{\mathcal{A}}
\newcommand{\J}{\mathcal{J}}
\newcommand{\M}{\mathcal{M}}
\newcommand{\W}{\mathcal{W}}
\newcommand{\X}{\mathcal{X}}
\newcommand{\BOP}{\mathbf{B}}
\newcommand{\BH}{\mathbf{B}(\mathcal{H})}
\newcommand{\KH}{\mathcal{K}(\mathcal{H})}
\newcommand{\Real}{\mathbb{R}}
\newcommand{\Complex}{\mathbb{C}}
\newcommand{\Field}{\mathbb{F}}
\newcommand{\RPlus}{\Real^{+}}
\newcommand{\Polar}{\mathcal{P}_{\s}}
\newcommand{\Poly}{\mathcal{P}(E)}
\newcommand{\EssD}{\mathcal{D}}
\newcommand{\Lom}{\mathcal{L}}
\newcommand{\States}{\mathcal{T}}
\newcommand{\abs}[1]{\left\vert#1\right\vert}
\newcommand{\set}[1]{\left\{#1\right\}}
\newcommand{\seq}[1]{\left<#1\right>}
\newcommand{\norm}[1]{\left\Vert#1\right\Vert}
\newcommand{\essnorm}[1]{\norm{#1}_{\ess}}


% Some setup
\pagestyle{plain}
\geometry{a4paper, top=2cm, bottom=2cm, left=2cm, right=2cm}
\CTEXsetup[format={\raggedright\bfseries\Large}]{section}
\lstset{numbers=left, %设置行号位置
        numberstyle=\small, %设置行号大小
        keywordstyle=\color{blue}, %设置关键字颜色
        commentstyle=\color{purple}, %设置注释颜色
        %frame=single, %设置边框格式
        escapeinside=``, %逃逸字符(1左面的键),用于显示中文
        breaklines, %自动折行
        extendedchars=false, %解决代码跨页时,章节标题,页眉等汉字不显示的问题
        %xleftmargin=2em,xrightmargin=2em, aboveskip=1em, %设置边距
        tabsize=4, %设置tab空格数
        showspaces=false %不显示空格
       }

% About math
\newcommand{\rmnum}[1]{\romannumeral #1}
\newcommand{\Emph}{\textbf}
\newcolumntype{Y}{>{\centering\arraybackslash}X}
\newcommand{\resetcounter}{\setcounter{equation}{0}}
\newcommand{\equsuf}[1][x]{\equiv_{\textit{Suff(#1)}}}	
\newcommand{\Suff}{\textit{Suff}}
\newcommand{\len}[1][x]{\textit{length}_{#1}}

% section deeep to 3 1.1.1
\setcounter{secnumdepth}{3}

\begin{document}

\title{\Huge StarTraveller}
\vspace{2cm}
\author{\Large Trasier}
\date{\today}
\maketitle

\section{Problem Statement}
\label{sec:problem_statement}
	
	You are given \Emph{NStar} stars in 2D space.
	You have \Emph{NShip} space ships that can travel between stars.
	The amount of energy used by a space ship to travel from one star to another is calculated as the
	\Emph{Euclidean} distance between these stars.
	There are \Emph{NUfo} unidentified flying objects (UFO) moving around in space.
	UFO's reduce the energy required to travel between two stars.
	The energy consumed by your space ship is multiplied by 0.001 when your ship travels in the same direction at the same time as a UFO.
	For example, if you travel between star A(0,0) and B(10,0) it will cost you 10 energy.
	However, if one UFO is flying from A to B at the same time, it will cost you $10 \times 0.001=0.01$ energy.
	If two UFO's are flying from A to B at the same time, it will cost you $10 \times 0.001 \times 0.001=0.00001$ energy.
	Your task is to minimize the total energy used by your ships in order to visit every star at least once.

\section{Functions}
	
	Your code should implement the methods \proc{init}(vector<int> stars) and \proc{makeMoves}(vector<int> ufos, vector<int> ships).
	Your \proc{init} method will be called once and can return any integer.
	Your \proc{makeMoves} method will be called until all stars have been travelled to or when you reached a maximum the NStar*4 turns.
	Which means, the \Emph{Data Generator} will continue to generate data for calling \proc{makeMoves}
	until one of previous two condition fits.
	Your \proc{makeMoves} method should return a \Emph{vector<int>} containing your space ship moves for a single turn.
	\begin{itemize}
		\item \id{stars} gives you the location of each star.
		
		The 2D location of the ith star is given by $(\id{stars}[i*2], \id{stars}[i*2+1])$.
		The range of values will be in $[0, 1023]$.
		
		\item \id{ufos} gives you the location of each ufo and the star indices where it is travelling towards in the next two turns.
		
		The star index of where the ith ufo is located, is given by $\id{ufos}[i*3]$.
		In the next move, the ith ufo will travel to $\id{ufos}[i*3+1]$ and it will travel to $\id{ufos}[i*3+2]$ after that.
		
		\item \id{ships} gives you the location of each space ship.
		
		The star index of where the ith ship is located, is given by $\id{ships}[i]$.
		A space ship can only travel directly between stars, therefor it's location is described by the star index where it is located.
		
	\end{itemize}
	You must return the list of moves for each space ship.
	The ith element of your return should give the zero based star index of the star where space ship i should travel towards.

\section{Scoring}	
\label{sec:scoring}

	For each test case we will calculate your raw and normalized scores.
	If you were not able to produce a valid return value, then your raw score is -1 and the normalized score is 0.
	Otherwise, the raw score is equal to the sum of energy used over all space ship moves.
	The normalized score for each test is $1,000,000.0 * BEST / YOUR$,
	where BEST is the lowest raw score currently obtained on this test case (considering only the last submission from each competitor).
	Finally, your total score is equal to the arithmetic average of normalized scores on all test cases.
	You can see your raw scores on each example test case by making an example submit.
	You can also see total scores of all competitors on provisional test set in the match standings.
	No other information about scores is available during the match.

	
\section{Clarifications}
\label{sec:clarifications}
	
	\begin{itemize}
	
		\item Space ships start at reandomly selected stars, these stars are not marked as visited
		initially and need to travelled during a turn.
		
		\item A space ship may remain stationary at the same star. (The star will be marked as visited).
		
		\item You can visit the same star multiple times.
		
		\item A maximum of $\Emph{Nstar}*4$ turns are allowed, thereafter you will score zero for the test.
		
		\item A space ship can travel from it's current star to any other star.
		
		\item \Emph{NStar} will be in the range of $[100, 200]$ (Except for seed 1).
		
		\item \Emph{NShip} will be in the range of $[1, 10]$.
		
		\item \Emph{NUfo} will be in the range of $[0, \Emph{NStar}/100)$.
		
		\item The range of the integer star coordinates is $[0, 1023]$.
		
		\item Stars are generated around a random number of galaxy centers with a \Emph{Gaussian} distribution.
		See the visualizer source code for exact implementation.
	
	\end{itemize}
	
\section{Notes}	
\label{sec:notes}

	\begin{itemize}
	
		\item The time limit is \Emph{20 seconds} per test case (this includes only the time spent in you code.)
		
		\item The memory limit is \Emph{1024 MB}.
		
		\item The implicit source code size limit is around \Emph{1 MB}.
		Once you code is compiled, the binary size should not exceed \Emph{1 MB}.
		
		\item The compilation time limit is \Emph{30 seconds}.
		
		\item There are \Emph{10} example test cases and \Emph{100} full submission (provisional) test cases.

		\item The match is \Emph{rated}.
		
	\end{itemize}
	
\section{Symbol Description}
\label{sec:symbol_descrip}

	$Star_k$表示第$k$个结点的坐标,
	$\proc{Length}$表示距离计算方法,
	$\proc{Length}(a, b)$表示点\id{a}到点\id{b}的距离。
	
	$\proc{Phop}(a, b)$UFO的第2跳为\id{a}并且第3跳为\id{b}的概率。
	
	$P_r$表示每次遇到UFO的衰减率,这里$P_r \equiv 0.001$。
	$\proc{Decay}(a, b)$表示从点\id{a}移动到\id{b}的衰减次数。
	
	$cost$表示路径花销,其实质为路径距离。
	$cturn$表示调用\proc{makeMoves}的花销,其实质为调用次数。
	
	$gain$表示路径花销的增益,即利用UFO相遇的情况,在同等条件经过相同路径可以减少的路径花销。
	
    $P$表示所有结点的集合,
	$U$表示已经访问结点的集合,
	$\bar{U}$表示未访问结点的集合,
	其本质是$\bar{U} = P \setminus U$。
	
\section{Thought}
\label{sec:thought}

\subsection{从局部最优出发}
\label{subsec:local_optimum}

\subsubsection{基本思路}

	我认为这又是一道设计概率的问题,显然当UFO存在的时候,选择与UFO相同的路径可以一定程度上减少花销。
	当然,这仅仅是个概率上的减少。题目中的距离采用的欧式距离,因此,这里一定存在一个这样的基本定理:
	\begin{mthm}
	\label{mthm:triangle}
		\[
            \proc{Length}(a, c) \ge \proc{Length}(a, b) + \proc{Length}(b, c)
        \]
	\end{mthm}
	这其实就是三角形定理,在不考虑与UFO路径重合的时候,这一定是正确的。
	
	如果存在UFO,UFO的移动趋势仅仅包含两跳的。因此,可以利用这个特性规划局部最优解。
	显然每次ship的选择共有如下几种情况:
	\begin{enumerate}[(1)]
		
		\item 原地不动
		
		\[
			cost_1 = 0
		\]
		
		\item 移动到未访问结点
		
		\[
			cost_1 = \proc{Length}(u, v), v \in Unvisited
		\]
		
		\item 访问到已访问结点,目的是为了更快的访问未访问结点(2跳)
		
		根据定理~\ref{mthm:triangle},显然由当前结点访问到下一结点,直到访问到未访问结点间,
		必然存在着一条路径与至少一个UFO重合。
		\[
			cost_1 = \proc{Length}(u, v) * P_r^k
		\]
	
	\end{enumerate}
	上面的等式仅仅考虑的是一跳的情况,这里存在着一个问题,显然这三种情况并不一定都会访问到未访问结点。
	所以,我认为应当以此为前提比较几种方案的优劣。显然,这个优劣是局部性的优劣,后续应当加入惩罚函数,
	以防止爬山。故,我们换一个思路讨论\Emph{cost}函数,这里\Emph{cost}应当表示为从当前结点访问到未访问结点
	的路径花销。不妨设当前结点为$u$,目标结点为$v$,故可分以下几种讨论
	\begin{enumerate}[(1)]
	
		\item 若$u = v$
		\begin{align*}
			cost	&=	0	\\
			cturn	&=	1
		\end{align*}
		
		\item 若仅仅通过一跳即可访问到$v$
		\begin{align*}
			cost	&= 	\proc{Length}(u, v) * P_r^{\proc{Decay}(u, v)}	\\
			cturn	&=	1
		\end{align*}
		
		\item 若通过两跳才能访问到$v$,并假定路径为$u \rightarrow a \rightarrow v$
		\begin{align*}
			cost	&=	\proc{Length}(u, a) * P_r^{\proc{Decay}(u, a)} + \proc{Length}(a, v) * P_r^{\proc{Decay}(a, v)}	\\
			cturn	&=	2
		\end{align*}
		
	\end{enumerate}
	显然,上述公式都是可计算的,因为每次给出UFO的接连两条的移动趋势。但是,一定存在着这样的可能,
	就是$v$到其他任何结点非常远,然而,经过连续若干次的拖延,直到第k次保证路径与UFO重合,显然
	这种情况存在并且可以使总花销更小。而这恰恰是个概率问题。
	又因为$|UFO| = |Star|/100$,显然两者的差异很大。故,我们不妨以一定概率计算到三跳甚至四跳的可能。
	不选择更多的跳数是因为受到总的跳数一定少于$|Star| \times 4$的限制。因此我们额外增加一种情况
	\begin{enumerate}[(1)]
	\setcounter{enumi}{2}
		
		\item 通过三跳访问到$v$,并假定路径为$u \rightarrow a \rightarrow b \rightarrow v$
		\begin{align*}
			cost	&= \proc{Length}(u, a) * P_r^{\proc{Decay}(u, a)} + \proc{Length}(a, b) * P_r^{\proc{Decay}(a, b)} + \proc{Length}(b, v) * P_r / \proc{Phop}(b, v)	\\
			cturn	&= 3
		\end{align*}
		
	\end{enumerate}
	
	显然将原问题转化为根据上述方案如何选择局部最优路径。
	
\subsubsection{存在的问题}

	仅从局部最优解出发,选择方案构建全局最优解,存在如下几个问题:
	\begin{enumerate}[(1)]
		
		\item 静态选择方案存在着不合理
		
		由于跳数不同,这也导致在每跳时,$\proc{Decay}$其实是不同的,这就引发了一个问题,决策时动态进行的还是静态执行的。
		静态执行显然一定可以得到解,但是解难免不如动态决策。而动态决策存在的主要问题是,前面进行的决策是否荒废了,
		是否可以保证一定能在限定的\Emph{turn}次数内得到解。
		
		\item $cost$函数计算的不科学
		
		显然每当结点$u$移动到结点$v$了,那么可能由某点距离其他结点都很远,仅仅距离结点\id{u}很近。按照点访问的顺序,
		可能先访问那个较远点更合理,诸如此类的情况如何选择惩罚项进行处理。
		
		此外,$cost$函数选择唯一点执行决策还是多个点并行进行,如果每次都选择唯一点进行,显然存在着饥饿的情况。
		那么,能不能保证在\Emph{turn}次数内得到解成为关键。
		
		\item  仅仅考虑了花销,却没有考虑增益
		
		显然,跳数越多访问的结点应该越多,那么未访问点的增加也应该越多。
		显然,经过若干个未访问点到达结点$v$与经过较少甚至0个未访问点到达结点
		$v$的权重或者优先级应该是不同的。
		
		同样的,访问越多的未访问点意味着减少其他结点访问该结点的机会。
		这在一定程度上会影响全局最优性。
		
	\end{enumerate}
	
\subsubsection{改进}

	这个\Emph{Base}算法需要改进的关键点在于\Emph{cost}的计算公式,
	显然,仅仅考虑花销而不考虑增益是不科学的。这导致了几种决策其实并不处在
	同等地位进行比较。
	
	改进后的计算公式为
	\begin{enumerate}[(1)]
		
		\item 1-hop
		
		\begin{align}
			cost &= \proc{Length}(u, v) * P_r^{\proc{Decay}(u, v)}	\\
			gain &= cost - \min_{u \in ships} {\proc{Length}(u, v)}
		\end{align}
		
		\item 2-hop
		
		\begin{align}
			cost &= \proc{Length}(u, a) * P_r^{\proc{Decay}(u, a)} + \proc{Length}(a, v) * P_r^{\proc{Decay}(a, v)}	\\
			gain &= cost - \begin{cases}
				\min_{u \in ships} {\proc{Length}(u, a) + \proc{Length}(a, v)} & \text{if } a \in \bar{U};	\\
				\min_{u \in ships} {\proc{Length}(u, v)} & \text{else}.	
			\end{cases}
		\end{align}
		
		\item 3-hop
		\begin{align}
			cost &= \proc{Length}(u, a) * P_r^{\proc{Decay}(u, a)} + \proc{Length}(a, b) * P_r^{\proc{Decay}(a, b)} + \proc{Length}(b, v) * P_r / \proc{Phop}(b, v)	\\
			gain &= cost - \begin{cases}
				\min_{u \in ships} {\proc{Length}(u, a) + \proc{Length}(a, b) + \proc{Length}(b, v)} & \text{if } a \in \bar{U} \kw{ and } b \in \bar{U};	\\
				\min_{u \in ships} {\proc{Length}(u, a) + \proc{Length}(a, v)} & \text{if } a \in \bar{U} \kw{ and } b \notin \bar{U};	\\
				\min_{u \in ships} {\proc{Length}(u, b) + \proc{Length}(b, v)} & \text{if } a \notin \bar{U} \kw{ and } b \in \bar{U};	\\
				\min_{u \in ships} {\proc{Length}(u, v)} & \text{else}.
			\end{cases}
		\end{align}
		
	\end{enumerate}
	
\subsubsection{总结}

	这个\Emph{Base}算法的主要缺点在于矫枉过正,显然每次决策考虑的是当前最优解。
	并不讨论全优解,而随着结点的饥饿程度的增加,显然局部最优解的得分呈现递增趋势,
	这导致全局最优解会随着$|Star|$数量的增加、$|Ship|$数量的减少而变得更差。
	
	同时,其实坐标指定了一个很小的范围,这意味着通过适当的选择阈值控制决策在一定程度
	上可以提高性能。不妨考虑如下情况:
	\begin{itemize}
	
		\item 将距离$1000 \times 0.001 = 1$,增益为$999$,花销为$1$
		
		\item 而将距离$1 \times 0.001 = 0.001$,增益为$0.999$,花销为$0.001$
		
	\end{itemize}
	
	显然对于距离已经足够小的点,其实并不需要刻意的与UFO碰撞,应当在最短的$turn$数量内
	访问该结点,显然由于距离足够小,那么即使从该结点返回起点也应当话费较小。
	然而,对于那么距离已经很远的结点,如果刻意借UFO的东风进行访问,将获得很大的增益。
	将很大程度上降低最终的得分。
	
	显然决策应该是动态的,而不应当是静态的。
	即每次决策不仅受到固有决策的影响,也应当受到随机的UFO的路径进行。
	即可能由于增益极度增加等情况,需要改变当前的策略。
	这显然类似于AI问题。
	
	还有个可能的问题是,$Star$的坐标可能重合,预处理时应该注意。
	
\subsection{遗传算法SA}
\label{sec:sa}

\subsubsection{Introduction}

	选择\Emph{遗传算法}解决这个问题的主要原因是
	\begin{itemize}
		
		\item 这是一个动态决策问题,和遗传、模拟退火等随机算法的模型很像
		
		\item 遗传的交叉、变异的策略很复杂,并没有完全掌握
		
		\item 感觉
		
	\end{itemize}
	
\end{document}
