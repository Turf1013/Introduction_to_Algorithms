%-*- coding:UTF-8 -*-
% 算法导论-第13章 红黑树.tex
\documentclass[UTF8]{ctexart}
\usepackage{geometry}
\usepackage{enumerate}
\usepackage{amsmath}
\usepackage{listings} %插入代码
\usepackage{xcolor} %代码高亮
\usepackage{diagbox}
\usepackage{tabularx}
\usepackage{graphicx}
\usepackage{caption}
\usepackage{subcaption}
\usepackage{float}

% Some setup
\pagestyle{plain}
\geometry{a4paper, top=2cm, bottom=2cm, left=2cm, right=2cm}
\CTEXsetup[format={\raggedright\bfseries\Large}]{section}
\lstset{numbers=left, %设置行号位置
        numberstyle=\small, %设置行号大小
        keywordstyle=\color{blue}, %设置关键字颜色
        commentstyle=\color{purple}, %设置注释颜色
        %frame=single, %设置边框格式
        escapeinside=``, %逃逸字符(1左面的键),用于显示中文
        breaklines, %自动折行
        extendedchars=false, %解决代码跨页时,章节标题,页眉等汉字不显示的问题
        %xleftmargin=2em,xrightmargin=2em, aboveskip=1em, %设置边距
        tabsize=4, %设置tab空格数
        showspaces=false %不显示空格
       }

% About math
\newcommand{\rmnum}[1]{\romannumeral #1}

\begin{document}

\title{\Huge算法导论习题\\}
\vspace{2cm}
\author{\Large曾宇祥\\SY1406122}
\date{}
\maketitle

%\newpage
\section*{第13章\quad红黑树}
\begin{enumerate}
    \item 13.1-2 \\
    解:\\
        均不是,若结点为红色不满足性质4;若结点为黑色不满足性质5.
		
	\item 13.1-3 \\
	解:\\
		仍旧是红黑树,满足性质1-5。
		
	\item 13.1-4 \\
	解:\\
		这个题目描述的不清楚,其实就是把红色结点和黑结点归并到一起组成一个新的结点。因此,\\
		当黑色父结点的子结点均为黑色时,度为2;	\\
		当黑色父结点仅含一个黑色子结点时,度为3;	\\
		当黑色父结点的两个子结点均为红色时,度为4.	\\
		由于红色结点全部被吸收,即仅余下黑结点。由红黑树性质5可知叶结点深度均相同。
	
	\item 13.1-5 \\
	解:\\
		可以先形式化验证, 考虑结点x的黑高为BH(x), 红高为RH(x), 而RH(x)<=BH(x).
		结点x到后代叶结点的最长路径为BH(x)+RH(x), 最短路径为BH(x). 因此,至多为2倍。\\
		再形象地考虑一下路径会是什么样的。\\
		设x的黑高为h,
		当x的颜色为黑色,则最长路径的着色为黑-红-黑-$\cdots$-红-黑-红-黑, 长度为(2h-2),
		最短路径的着色为黑-黑-$\cdots$-黑-黑,长度为(h-1), 比值为2。\\
		当x的颜色为红色,则最长路径的着色为红-黑-红-黑-$\cdots$-红-黑-红-黑, 长度为(2h-1),
		最短路径的着色为红-黑-黑-$\cdots$-黑-黑,长度为h, 比值趋近于2。\\
		因此,综上可知比值最多为2.
		
	\item 13.1-6 \\
	解:\\
		内部结点最多为$2^(2k)-1$, 显然为一棵完全数,每层结点颜色为黑-红交替。\\
		内部结点最少为$2^k-1$,显然为一棵全黑树。
		
	\item 13.1-7 \\
	解:\\
		比值最大为2:1, 比值最小为0(全黑树)。
		
	\item 13.2-2 \\
	证明:\\
		因为红黑树本身就没什么引理或者公理,因此想到使用数学归纳法证明。
		当n=1时,结论显然成立,因为无论左旋还是右旋,根本不会改变二叉搜索树。
		当n=k时,不妨假设当n<k时,恰有n-1种可能的旋转。仅需证明当n=k时,结论同样成立。\\
		从n个结点的二叉搜索树种任取一个叶子结点x,余下n-1个结点,存在n-2种可能的旋转 。
		这意味着x的父节点必将取便n-1个结点,否则对于n-1个结点,至少还有一种旋转。\\
		因此n个结点的旋转总数就是结点x与其父节点旋转后的可能总数(左旋还是右旋取决于两者大小),
		且这n-1种旋转互不相同,即为n-1。\\
		故结论成立。
		
	\item 13.2-3 \\
	解:\\
		a加1,b不变,c减1
		
	\item 13.2-4 \\
	证明:\\
		
	
\end{enumerate}


\end{document}

