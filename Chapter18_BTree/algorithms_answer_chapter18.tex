%-*- coding:UTF-8 -*-
% 算法导论-第18章 B树.tex
\documentclass[UTF8]{ctexart}
\usepackage{geometry}
\usepackage{enumerate}
\usepackage{amsmath}
\usepackage{listings} %插入代码
\usepackage{xcolor} %代码高亮
\usepackage{diagbox}
\usepackage{tabularx}
\usepackage{graphicx}
\usepackage{caption}
\usepackage{subcaption}
\usepackage{float}

% Some setup
\pagestyle{plain}
\geometry{a4paper, top=2cm, bottom=2cm, left=2cm, right=2cm}
\CTEXsetup[format={\raggedright\bfseries\Large}]{section}
\lstset{numbers=left, %设置行号位置
        numberstyle=\small, %设置行号大小
        keywordstyle=\color{blue}, %设置关键字颜色
        commentstyle=\color{purple}, %设置注释颜色
        %frame=single, %设置边框格式
        escapeinside=``, %逃逸字符(1左面的键),用于显示中文
        breaklines, %自动折行
        extendedchars=false, %解决代码跨页时,章节标题,页眉等汉字不显示的问题
        %xleftmargin=2em,xrightmargin=2em, aboveskip=1em, %设置边距
        tabsize=4, %设置tab空格数
        showspaces=false %不显示空格
       }

% About math
\newcommand{\rmnum}[1]{\romannumeral #1}

\begin{document}

\title{\Huge算法导论习题\\}
\vspace{2cm}
\author{\Large曾宇祥\\SY1406122}
\date{}
\maketitle

%\newpage
\section*{第18章\quad B树}
\begin{enumerate}
    \item 18.1-1 \\
	当t=1时,结点势必不包含任何元素。
	
	\item 18.1-2 \\
	当t=2或3时,图18-1所示的树是一棵合法的B树。
	
	\item 18.1-4 \\
	相当于每个结点含有$2t-1$个元素,每个结点共有$2t$的分支。则
	$n = (2t)^{h+1}-1$。
	
	\item 18.1-5 \\
	其实就是$t=2$时的B树。
	
	\item 18.2-2 \\
	仅想到了包含重复元素的时候,存在冗余读写的情况。
	
	\item 18.2-3 \\
	\begin{lstlisting}[language=c++]
	int BTree_Minimum(Node_t *x) {
		if (x->leaf)
			return x->key[1];
		else
			return BTree_Minimum(x->c[1]);
	}

	int BTree_Maximum(Node_t *x) {
		if (x->leaf)
			return x->key[x->n];
		else
			return BTree_Minimum(x->c[x->n+1]);
	}

	int BTree_Predecessor(Node_t *x, int i) {
		if (x->leaf) {
			if (i == 1)
				return -INF;
			else
				return x->key[i-1];
		} else {
			return BTree_Maximum(x->c[i]);
		}
	}

	int BTree_Successor(Node_t *x, int i) {
		if (x->leaf) {
			if (i == x->n)
				return INF;
			else
				return x->key[i+1];
		} else {
			return BTree_Minimum(x->c[i+1]);
		}
	}
	\end{lstlisting}
	
	\item 18.2-4 \\
	规律没找到。
	
	\item 18.2-5 \\
	在B-TREE-InSERT以及B-TREE-InSERT-NONFULL函数中都需要加入判断当前结点是否为叶子结点,
	若为叶子结点则根据$t'$值split。否则,依然按照旧$t$值split。
	
	\item 18.2-6 \\
	\[
		T(n) = O(\log t*\log_t n) = O(lgn)
	\]
	
	\item 18.3-2 \\
	\begin{lstlisting}[language=c++]
	void BTree_Delete(BTree_t *t, Node_t *x, int key) {
		int i = 1, j, k;
		Node_t *y, *z;

		while (i<=x->n && key>x->key[i])
			++i;
		if (i<=x->n && key==x->key[i]) {
			if (x->leaf) {
				--x->n;
				while (i <= x->n) {
					x->key[i] = x->key[i+1];
					++i;
				}
			} else {
				if (x->c[i]->n >= BTREE_TN) {
					// y is the predessor of x->c[]
					k = BTree_Predecessor(x, i);
					BTree_Delete(t, x->c[i], k);
					x->key[i] = k;
				} else if (i<=x->n && x->c[i+1]->n >= BTREE_TN) {
					k = BTree_Successor(x, i);
					BTree_Delete(t, x->c[i+1], k);
					x->key[i] = k;
				} else {
					BTree_Merge_Child(x, i);
					BTree_Delete(t, x->c[i], key);
				}
			}
		} else {
			if (x->c[i]->n == BTREE_TN-1) {
				y = x->c[i];
				if (i<=x->n && x->c[i+1]->n >= BTREE_TN) {
					z = x->c[i+1];
					// push x->key[i] to y
					++y->n;
					y->key[y->n] = x->key[i];
					x->key[i] = z->key[1];
					y->c[y->n+1] = z->c[1];
					// adjust x->c[i+1]
					--z->n;
					for (j=1; j<=z->n; ++j)
						z->key[j] = z->key[j+1];
					for (j=1; j<=z->n+1; ++j)
						z->c[j] = z->c[j+1];
				} else if (i>1 && x->c[i-1]->n >= BTREE_TN) {
					z = x->c[i-1];
					// push x->key[i] to y
					++y->n;
					for (j=y->n; j>1; --j)
						y->key[j] = y->key[j-1];
					y->key[1] = x->key[i-1];
					x->key[i-1] = z->key[z->n];
					for (j=y->n+1; j>1; --j)
						y->c[j] = y->c[j-1];
					y->c[1] = z->c[z->n+1];
					// adjust x->c[i-1];
					--z->n;
				} else {
					if (i <= x->n) {
						BTree_Merge_Child(x, i);
					} else {
						--i;
						BTree_Merge_Child(x, i);
					}
					if (x==t->root && x->n==0)
						t->root = x->c[i];
				}
			}
			BTree_Delete(t, x->c[i], key);
		}
	}	
	\end{lstlisting}
	
\end{enumerate}


\end{document}

