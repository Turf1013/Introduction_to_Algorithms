%-*- coding:UTF-8 -*-
% onlineExam.tex
\documentclass[hyperref,UTF8]{ctexart}
\usepackage{geometry}
\usepackage{enumerate}
\usepackage{amsmath}
\usepackage{amssymb}
\usepackage{dsfont}
\usepackage{amsthm}
\usepackage{listings} %插入代码
\usepackage{xcolor} %代码高亮
\usepackage{blkarray}
\usepackage{diagbox}
\usepackage{tabularx}
\usepackage{graphicx}
\usepackage{caption}
\usepackage{subcaption}
\usepackage{float}
\usepackage{color}
\usepackage{multirow}
\usepackage[all,pdf]{xy}
\usepackage{verbatim}   %comment
\usepackage{cases}
\usepackage{clrscode3e}	% need clrscode3e package which is not included in CTex.
\usepackage{hyperref}

\geometry{screen}
\hypersetup{
    colorlinks=true,
    bookmarks=true,
    bookmarksopen=false,
    % pdfpagemode=FullScreen,
    pdfstartview=fit,
    pdftitle={Online-Exam},
    pdfauthor={Trasier}
}

% THEOREM Environments --------------------------------------------------------
\newtheorem{thm}{Theorem}[subsection]
\newtheorem{cor}[thm]{Corollary}
\newtheorem{lem}[thm]{Lemma}
\newtheorem{prop}[thm]{Proposition}
\newtheorem{prob}[thm]{Problem}
\newtheorem{mthm}[thm]{Main Theorem}
\theoremstyle{definition}
\newtheorem{defn}[thm]{Definition}
\theoremstyle{remark}
\newtheorem{rem}[thm]{Remark}
\numberwithin{equation}{subsection}
% MATH ------------------------------------------------------------------------
\DeclareMathOperator{\RE}{Re}
\DeclareMathOperator{\IM}{Im}
\DeclareMathOperator{\ess}{ess}
\newcommand{\eps}{\varepsilon}
%\newcommand{\To}{\longrightarrow}  conflict with \package{clrscode3e}p
\newcommand{\h}{\mathcal{H}}
\newcommand{\s}{\mathcal{S}}
\newcommand{\A}{\mathcal{A}}
\newcommand{\J}{\mathcal{J}}
\newcommand{\M}{\mathcal{M}}
\newcommand{\W}{\mathcal{W}}
\newcommand{\X}{\mathcal{X}}
\newcommand{\BOP}{\mathbf{B}}
\newcommand{\BH}{\mathbf{B}(\mathcal{H})}
\newcommand{\KH}{\mathcal{K}(\mathcal{H})}
\newcommand{\Real}{\mathbb{R}}
\newcommand{\Complex}{\mathbb{C}}
\newcommand{\Field}{\mathbb{F}}
\newcommand{\RPlus}{\Real^{+}}
\newcommand{\Polar}{\mathcal{P}_{\s}}
\newcommand{\Poly}{\mathcal{P}(E)}
\newcommand{\EssD}{\mathcal{D}}
\newcommand{\Lom}{\mathcal{L}}
\newcommand{\States}{\mathcal{T}}
\newcommand{\abs}[1]{\left\vert#1\right\vert}
\newcommand{\set}[1]{\left\{#1\right\}}
\newcommand{\seq}[1]{\left<#1\right>}
\newcommand{\norm}[1]{\left\Vert#1\right\Vert}
\newcommand{\essnorm}[1]{\norm{#1}_{\ess}}


% Some setup
\pagestyle{plain}
\geometry{a4paper, top=2cm, bottom=2cm, left=2cm, right=2cm}
\CTEXsetup[format={\raggedright\bfseries\Large}]{section}
\lstset{numbers=left, %设置行号位置
        numberstyle=\small, %设置行号大小
        keywordstyle=\color{blue}, %设置关键字颜色
        commentstyle=\color{purple}, %设置注释颜色
        %frame=single, %设置边框格式
        escapeinside=``, %逃逸字符(1左面的键),用于显示中文
        breaklines, %自动折行
        extendedchars=false, %解决代码跨页时,章节标题,页眉等汉字不显示的问题
        %xleftmargin=2em,xrightmargin=2em, aboveskip=1em, %设置边距
        tabsize=4, %设置tab空格数
        showspaces=false %不显示空格
       }

% About math
\newcommand{\rmnum}[1]{\romannumeral #1}
\newcommand{\Emph}{\textbf}
\newcolumntype{Y}{>{\centering\arraybackslash}X}
\newcommand{\resetcounter}{\setcounter{equation}{0}}
\newcommand{\equsuf}[1][x]{\equiv_{\textit{Suff(#1)}}}	
\newcommand{\Suff}{\textit{Suff}}
\newcommand{\len}[1][x]{\textit{length}_{#1}}

% section deeep to 3 1.1.1
\setcounter{secnumdepth}{3}

\begin{document}

\title{\Huge OnlineExam}
\vspace{2cm}
\author{\Large Trasier}
\date{\today}
\maketitle

\section{Problem Statement}
\label{sec:problem_statement}
	
	There is an exam consists of $n = 5000$ questions each of which has only two possible answers: \Emph{yes} and \Emph{no}.
	You can attempt to pass the exam at most $x = 100$ times.
	The questions, their order and right answers don't change from one attempt to another.
	You can try to pass the exam in the following way:
	
	First, You fixes the answers for all questions.
	The result is a string of \id{n} bits: answers for the first, second, ..., \id{n}-th questions.
	In this string, \Emph{0} means that the answer is \Emph{no} and \Emph{1} is \Emph{yes}.
	Then he answers the questions according this string, spending one attempt.
	After that, you can fix another string of \id{n} bits and make another attempt, and so on until there are no more attempts.
	
	In the online system for exam, the following optimization is implemented:
	if at some moment of time, the examinee already got $k = 2000$ answers wrong, the attempt is immediately terminated with the corresponding message.
	Which means the remaining bits in the string are ignored.
	If there were strictly less than \id{k} wrong answers for all \id{n}, the exam is considered passed.
	
	The result of an attempt is a number from \id{k} to \id{n} inclusive: the number of questions after which the attempt was terminated.
	If the exam is passed, this number is considered to be $n + 1$.
	The exam result is the highest result among all attempts.
	If there were no attempts, the exam result is zero.
	
	You task is to write a program which will determine the bit strings.
	After each attempt, your program will get its result immediately.
	Try to get the highest exam result you can!
	

\section{Interaction Protocal}
	
	Your solution can make from $0$ to \id{x} attempts inclusive.
	To make an attempt, print a string to the standard output.
	The string must consist of exactly n binary digits without spaces and end with a newline character.
	
	To prevent output buffering, after printing a string, insert a command to flush the buffer:
	for example, it can be \Emph{fflush(stdout)} in C or C++, \Emph{System.out.flush()} in Java, \Emph{flush(output)} in Pascal or \Emph{sys.stdout.flush()} in Python.
	
	After each attempt you make, you can immediately read its result from the standard input.
	The result is an integer from \id{k} to $n+1$ inclusive, followed by a newline character.
	
\section{Scoring}	

	A test is defined by a string of \id{n} binart digits: the right answers to \id{n} questions.
	This string is kept secret from solution. Each test is evaluated separately.	\\
	
	If a solution followed the interaction protocol and terminated correctly on a test,
	it gets a score of $\max (0, S-4000)$ where \id{S} is the exam result.
	Otherwise, the solution gets zero score for the test.
	
\section{Testing}
	
	Your solution will be checked on sets of tests generated in advance.
	Each test is created using a pseudo-random number generator.
	You can consider the answers are uniformly distributed and mutually independent.
	A solution gets the score which is the sum of its score on all the tests.	\\
	
	During the main phase of the contest, there are two ways to send a solution for checking.
	\begin{itemize}
		
		\item The first one is the check on examples. There are $10$ example tests which are alson
		available for local testing. As soon as the solution is checked, you can sse reports for all
		examples by clicking on the submission result.
		
		\item The second way is to check on preliminary tests. There are $100$ preliminary tests which
		are generated in advance but kept secret. The score for prelimiary tests(but not for example tests)
		is used in the preliminary scoreboard. This scores does not affect the final results, but
		nevertheless allows to roughly compare a solution with others.
		
	\end{itemize}
	
	After the main phase ends, for each participant, the system chooses the final solution:
	\begin{itemize}
		
		\item consider all solutions sent for \Emph{preliminary testing}
		
		\item choose the ones which got a total score strictly greater than zero
		
		\item define the final solution as the one of chosen solutions which has \Emph{the latest} submission time
		
	\end{itemize}
	
	Note that the solutions sent only to be checked on examples are not considered when choosing the final solution.
	
	During the final testing, all final solutions will be checked on the same large set of a large
	number ($\approx 1000$) of final tests. The score for final tests determines the final scoreboard.
	The winner is the contestant whose solution gets the highest total score.
	In case two or more participants have equal total score, the contestants with such score tie for the same place.
	A package for local development is available on GitHub at the following address:
	\href{https://github.com/GassaFM/online-exam}{https://github.com/GassaFM/online-exam}.
	You can download sources or the latest release: \href{https://github.com/GassaFM/online-exam/releases}{https://github.com/GassaFM/online-exam/releases}.
	
\section{Notes}	

	\begin{itemize}
	
		\item The time limit is \Emph{5 seconds} per test case.
		
		\item The memory limit is \Emph{512 MB}.
		
		\item There are \Emph{10} example test cases, \Emph{100} preliminary test cases and \Emph{1000} final tests.

		\item The match is \Emph{unrated}.
		
	\end{itemize}

	
\section{Algorithm v0}
\label{sec:algo_v0}

\section{Basic Thoughts}

	\begin{mthm}
	假设$Ans_0 = S_1 \cdot S_2 \cdots S_k$,并且结果为$K_0$
	那么$Ans_1 = \hat{S_1} \cdot S_2 \cdots S_k$,并且结果为$K_1$。
	若$|S_1|$为奇数,那么$K_0 \neq K_1$。那么,选择最好的决策一定可以使得正确率在50\%以上。
	\end{mthm}
	
	这道理是显然的,因此关键问题是如何选择合适的SLICE进行切片。并且考虑到$|Ans| = 5000$,那么估计范围一定小于$[20, 50]$。
	通过测试可知较好的参数为41,此时的$K$的均值为$643.798$。这意味着此时的正确率可以达到
	$1.0 - 2000 \/ (4000 +  643.798) = 0.569244$。

\section{Existing Problem}
	
	\begin{enumerate}[(1)]
	
		\item 初始的切片长度是固定的,后面衰减,这导致了长度到达一定规模后。由于前面的错误答案已经很多,
		长度很难进一步增长。
		
		\item 采用上述贪心算法的问题在于所有的区间一定为一个固定值,缺少噪声。这导致最终的解得性能一定不够好。
		因为概率上是均匀分布的。
		
	\end{enumerate}
	
\section{Optimization Strategy}



\end{document}