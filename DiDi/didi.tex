%-*- coding:UTF-8 -*-
% 滴滴打车供需预测.tex
\documentclass[hyperref,UTF8]{ctexart}
\usepackage{geometry}
\usepackage{enumerate}
\usepackage{amsmath}
\usepackage{amssymb}
\usepackage{dsfont}
\usepackage{amsthm}
\usepackage{listings} %插入代码
\usepackage{xcolor} %代码高亮
\usepackage{blkarray}
\usepackage{diagbox}
\usepackage{tabularx}
\usepackage{graphicx}
\usepackage{caption}
\usepackage{subcaption}
\usepackage{float}
\usepackage{color}
\usepackage{multirow}
\usepackage[all,pdf]{xy}
\usepackage{verbatim}   %comment
\usepackage{cases}
\usepackage{clrscode3e}	% need clrscode3e package which is not included in CTex.
\usepackage{hyperref}

\geometry{screen}
\hypersetup{
    colorlinks=true,
    bookmarks=true,
    bookmarksopen=false,
    % pdfpagemode=FullScreen,
    pdfstartview=fit,
    pdftitle={DiDi},
    pdfauthor={Trasier}
}

% THEOREM Environments --------------------------------------------------------
\newtheorem{thm}{Theorem}[subsection]
\newtheorem{cor}[thm]{Corollary}
\newtheorem{lem}[thm]{Lemma}
\newtheorem{prop}[thm]{Proposition}
\newtheorem{prob}[thm]{Problem}
\newtheorem{mthm}[thm]{Main Theorem}
\theoremstyle{definition}
\newtheorem{defn}[thm]{Definition}
\theoremstyle{remark}
\newtheorem{rem}[thm]{Remark}
\numberwithin{equation}{subsection}
% MATH ------------------------------------------------------------------------
\DeclareMathOperator{\RE}{Re}
\DeclareMathOperator{\IM}{Im}
\DeclareMathOperator{\ess}{ess}
\newcommand{\eps}{\varepsilon}
%\newcommand{\To}{\longrightarrow}  conflict with \package{clrscode3e}p
\newcommand{\h}{\mathcal{H}}
\newcommand{\s}{\mathcal{S}}
\newcommand{\A}{\mathcal{A}}
\newcommand{\J}{\mathcal{J}}
\newcommand{\M}{\mathcal{M}}
\newcommand{\W}{\mathcal{W}}
\newcommand{\X}{\mathcal{X}}
\newcommand{\BOP}{\mathbf{B}}
\newcommand{\BH}{\mathbf{B}(\mathcal{H})}
\newcommand{\KH}{\mathcal{K}(\mathcal{H})}
\newcommand{\Real}{\mathbb{R}}
\newcommand{\Complex}{\mathbb{C}}
\newcommand{\Field}{\mathbb{F}}
\newcommand{\RPlus}{\Real^{+}}
\newcommand{\Polar}{\mathcal{P}_{\s}}
\newcommand{\Poly}{\mathcal{P}(E)}
\newcommand{\EssD}{\mathcal{D}}
\newcommand{\Lom}{\mathcal{L}}
\newcommand{\States}{\mathcal{T}}
\newcommand{\abs}[1]{\left\vert#1\right\vert}
\newcommand{\set}[1]{\left\{#1\right\}}
\newcommand{\seq}[1]{\left<#1\right>}
\newcommand{\norm}[1]{\left\Vert#1\right\Vert}
\newcommand{\essnorm}[1]{\norm{#1}_{\ess}}


% Some setup
\pagestyle{plain}
\geometry{a4paper, top=2cm, bottom=2cm, left=2cm, right=2cm}
\CTEXsetup[format={\raggedright\bfseries\Large}]{section}
\lstset{numbers=left, %设置行号位置
        numberstyle=\small, %设置行号大小
        keywordstyle=\color{blue}, %设置关键字颜色
        commentstyle=\color{purple}, %设置注释颜色
        %frame=single, %设置边框格式
        escapeinside=``, %逃逸字符(1左面的键),用于显示中文
        breaklines, %自动折行
        extendedchars=false, %解决代码跨页时,章节标题,页眉等汉字不显示的问题
        %xleftmargin=2em,xrightmargin=2em, aboveskip=1em, %设置边距
        tabsize=4, %设置tab空格数
        showspaces=false %不显示空格
       }

% About math
\newcommand{\rmnum}[1]{\romannumeral #1}
\newcommand{\Emph}{\textbf}
\newcolumntype{Y}{>{\centering\arraybackslash}X}
\newcommand{\resetcounter}{\setcounter{equation}{0}}
\newcommand{\equsuf}[1][x]{\equiv_{\textit{Suff(#1)}}}	
\newcommand{\Suff}{\textit{Suff}}
\newcommand{\len}[1][x]{\textit{length}_{#1}}

% section deeep to 3 1.1.1
\setcounter{secnumdepth}{3}

\begin{document}

\title{\Huge 滴滴打车供需预测}
\vspace{2cm}
\author{\Large Trasier}
\date{\today}
\maketitle

\section{Introduction}
\label{sec:introduction}

	在出行问题上,中国市场人数多、人口密度大,总体的出行频率远高于其他国家,这种情况在大城市尤为明显。然而,截至目前中国拥有汽车的人口只有不到10%,这意味着
在中国人们的出行更加依赖于出租车、公共交通等市场提供的服务。另一方面,滴滴出行占领了国内绝大部分的网络呼叫出行市场,面对着巨大的数据量以及与日俱增的数据处理
需求。截至目前,滴滴出行平台日均需处理1100万订单,需要分析的数据量达到50TB,路径规划服务请求超过90亿。面对如此庞杂的数据,我们需要通过不断升级、完善与创新
背后的云计算与大数据技术,从而保证数据分析及相关应用的稳定,实现高频出行下的运力均衡。供需预测就是其中的一个关键问题。
供需预测的目标是准确预测出给定地理区域在未来某个时间段的出行需求量及需求满足量。调研发现,同一地区不同时间段的订单密度是不一样的,例如大型居住区在早高峰
时段的出行需求比较旺盛,而商务区则在晚高峰时段的出行需求比较旺盛。如果能预测到在未来的一段时间内某些地区的出行需求量比较大,就可以提前对营运车辆提供一些引
导,指向性地提高部分地区的运力,从而提升乘客的整体出行体验。

\section{Problem Statement}
\label{sec:prob_statement}

\subsection{问题定义}	
\label{subsec:prob_definition}
	
	乘客打开滴滴出行app,输入出发地和目的地并点击“呼叫”后就完成一次发单(request),有司机接单后就完成一次应答(answer)。
将一个城市划分为\id{n}个互不重叠的正方形区域$D = \{d_1, d_2, \cdots, d_n\}$,
将每一天的24小时划分为144个10分钟长的时间片$t_1, t_2, \cdots, t_{144}$。
对于区域$d_i$,在时间片$t_j$,有$r_{ij}$个乘客发单,有$a_{ij}$个司机成功应答了$a_{ij}$次发单。
对于区域$d_i$,在时间片$t_j$,定义需求$demand_{ij} = r_{ij}$,
供给$supply_{ij} = a_{ij}$,则有供需缺口$gap_{ij}:\ gap_{ij} = r_{ij} - a_{ij}$。
给定每个区域在时间片$t_j, t_{j-1}, \cdots$的各项数据,预测$gap_{i,j+1}, \forall d_i \in D$。
	
	
\subsection{评价指标}
\label{subsec:evaluate_indicator}

	对$n$个区域和$q$个时间片,区域$d_i$在时间片$t_j$的供需缺口为$gap_{ij}$,选手预测值为$s_{ij}$,
以\id{MAPE}作为最终的评价指标:
\[
	\id{MAPE} = \frac{1}{n} \sum_{i=1}^{n}{\Bigg( \frac{1}{q}\sum_{j=1}^{144}\bigg| \frac{gap_{ij} - s_{ij}}{gap_{ij}} \bigg| \Bigg)}
\]
\id{MAPE}越小越好。


\subsection{提交结果格式}
\label{subsec:submit_format}

	选手提交的数据格式为:区域ID,时间片,预测值。其示例如下:\\
	\begin{tabular}{lll}
		1, &2016-01-23-1, &30.0 	\\
		1, &2016-01-23-4, &5.0 		\\
		1, &2016-01-23-10, &6.0 	\\
		2, &2016-01-23-1, &30.0 	\\
		2, &2016-01-23-4, &5.0 	\\
	\end{tabular}
	
	
	其中每个字段的具体描述如下:
	\begin{table}[H]
    \centering
	\caption{提交结果格式说明}
	\begin{tabular}{|c|c|c|}
    \hline
		数据名称	&	数据类型	&	示例							\\
    \hline
		区域ID		&	string		&	1,2,3,4(与区域映射ID一致)		\\
		时间片		&	string		&	2016-01-23-1(即2016年1月23日第1个时间片) \\
		预测值		&	double		&	6.0	\\
    \hline
	\end{tabular}
	\end{table}
	
\subsection{数据形式}
\label{subsec:data_format}

	训练集中给出M市2016年连续三周的数据信息,需预测M市第四周和第五周中某五天的某些时间段的供需。
测试集中给出了每个需预测的时间片的前半小时的数据信息,具体需预测的时间片见说明文件
(说明文件含在数据集下载包内)。具体数据如下,其中订单信息表、天气信息表和POI信息表为数据库中直接的表信息,
而区域定义表、拥堵信息表是由数据库中其他表衍生的信息。

	\begin{table}[H]
    \centering
	\caption{订单信息表}
	\begin{tabular}{|c|c|c|c|}
	\hline
	字段	&	类型	&	含义	&	示例		\\
	\hline
	order\_id & string & 订单ID & 70fc7c2bd2caf386bb50f8fd5dfef0cf 		\\
	driver\_id & string & 司机ID & 56018323b921dd2c5444f98fb45509de		\\
	passenger\_id & string & 用户ID & 238de35f44bbe8a67bdea86a5b0f4719	\\
	start\_district\_hash & string & 出发地区域哈希值 & d4ec2125aff74eded207d2d915ef682f	\\
	dest\_district\_hash & string & 目的地区域哈希值 & 929ec6c160e6f52c20a4217c7978f681		\\
	Price & double & 价格 & 37.5	\\
	Time & string & 订单时间戳 & 2016-01-15 00:35:11	\\
	\hline
	\end{tabular}
	\end{table}
	
	订单信息表主要覆盖了一张订单的基本信息,包括这张订单的乘客,以及接单的司机
($driver\_id = \const{NIL}$表示$driver\_id$为空,即这个订单没有司机应答),及出发地,目的地,价格和时间。
	
	\begin{table}[H]
    \centering
	\caption{区域定义表}
	\begin{tabular}{|c|c|c|c|}
	\hline
	字段	&	类型	&	含义	&	示例		\\
	\hline
	district\_hash & string & 区域哈希值 & 90c5a34f06ac86aee0fd70e2adce7d8a	\\
	district\_id & string & 区域映射ID & 1	\\
	\hline
	\end{tabular}
	\end{table}
	
	区域定义表主要表示比赛评测区域的信息,选手需选择区域定义表中的区域来做预测,
并在最终提交的结果中需将区域哈希值映射为其相应的ID。

	\begin{table}[H]
    \centering
	\caption{POI信息表}
	\begin{tabular}{|c|c|c|c|}
	\hline
	字段	&	类型	&	含义	&	示例		\\
	\hline
	district\_hash & string & 区域哈希值 & 74c1c25f4b283fa74a5514307b0d0278	\\
	poi\_class & string & POI类目及其数量 & 1\#1:41 2\#1:22 2\#2:32			\\
	\hline
	\end{tabular}
	\end{table}
	
	POI信息表主要表征区域的地域属性,由其中所含的不同类别设施的数量表示,
如2\#1:22表示在此区域中含有类别为2\#1的设施22个,2\#1表示一级类别为2,二级类别为1,例如
休闲娱乐\#剧院,购物\#家电数码,运动健身\#其他等等。不同类别及其数量以$\verb|\|$t分割。
	
	\begin{table}[H]
    \centering
	\caption{拥堵信息表}
	\begin{tabular}{|c|c|c|c|}
	\hline
	字段	&	类型	&	含义	&	示例		\\
	\hline
	district\_hash & string & 区域哈希值 & 1ecbb52d73c522f184a6fc53128b1ea1		\\
	tj\_level & string & 不同拥堵程度的路段数 & 1:231 2:33 3:13 4:10			\\
	tj\_time & string & 时间戳 & 2016-01-15 00:35:11							\\
	\hline
	\end{tabular}
	\end{table}
	
	拥堵信息表主要表示区域中道路的总体拥堵情况,其中主要包括不同时间段不同区域的不同拥堵情况的路段数,
其中的拥堵级别是越大越拥堵。

	\begin{table}[H]
    \centering
	\caption{天气信息表}
	\begin{tabular}{|c|c|c|c|}
	\hline
	字段	&	类型	&	含义	&	示例		\\
	\hline
	Weather & int & 天气 & 7	\\
	temperature & double & 温度 & -9	\\
	PM2.5 & double & pm2.5 & 66	\\
	Time & string & 时间戳 & 2016-01-15 00:35:11	\\
	\hline
	\end{tabular}
	\end{table}
	
	天气信息表主要表示整个城市的每天间隔10分钟段的天气情况。其中的weather字段表示天气的实时描述信息,
而温度以摄氏温度表示,PM2.5为实时空气污染指数。

	
\subsection{数据集}	
\label{sec:data_set}

\subsubsection{order\_data}

	\Emph{order}表是整个数据的核心,其中存在如下问题和思考:
	\begin{enumerate}[(1)]
		
		\item 数据中的问题是同一个乘客连续发单,但是均未得到响应,那么gap应该算是1还是N?
		
		
		\item 区域的趋势
		
		区域中每个时间片的需求情况,区域中周末和工作日的需求情况;	\\
		区域中每个时间片的供求情况,区域中周日和工作日的供求情况。
		
		\item 乘客的趋势
		
		乘客每天的需求情况:哪个区域、哪个时间段需要打的
		
		\item 司机的趋势
		
		司机的等活策略:哪个区域、哪个时间段等活
		
	\end{enumerate}
	
\subsubsection{cluster\_map}

	区域哈希值到Id值得映射。

\subsubsection{poi\_data}

	Naive算法不使用任何地理数据。
	\Emph{POI}数据中的entry特别多,需要处理(仅仅使用大类等方式)。按照一般性认识,基础设施不同的区域供求量应该是不同的。有待验证。

\subsubsection{traffic\_data}

	拥堵信息仅仅包含4中不同拥堵程度,可能与供求有关。

\subsubsection{weather\_data}

	Naive算法不使用任何天气数据。
	\begin{enumerate}[(1)]
		
		\item 天气差可能会导致订单的增加
		
		\item 节假日会导致订单的增加
		
	\end{enumerate}
	

\section{现有算法}
\label{sec:current_algo}



\section{我的算法}
\label{sec:basic_thought}

\subsection{v0}

	Naive算法仅从订单信息表、区域定义表的数据出发,去推导供需关系。
	即Naive算法从区域的角度思考问题,我们假定每个区域每个时间周期的趋势应当是相似的。即存在一条曲线可以拟合这个趋势,
	那么对于给定的测试集,带入到曲线中求解,即为所求。
	
\subsubsection{基本思路}

	\begin{thm}
	\label{thm:v0_district_trend}
	每个区域每周的需求量趋势应当相似,每个区域对应每天的需求量趋势应当相似。
	\end{thm}
	
	定理~\ref{thm:v0_district_trend}主要描述的从每个区域每天的供求量应当是类似的,更细粒度的说。每个区域每个工作日的变化
	趋势是类似的,周末的变化趋势是类似的。主定理是否正确可以由数据导出图证明。
	可能影响主定理正确性的原因:
	\begin{enumerate}[(1)]
	
		\item 天气的影响
		
	\end{enumerate}

	\begin{prop}
	\label{prop:v0_time_slice_trend}
	每个区域每个时间片的需求量趋势是否相似?
	\end{prop}
	
	命题~\ref{prop:v0_time_slice_trend}的主要依据是显然上班和下班时间打的的人群应当是有规律的。那么,其他时间片的供求量是否
	也呈一定的规律?
	
	\begin{mthm}
	\label{mthm:v0_basis}
	Naive算法的主要依据是每个区域的供求关系具有一定规律。即每个slice的供给趋势与需求趋势类似。
	\end{mthm}
	
	由~\ref{mthm:v0_basis}可知假定我们已经获得了供给和需求其中之一的数学模型,那么我们一定可以得到另外一个的数学模型。
	而二者之差恰恰为\Emph{gap}所以问题转化为对于给定的区域。而每周内的需求曲线是类似的。
	因此,我们可以分别对于每个区域计算属于该区域的模型。
	
	即对于所有区域选择如下feature计算回归模型:
	\begin{itemize}
		\item slice\_id
		\item day\_id
		\item request
		\item answer
	\end{itemize}
	
	先预估测试集的request数量,再带入answer模型求出测试集的answer数量,两者相减计算出gap。
	
\subsubsection{我的疑问}

	\begin{enumerate}[(1)]
		
		\item 数据中显然存在$r_{ij} >= a_{ij}$的情况,那么是否存在$r_{ij} < a_{ij}$的情况?
		
		即多个司机响应了同一张订单。
		
		\item 假定同一个乘客连续发送请求均得不到相应,那么这样一个gap只计算一次还是多次?
		
	\end{enumerate}
	
	
\end{document}
