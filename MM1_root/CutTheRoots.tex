%-*- coding:UTF-8 -*-
% link-cut.tex
\documentclass[UTF8]{ctexart}
\usepackage{geometry}
\usepackage{enumerate}
\usepackage{amsmath}
\usepackage{amssymb}
\usepackage{dsfont}
\usepackage{amsthm}
\usepackage{listings} %插入代码
\usepackage{xcolor} %代码高亮
\usepackage{blkarray}
\usepackage{diagbox}
\usepackage{tabularx}
\usepackage{graphicx}
\usepackage{caption}
\usepackage{subcaption}
\usepackage{float}
\usepackage{color}
\usepackage{multirow}
\usepackage[all,pdf]{xy}
\usepackage{verbatim}   %comment
\usepackage{cases}
\usepackage{clrscode3e}	% need clrscode3e package which is not included in CTex.

% THEOREM Environments --------------------------------------------------------
\newtheorem{thm}{Theorem}[subsection]
\newtheorem{cor}[thm]{Corollary}
\newtheorem{lem}[thm]{Lemma}
\newtheorem{prop}[thm]{Proposition}
\newtheorem{prob}[thm]{Problem}
\newtheorem{mthm}[thm]{Main Theorem}
\theoremstyle{definition}
\newtheorem{defn}[thm]{Definition}
\theoremstyle{remark}
\newtheorem{rem}[thm]{Remark}
\numberwithin{equation}{subsection}
% MATH ------------------------------------------------------------------------
\DeclareMathOperator{\RE}{Re}
\DeclareMathOperator{\IM}{Im}
\DeclareMathOperator{\ess}{ess}
\newcommand{\eps}{\varepsilon}
%\newcommand{\To}{\longrightarrow}  conflict with \package{clrscode3e}p
\newcommand{\h}{\mathcal{H}}
\newcommand{\s}{\mathcal{S}}
\newcommand{\A}{\mathcal{A}}
\newcommand{\J}{\mathcal{J}}
\newcommand{\M}{\mathcal{M}}
\newcommand{\W}{\mathcal{W}}
\newcommand{\X}{\mathcal{X}}
\newcommand{\BOP}{\mathbf{B}}
\newcommand{\BH}{\mathbf{B}(\mathcal{H})}
\newcommand{\KH}{\mathcal{K}(\mathcal{H})}
\newcommand{\Real}{\mathbb{R}}
\newcommand{\Complex}{\mathbb{C}}
\newcommand{\Field}{\mathbb{F}}
\newcommand{\RPlus}{\Real^{+}}
\newcommand{\Polar}{\mathcal{P}_{\s}}
\newcommand{\Poly}{\mathcal{P}(E)}
\newcommand{\EssD}{\mathcal{D}}
\newcommand{\Lom}{\mathcal{L}}
\newcommand{\States}{\mathcal{T}}
\newcommand{\abs}[1]{\left\vert#1\right\vert}
\newcommand{\set}[1]{\left\{#1\right\}}
\newcommand{\seq}[1]{\left<#1\right>}
\newcommand{\norm}[1]{\left\Vert#1\right\Vert}
\newcommand{\essnorm}[1]{\norm{#1}_{\ess}}


% Some setup
\pagestyle{plain}
\geometry{a4paper, top=2cm, bottom=2cm, left=2cm, right=2cm}
\CTEXsetup[format={\raggedright\bfseries\Large}]{section}
\lstset{numbers=left, %设置行号位置
        numberstyle=\small, %设置行号大小
        keywordstyle=\color{blue}, %设置关键字颜色
        commentstyle=\color{purple}, %设置注释颜色
        %frame=single, %设置边框格式
        escapeinside=``, %逃逸字符(1左面的键),用于显示中文
        breaklines, %自动折行
        extendedchars=false, %解决代码跨页时,章节标题,页眉等汉字不显示的问题
        %xleftmargin=2em,xrightmargin=2em, aboveskip=1em, %设置边距
        tabsize=4, %设置tab空格数
        showspaces=false %不显示空格
       }

% About math
\newcommand{\rmnum}[1]{\romannumeral #1}
\newcommand{\Emph}{\textbf}
\newcolumntype{Y}{>{\centering\arraybackslash}X}
\newcommand{\resetcounter}{\setcounter{equation}{0}}
\newcommand{\equsuf}[1][x]{\equiv_{\textit{Suff(#1)}}}	
\newcommand{\Suff}{\textit{Suff}}
\newcommand{\len}[1][x]{\textit{length}_{#1}}

% section deeep to 3 1.1.1
\setcounter{secnumdepth}{3}

\begin{document}

\title{\Huge CutTheRoots}
\vspace{2cm}
\author{\Large Trasier}
\date{\today}
\maketitle

\section{Problem Statement}
\label{sec:problem_statement}
	
	Several seeds were planted in one jar.
	To separate the plants, you need to cut the soil in the jar.
	Each cut is modeled as an \Emph{infinite} staight line which goes through points(X1, Y1) and (X2, Y2).
	Cut lines cut every root they intersect.
	Two plants are considered to be separated as long as there exists at least one line which separates their base points.
	
	You need implement a single method \Emph{int[] makeCuts(int NP, int[] points, int[] roots)}.
	The paramenters give the following information:
	\begin{itemize}
	
		\item \Emph{NP}: the number of plants.
		
		\item \Emph{points}: the coordinates of endpoints of roots.
			Endpoint i (0-based) has coordinates (points$[2*i]$, points$[2*i+1]$).
			Note that there will be more than \Emph{NP} points in this array;
			the first \Emph{NP} points wil be base points of plants.
			
		\item \Emph{roots}:	represend with indice of their endpoints,
			Root j (0-based) connects endpoints root$[2*j]$ and root$[2*j+1]$.
			It is guaranteed that each root endpoint belongs to exactly one plant.
			Root of one plant or different plants can intersect each other, but not share an endpoint.
			Due the code of generate, root$[2*j]$ is exists points, root$[2*j+1]$ is new point.
		
	\end{itemize}
	
	You must return a vector<int> represenint a set at most $4 * NP$ lines which separate all plants.
	If you return describes L lines, it must contain exactly $4 * L$ elementts.

\section{Notes}	
\label{sec:notes}

	\begin{itemize}
	
		\item Time limit is \Emph{10 seconds} and the memory limit is \Emph{1024MB} for a single test case.
		
		\item $\Emph{NP} \in [5, 105]$, $\Emph{|roots|} \in [50, 105000]$.
		
		\item All coordinates will faill into the range $[0, 1024]$.
		
	\end{itemize}
	
	
\section{Thought}	

	基于思路是尽量保持原有植物的外接轮廓。基本算法是贪心+二分。
	\begin{enumerate}[(1)]
	
		\item 如何判定每条根属于哪个植物?
		
		并查集,将每个root连接的两点看成一条边,构建图,前$NP$个点一定是各自的祖先。
		
		\item 如何表示植物及其根?
		
		显然的思路是对每个植物,构建凸包,然后求外接多边形或者外接圆进行运算。
		这个思路存在的问题是没有把祖先结点当做关键点。
		因此,采用求所有root到祖先结点的距离,取\Emph{中位数}构建外接圆。
		使用圆表示每棵植物。
		
		\item 切割基于什么理论?
		
		假定我们有一个最优的切割,那么再加任何一次切割的行为必然是一种\Emph{画蛇添足}的行为。
		显然,我们随着切割的增加,保留的根的数目单调递减。故,这里是可以采用二分的。
		然而,这个二分的关键在于上下限如何确定以及如何切割?
		
		\item 如何切割?
		
		切割的思路可以看成将元素进行分类,使用线性分类器对元素分类。
		这个线性分类器的模型是一条直线,者很糟糕,导致很多分类算法都不能直接适用。
		因此,我们适用启发式分割。每次,取两个规模最大的圆。
		
		\begin{enumerate}[(1)]
			
			\item 相交
			
			过两个交点的中点做直线$y=kx+b$,使得$W$函数最大。
			
			\item 相离
			
			过两条内切线的交点做直线$y=kx+b$,使得$W$函数最大。
			
			\item 内切
			
			过切点做直线$y=kx+b$,使得$W$函数最大。
			
			\item 内含
			
			过两个圆心做线段与圆的交点,过交点做直线$y=kx+b$,使得$W$函数最大。
			
			\item 重合
			
			不存在
			
		\end{enumerate}
		
		$W$函数描述的是所有点到直线的距离,我们希望这个距离越大越好,这样被切掉的root近似最少。
		
		点$(x0, y0)$到直线$Ax+By+C=0$的距离为
		\[
			d = \frac{|Ax0 + By0 + C|}{\sqrt{A^2+B^2}}
		\]
		直线$y=kx+b$过点$x0, y0$,可得直线$k(x-x0)-y+y0=0$。
		\begin{align*}
			W	&= \sum_{i=1}^{np}\sum_{j=1}^{|Root_i|} d^2	\\
				&= \sum_{i=1}^{np}\sum_{j=1}^{|Root_i|} \frac{|k(x_{ij}-x0) - y_{ij} + y0|^2}{k^2+1}	\\
				&= \frac{Ak^2 + Bk + C}{k^2+1}	\\
				&= A + \frac{Bk + C - A}{k^2+1}
		\end{align*}
		若$B = 0$时,取$x = - \frac{C}{A}$
		若$B \neq 0$时,不妨令$Y = \frac{Bk + C - A} {k^2+1} = \frac{Bk + D} {k^2+1}$。
		对该式求导可知
		\begin{align*}
			dY 	&= \frac{B(k^2+1) - (Bk+D)*2k} {k^2+1}	\\
				&= - \frac{Bk^2 + 2Dk - B} {k^2+1}	\\
			\Delta	&= (2D)^2 + 4 \times B \times B	\\
					&= 4(D^2 + B^2)
		\end{align*}
		显然一定存在两点$k_0, k_1$使得导数为0。
		因此,最优解的取值范围是$[-\inf, k0], k1$。
		
		\item 如何动态修改?
		
		当划定一次分割后,断掉所有被切割的root点。
		同时由于分割即分类,即对所有植物进行了一次二分类。
		从而,可以对每个类别递归的进行操作\proc{CutTheRoots}。
		
		则$L = 2*L(n/2) + 1$,这里$L = [\log(n), n]$
		
		\item 如何判定植物被分割?
		
		任意base点对至少被一条直线分割。时间复杂度$O(n ^ 3)$.
		
		\item 如何打补丁?
		
		当函数执行完成后,扫描每个点对,假设$p0, p1$没有被分割。abort()
		% 这种情况可能发生么?
		
		\item 如何参数化?
		
		中点和$k$的范围均可以参数化,
		选择最大的两个圆也可以随机化。
		
	\end{enumerate}
	
\end{document}

